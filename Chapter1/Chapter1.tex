%This is the first chapter of the dissertation

%The following command starts your chapter. If you want different titles used in your ToC and at the top of
%the page throughout the chapter, you can specify those values here. Since Columbia doesn't want extra
%information in the headers and footers, the "Top of Page Title" value won't actually appear.

%By using the asterisk to start a new section, I keep the section from appearing in the table of contents.
%If you want your sections to be numbered and to appear in the table of contents, remove the asterisk.



\pagestyle{cu}
\graphicspath{{./Chapter1/Figures/}}
\chapter[Dark Matter][Dark Matter]{Dark Matter}
\label{chap:dark_matter}

For much of the last century evidence has been building for two non-luminous components of our universe that cannot be explained with
our current model of physics.  The first is the acceleration in the expansion of the universe, which is hypothesized to result from an
unknown kind of energy known as dark energy.  Measurements estimate dark energy composes ${\sim}70$\% of
matter in the current observable universe.  The second is the presence of a gravitational field that many times too large to be caused
by ordinary - or baryonic - matter, but whose source cannot be seen.  Dark matter is hypothetical form of matter that does not interact
electromagnetically and provides the additional massive component that would solve this discrepancy.  Observations predict dark matter
constitutes ${\sim}26$\% of the universe.  For both the term ``dark'' is a relic from their early histories and today both are understood
to be invisible.

This chapter presents an overview on dark matter and the different methods of detection.  It begins with a summary of the
$\Lambda \mathrm{CDM}$ model (\secref{sec:dark_matter_lambda_cdm}) followed by a discussion on evidence of dark matter's existence
(\secref{sec:dark_matter_evidence}) and potential candidates (\secref{sec:dmcandidates}).  Next detection methods for Weakly
Interacting Massive Particle (WIMP) dark matter are detailed (\secref{sec:detection}), and the final section covers the various
categories of direct detection experiments (\secref{sec:direct_detect}).



%====================================
\section[$\Lambda$CDM Model][$\Lambda$ CDM Model]{$\boldsymbol{\Lambda}$CDM Model}
\label{sec:dark_matter_lambda_cdm}
The $\Lambda \mathrm{CDM}$ model describes the evolution of the universe from its inception at the Big Bang to present day.  It is named
from its inclusion of a cosmological constant (dark energy) $\Lambda$ and cold dark matter (CDM), and is the most successful model to
date.  Its validity is dependent on our universe being homogeneous and isotropic over large (${\sim} 100\ \mathrm{Mpc}$) distances,
Einstein's General Relativity, and a path connected universe - all of which are believed to be true.  The model can be solved for the
Robertson-Walker metric, which in polar coordinates gives

\begin{equation}
ds^{2} = -c^{2}dt^{2} + a(t)^{2} \bigg( \frac{dr^{2}}{1 - kr^{2}} + r^{2}d\Omega^{2} \bigg)
\label{eq:dark_matter_lambda_cdm_rw_metric}
\end{equation}

\noindent where $ds$ is the distance traversed in space-time, $a(t)$ is the scale factor with $a=1$ today, and $k \in \{-1, 0, 1\}$ is the
curvature of the
universe.  A universe with $k = -1$ is an open universe and has negative curvature.  Its geometry is hyperbolic, meaning the
sum of the angles in a triangle is $< 180^{\circ}$ and two lines that do not cross are never equidistant.  An open universe will expand
forever as long as $\Lambda \geq 0$ (observations disqualify a universe with negative $\Lambda$).  A universe with $k = 0$ is flat and
thus has Euclidean geometry (angles in a triangle equal $180^{\circ}$ and parallel lines remain equidistant).  If $\Lambda = 0$ it will
expand forever at a decelerating rate, asymptotically approaching 0.  For $\Lambda > 0$ the universe slows due to gravity but will
ultimately increase.  A universe with $k = 1$ is closed and has positive curvature.  Its geometry is similar to a sphere - the angles in
a triangle sum to $> 180^{\circ}$ and all lines eventually meet.  The expansion in a closed universe will come to a stop and begin to
contract, returning to a singularity referred to as the ``Big Crunch''.

Using Einstein's field equations with \eqnref{eq:dark_matter_lambda_cdm_rw_metric} gives the Friedman Equations

\begin{equation}
\Big(\frac{\dot{a}}{a}\Big)^{2} = \frac{8\pi G \rho}{3} - \frac{k}{a^{2}}
\label{eq:friedman1}
\end{equation}

\begin{equation}
\frac{\ddot{a}}{a} = -\frac{4\pi G}{3}(\rho + 3p)
\label{eq:friedman2}
\end{equation}

\noindent where $\rho$ is the average energy density of the universe, $\dot{a}/a$ is referred to as the Hubble
constant $H$ ($H_{0}$ today), and $p$ is the pressure.  The density for a flat universe, known as the critical density, can be derived
from \eqnref{eq:friedman1} to be $\rho_{\mathrm{crit}} = 3H^2 / 8\pi G$.  A useful
notation is the ratio of the density to the critical density, known as the density parameter, $\Omega = \rho\rho_{\mathrm{crit}}$
($\Omega = 1$ is a flat universe).  Substituting
$\Omega$ into \eqnref{eq:friedman1} gives the density parameter

\begin{equation}
\Omega - 1 = \frac{k}{H^{2}{a^{2}}}
\end{equation}

The density parameter can be broken down as $\Omega = \sum \Omega_{i}$ where $\Omega_{i}$ is a component of the universe.  Measurements
estimate density parameters of $\Omega_{\mathrm{b}} ~ 0.05$ for baryonic matter, $\Omega_{\mathrm{dm}} = 0.26$ for non-baryonic (dark)
matter, $\Omega_{\mathrm{r}} \sim 0$
for radiation, and $\Omega_{\Lambda} \sim 0.69$ for dark energy (\secref{subsec:cmb}).  These numbers reveal cold dark matter and dark
energy are the dominant components, making up $> 94\%$ of the total matter density.



%====================================
\section[Evidence for Dark Matter][Evidence for Dark Matter]{Evidence for Dark Matter}
\label{sec:dark_matter_evidence}
%========
\subsection{Dynamical Contraints} \label{subsec:dynamics}
The first evidence for Dark Matter came in 1933 when Swiss astronomer Fritz Zwicky was observing the Coma Cluster
and noticed the velocity of the galaxies was larger than the total luminous matter.  He argued that if there were
some additional mass that only acted gravitationally this could explain his observations, and estimated this
to be ${\sim} 400$ times greater than the luminous matter (\citeref{Zwicky1933}).  Today this factor is much
smaller as the observed luminous matter is larger, but the discrepancy persists.  Zwicky coined this
\textit{dunkel Materie} or ``dark matter".  Since then other galaxy clusters have been observed and supported
this claim.

In 1970 Vera Rubin and Kent Ford used H$\alpha$ from H II regions to measure the rotation of stars around the
center of the Andromeda galaxy \citeref{Rubin1970}.  The virial theorem predicts $v(r) = \sqrt{GM(r)/r}$ so
that at large radii the velocity of the galaxies should decrease as $r^{-1/2}$.  Rubin and Ford's observations
contradicted the value $M(r)$ derived from luminous mass.  As with Zwicky's measurement, the results could be
explained by introducing a non-luminous feature - in this case one with $M(r) \sim r$.  Observations of over
one hundred thousand galaxies have since shown similar results, with DM halos containing several times the
luminous mass.

One example is NGC 6501.  The dark halo mass density in \figref{fig:ngc_6501} is estimated as
$\rho (r) = \rho_{0} \Big[ 1 + \Big( \frac{r}{r_{c}} \Big)^{2} \Big]^{-1}$ where $\rho_{0}$ and $r_{c}$
are the central halo mass density and radius, respectively, by fitting a least square to the velocity
distribution \citeref{Begeman1991}.  The left panel shows the mass of luminous matter decreases after
peaking at ${\sim}2$ kpc, but the DM increases steadily and at $>10$ kpc is proportional to $r$ causing
the rotational velocity to flatten.  Unfortunately, there is no independent model for determining the
masses of a galaxy's disk, bulge, and halo, which makes disassociating the mass of baryonic matter from
DM very difficult.  Most models assume the luminous matter to compose the galaxy's disk and bulge with
the DM constituting its halo (\citeref{Sofu2001}).



\begin{figure}
\centering
\captionsetup[subfigure]{labelformat=empty}
\ffigbox[\linewidth]{%
  \begin{subfloatrow}
\ffigbox[\FBwidth][]{\caption{}}{\includegraphics[height=5cm]{ngc_6503}}
  \end{subfloatrow}
  \hspace*{\columnsep}
  \begin{subfloatrow}
\vbox to 5.3cm{
  \ffigbox[\FBwidth]{\caption{}}
  {\includegraphics[trim={2cm, 2cm, 2cm, 2cm}, clip, height=4.4cm]{ngc_6503_image}}
}
  \end{subfloatrow}
  }{
  \vspace*{-10mm}
  \caption{(left) Rotational velocity vs. radius for NGC 6503.  Visible components represented by dashed line,
  gas by dotted, and dark halo by dashed-dotted.
Image credit: (left) \citeref{Begeman1991} (right) Robert Gendler, the Subaru Telescope (NAOJ) and the Hubble Legacy Archive}
	\label{fig:ngc_6501}
  }
\end{figure}

%for figure trim info
%https://tex.stackexchange.com/questions/57418/crop-an-inserted-image





\begin{figure}
	\centering
	\includegraphics[width=0.5\textwidth]{coma_cluster}
	\label{fig:coma_cluster}
	\caption{Coma Cluster as seen by Sloan Digital Sky Survey and Spitzer Space Telescope.  Image credit: NASA/JPL-Caltech/GSFC/SDSS.}
\end{figure}


%========
\subsection{Big Bang Nucleosynthesis}



%========
\subsection{Gravitational Lensing}
\label{subsec:gravitational_lensing}
A gravitational lens is a distribution of matter capable of bending electromagnetic radiation between a luminous source and
an observer.  The deflection is caused by the gravitational distortion of space-time by the mass of the lens, which
while in theory can be anything with mass-energy, the effects are typically most noticeable for high-density objects
such as galaxies, galaxy clusters, or a stars.  A source that is gravitationally lensed will have two distinct
features. The convergence, $\kappa$, describes the focusing of the light rays.  The shear, denoted by $\gamma$ and $\phi$, which
represent the ellipticity and position ange, characterizes the distortion of the source.  One important feature of lensing
is a magnification of the source given by

\begin{equation}
\mu = \frac{1}{(1-\kappa)^{2} + \gamma^{2}}
\end{equation}

\noindent that produces an increase in brightness when the source and lens are close to one another.  This has given astronomers
a tool to see objects that previously were thought to be too faint to observe.

In the case of strong lensing an observer will see a misshapen source in the form of arcs.  Because the magnitude of the
deflection is depending on the proximity to the lens, it is possible to see multiple instances of the same source, with
independent distances traveled by each.  Thus a time-delay between images is also present, and their location to the observer
is incorrect.  In the special case
when the lens is directly between the source and observer an Einstein ring will be produced such that the source
appears as a circular distortion of the source around the object, and no time delay will occur.  \figref{fig:lensing}
shows a an image of a strong lens and a diagram depicting the path of the light.  Originally Einstein
thought gravitational lensing to be useless having only considered what today is known as micro lensing (deflection
about a star), but Fritz Zwicky promptly predicted galaxies could provide stronger lensing as well as
magnification.  When the arcs are sufficiently large in size and flux the source's luminosity can be determined.

\begin{figure}
 \centering
 \begin{subfigure}[t]{0.5\textwidth}
  \centering
  \includegraphics[trim={0cm, 2cm, 0cm, 0cm}, clip, height=4.5cm]{lensing_horseshoe}
 \end{subfigure}%
 \begin{subfigure}[t]{0.5\textwidth}
  \centering
  \includegraphics[trim={6cm, 0cm, 0cm, 0cm}, clip, height=4.5cm]{lensing_diagram}
 \end{subfigure}
 \caption{(left) an Einstein ``horseshoe" ring of a red galaxy distorting a blue galaxy behind it.  (right)
 A diagram of strong lensing.  The red source is bent by the blue lens, with the solid orange lines that connect
 to it representing the light's true trajectory.  The observer incorrectly views the source's
 position as indicated by the dashed orange lines, and the image behind them.  Image credit: (left) ESA/Hubble
 \& NASA (right) ALMA (NRAO/ESO/NAOJ)/Luis Calada (ESO)}
 \label{fig:lensing}
\end{figure}


In many astrophysical instances the lens' deflection of the source is much more subtle.  This is known as weak
lensing, and because of its minor effects is much more difficult to observe.  Astronomers must only consider a source's
shear because its luminosity is too low to be understood, and since the deviation in shear is small, systematic effects of
observing (i.e. atmosphere, instrument point spread function, noise, etc.) must be very small and well
understood \citeref{Paolis2016}.  Whereas strong lensing results from radiation passing around the lens, weak lensing
is the passage of light through a gravitational field where tidal effects distort the shape of the image.  Furthermore,
because galaxies are generally elliptical, deduction of the shear can be difficult to impossible.  Thus the gravitational
field is calculated statistically by randomly sampling galaxies from the known ellipticity distribution.  Since the
number of samples must equal the number of galaxies, this method is most effective when the number of galaxies is
large.

If the mass of the lens is well known, the luminous portion can be subtracted, yielding the fraction of DM.  This can typically
be done for strong lenses as shown in \figref{fig:lensing} but is more difficult for weak.

A galaxy collision provides a unique setting to study dark matter.  1E0657-56 (Bullet Cluster) is two galaxies that passed
through one another ${\sim}150$ million years ago at $4500_{-800}^{+1100}\ \mathrm{km\ s^{-1}}$
\citeref{Markevitch2004}.  \figref{fig:bullet_cluster} shows the x-ray map in pink and the mass
distribution from gravitational lensing in blue in the left panel, along with mass contours in the right.  The separation of
mass from baryonic matter can be explained by incorporating dark matter.  As the galaxies collide intergalactic dust interacts
and heats up, creating x-rays and slowing the speed at which they pass.  The dark matter does not interact and passes
through affected by only gravity.  In addition to providing evidence of dark matter, 1E0657-56 sets a limit on the
cross-section of dark matter self-interaction of $<1\ \mathrm{cm^{2}\ g^{-1}}$ \citeref{Markevitch2004}.  It also provides evidence
against modified gravity - an alternative theory to dark matter - since the observed mass distributions should now lay
outside of the luminous content (\secref{subsec:modified_gravity}).



\begin{figure}
    \centering
    \begin{subfigure}[t]{0.45\textwidth}
        \centering
        \includegraphics[height=4.5cm]{chandra_bullett_preview}
    \end{subfigure}%
    \begin{subfigure}[t]{0.45\textwidth}
        \centering
        \includegraphics[height=4.5cm]{bullet_cluster_paper}
    \end{subfigure}
    \caption{X-ray emission from hot gas (pink) and mass centroids (blue) from gravitational lensing after cluster
	collision of 1E 0657-558.  The white bar in the right panel represents 200 kpc at cluster.  The separation
	between the colors provides evidence for Dark Matter.
	Image credit: (left) X-ray: NASA/CXC/CfA/M.Markevitch et al.; Optical: NASA/STScI; Magellan/U.Arizona/D.Clowe et al.;
	Lensing Map: NASA/STScI; ESO WFI; Magellan/U.Arizona/D.Clowe et al. (right) NASA, ESA, CXC, M. Brada\u{c}
	(University of California, Santa Barbara), and S. Allen (Stanford University)}
	\label{fig:bullet_cluster}
\end{figure}




%========
\subsection{Cosmic Microwave Background} \label{subsec:cmb}
The Cosmic Microwave Background (CMB) was accidentally discovered by Penzias and Wilson, for which they received
the Nobel Prize (\citeref{Penzias1965}).  It
is a remnant from shortly after the Big Bang ($\sim380,000\ \mathrm{yrs}$, $z\sim1100$, $T\sim3000\ \mr{K}$) and a near-perfect
blackbody at $2.725\ \mr{K}$ today (most precise measurement at $2.72548 \pm 0.00057\ \mr{K}$ \citeref{Fixsen2009}).  A CMB
photon is considered to have originated at the time of last scattering $t_{\mathrm{ls}}$ and proceeded unperturbed since.

Deviations in the blackbody are small (root-mean square of $\delta T/T \sim 10^{-5}$) and the results of several
mechanisms at $t_{ls}$ that varied throughout space.  The dipole anisotropy results from
our motion with respect to the comoving rest frame of the CMB (\citeref{Smoot1991}).  Furthermore
energy density perturbations would cause fluctuations in gravitational potential $\delta \Phi$.  A
photon at larger $\delta \Phi$ at $t_{\mathrm{ls}}$ would become blueshifted as its
potential decreases, while one at lower $\delta \Phi$ would be redshifted.  This effect is known as the
Sachs-Wolfe effect (\citeref{Sachs1967}).  Additional effects, including intrinsic fluctuations and acoustic
oscillations, are not discussed.

\begin{figure}
\centering
\includegraphics[width=0.8\textwidth]{PlanckFig_map_columbi1_IDL_HighDR_colbar_1000px_CMB_moll}
\caption{CMB as observed by Planck.  Temperature deviations $\delta T/T \sim 10^{-5}$.  Image credit: \citeref{}}
\label{fig:planck_map}
\end{figure}


The CMB provides the most precise measurements on $\Omega_{\mathrm{b}}$, $\Omega_{\mathrm{dm}}$, $\Omega_{\Lambda}$,
$\Omega_{\mathrm{r}}$, $H_{0}$, and many other cosmological parameters.  It has been precisely charted by several satellites since its
discovery, with the most recent being Planck (\citeref{Plack2011}).  The CMB is shown in \figref{fig:planck_map},
where fluctuations are on the order of $\sim 10^{\pm 3} \mathrm{\mu K}$.  These fluctuations were caused by the number and
amount of each component - thus, by finding the correlation function

\begin{equation}
%C(\theta) = \Big \langle \frac{\delta T}{T}( \hat{n}) \frac{\delta T}{T} ( \hat{n}^\prime) \Big \rangle
C(\theta) = \Big \langle \delta T( \hat{n}) \delta T(\hat{n}^\prime) \Big \rangle
\end{equation}

\noindent we can identify the correction cosmological makeup.  To do this we make use of

\begin{equation}
\delta T = \sum\limits_{l=0}^{\infty} \sum\limits_{m=-l}^{l} a_{lm}Y_{lm}(\theta, \phi)
\end{equation}

\noindent where $\delta T = T(\theta, \phi) - \langle T \rangle$ for a given $\theta , \phi$
on the map, $Y_{lm}(\theta, \phi)$ corresponds to spherical harmonics, and $a_{lm}$ the preceding coefficients
with the condition that $\sum\limits_{l,m} |a_{lm}|^{2} = 1$.  This gives a correlation function of

\begin{equation}
C(\theta) = \frac{1}{4 \pi} \sum\limits_{l=0}^{\infty} (2l + 1) C_{l} P_{l}(cos \theta)
\end{equation}

\noindent where $P_{l}$ are the Legendre polynomials and $C_{l} = \frac{1}{2l + 1} \sum\limits_{m=-l}^{l} a_{lm}$.  The
only unknown is $C_{l}$, which is typically shown in an angular power spectrum with
$D_{l}^{TT} \equiv l(l+1)C_{l}/2\pi$ such as in \figref{fig:cmb_power_spectrum}.

\begin{figure}
%	\centering
	\includegraphics[width=0.8\textwidth]{cmb_power_spectrum}
	\centering
	\caption{Angular ower spectrum for CMB from Planck measurements, fit using the $\Lambda$CDM model.  Residuals
	are shown in bottom panel.  Image credit: \citeref{Planck2016}}
	\label{fig:cmb_power_spectrum}
\end{figure}

The first peak in the spectrum is related to the curvature of the universe, while the ratio of the heights of the
first and the second peaks provides details about the amount of baryonic matter.  The best fit gives
$H_{0} = 67.81 \pm 0.92\ \mathrm{km\ s^{-1}\ Mpc^{-1}}$, $\Omega_{\Lambda} = 0.692 \pm 0.012$,
$\Omega_{\mathrm{b}} = 0.0484 \pm 0.0005$, $\Omega_{\mathrm{dm}} = 0.258 \pm 0.004$, and
$\Omega_{k} \equiv 1 - \Omega_{\Lambda} - \Omega_{\mathrm{b}} - \Omega_{\mathrm{dm}} =  -0.005_{-0.017}^{+0.016}$.


%[5] Adams, F.C., Freese, K. & Guth, A.H. [1991], Phys. Rev. D43, 965.



%$\Omega_{m} = 0.308 \pm 0.012$
%$H_{0} = 67.81 \pm 0.92$
%$\Omega_{\Lambda} = 0.692 \pm 0.012$
%$\Omega_{b}h^{2} = 0.02226 \pm 0.00023$
%$\Omega_{c}h^{2} = 0.1186 \pm 0.0020$
%$N_{eff} = 3.046$
%$\Omega_{k} \equiv 1 - \Omega_{m} - \Omega_{\Lambda} = -0.005_{0.017}^{0.16}$ for $\Lambda CDM$

%========
%\subsection{Structure Formation} \label{subsec:structure}

%\endcsname

%====================================
\section[Dark Matter Candidates][Dark Matter Candidates]{Dark Matter Candidates}
\label{sec:dmcandidates}
Despite a strong claim for the existence of dark matter, there is no evidence for what its composition may
be.  A candidate for dark matter must have a lifetime much larger than the age of the universe, be electrically
neutral, and have a small matter-dark matter cross section.  Of course there may be more than one particle
that classifies as dark matter, but the sum of them should satisfy our observations of the universe.



\subsection{Modified Gravity}
\label{subsec:modified_gravity}
One possible explanation for the discrepancy between observed matter and its behavior is an incomplete understanding of gravity.  This
hypothesis suggests that Einstein's General Relativity works well at small distances but does not correctly explain large-scale
gravitation.  Thus dark matter would not be a separate particle, but the inevitable consequence of ordinary matter that should always be
present.

Major evidence against modified gravity emerged in 2004 when the collision of two galaxies 1E0657-56 was observed as mentioned
in \secref{subsec:gravitational_lensing}.  \citeref{Markevitch2004} reconstructed the mass distribution of the merger from gravitational
lensing measurements (\figref{fig:bullet_cluster}).  They found in their measurements required a significant portion of the mass
distribution to be offset from the visible matter, consistent with collisionless dark matter halos passing through one another.  This
presented a problem to modified gravity, which was unable to explain this.

In 2018 the ultra-diffuse galaxy NGC1052-DF2 was observed to have $M_{\mathrm{halo}} / M_{\mathrm{stars}} \sim 1$, a factor of
${\sim} 400$
lower than expected. (\citeref{VanDokkum2018a, VanDokkum2018b}).  This poses a challenge to modified gravity - including modified Newtonian dynamics
(MOND), a popular theory that has successfully predicted various galatic phenomena (\citeref{Milgrom1983}) - since a larger gravitational
field does not appear to exist around the ordinary baryonic matter.  However, several theories claimed this may be compatible with
expectations due to its proximity to its massive host galaxy NGC1052 and large uncertainties on some of the measurements by
\citeref{VanDokkum2018a} (\citeref{Famaey2018, Moffat2018}).  A galaxy without dark matter has implications outside of dark matter such
as how it might form (\citeref{Abraham2018, Ogiya2018}).  Two recent papers found conflicting results when studying the 175 SPARC
galaxies (\citeref{Lelli2016}).  The first found agreement with MOND using gaussian priors centered around values given by SPARC with
uncertainty set to observational errors (\citeref{Li2018}).  The second included an additional 18 galaxies from THINGS
(\citeref{deBlok2008}) and using flat priors excluded MOND
at $10 \sigma$ (\citeref{Rodrigues2018}).  Ultimately additional
measurements and improved statistical treatments are needed to better quantify NGC1052-DF2,
but confirmation of missing dark matter would strongly constrain theories of modified gravity.



%========
\subsection{Axions} \label{subsec:axions}
Axions were originally hypothesized by R.D. Peccei and Helen R. Quinn in 1977 as a solution to the strong CP
(charge and parity) problem (\citeref{Peccei1977}).  Quantum chromodynamics (QCD) predicts there should be
CP violation in strong interactions.

CP (charge and parity) violation in strong interactions has never been observed, despite its prediction
by quantum chromodynamics (QCD).  This forces a theoretically unjustified fine tuning of the model, which
is known as the strong CP problem.  Originally hypothesized by R.D. Peccei and Helen R. Quinn in 1977, the
axion - a new standard model particle - offered a solution (\citeref{Peccei1977}).  Shortly after it was
demonstrated that for axion decay constant $f_{\mathrm{a}} > 10^{12}$ axions would be overproduced in the
early universe and cause the axion density $\Omega_{a} > 1 > \Omega_{\mathrm{dm}}$ \citeref{Preskill1983}.  However,
a decay constant of $\sim 10^{12}$ could satisfy $\Omega_{\mathrm{dm}}$.  Because the axion mass $m_{\mathrm{a}}$ and $f_{\mathrm{a}}$
are inversely proportional, one can then set a limit on $m_{a}$.
%The axion
%mass $m_{a} = 57(10^{11}GeV/f_{a})\ \mu$eV, which gives a lower bound of $m_{a} \sim 5\ \mu$eV.

Because axions naturally offer an explanation for dark matter there are a number of experiments dedicated to
finding them.  Cavity searches such as ADMX use a resonant microwave cavity inside a superconducting magnet
to convert axions in microwaves.  Others, like CASPEr apply NMR.

\begin{figure}
\centering
\includegraphics[width=\textwidth]{axion_limits}
\caption{Image credit: \citeref{Patrignani2016}.}
\label{fig:axions}
\end{figure}


%========
\subsection{WIMPs} \label{subsec:wimps}
WIMPs (Weakly Interacting Massive Particles) are another favored candidate for dark matter.  As their name
suggests, they interact through the weak force and thereby would be difficult to observe.  They
are not constrained to the standard model, though would behave similarly
to neutrinos, which have a small cross-section and rarely interact with nuclei.  An additional requirement
is they must be produced early in the universe to account for observations of the CMB and galactic
structures.

At the beginning of the universe the temperature was hot enough where particles could annihilate with their
antiparticle counterpart and produce new particles, maintaining equilibrium.  As the universe cooled each
particle had a ``freeze-out", when they could no longer transfer freely to other particles.  Using the
$\Lambda$CDM model the density of DM in the universe today is given by

\begin{equation}
\Omega_{\mathrm{dm}}h^{2} = \frac{3 \times 10^{-27}\ \mathrm{cm^{3}\ s^{-1}}}{\langle \sigma_{\mathrm{ann}} v \rangle}
\end{equation}

\noindent where $h$ is the Hubble Constant divided by 100 and $\langle \sigma_{\mathrm{ann}} v \rangle$ is
the thermally averaged self-annihilation cross section
for dark matter.  Assuming DM has a cross-section and mass on the order of the weak force, such a
particle would give roughly the correct relic density of DM.  This is known as the ``WIMP miracle".

Another appealing argument for WIMPs is super-symmetry (SUSY), which is theorized to solve some problems
with the standard model,
predict WIMP-like particles of similar masses.  This has historically been one of the favored arguments
for dark matter.

%========
\subsection{Cold, Warm, or Hot} \label{subsec:hot_vs_cold}
An important property of dark is whether or not it was relativistic in the early universe.  Hot dark matter (HDM)
is defined as being relativistic at the time it decoupled from other components, $t_{\mathrm{dec}}$, and at
matter-radiation equality, $t_{\mathrm{rm}}$.  Warm dark matter (WDM) would have been relativistic at $t_{\mathrm{dec}}$
but not at $t_{\mathrm{rm}}$.  Cold dark matter (CDM) would have been non-relativistic at both.  Candidates for CDM include
WIMPs, axions, and primordial black holes while WDM might be the gravitino.  Neutrinos are possible candidates for both
WDM and HDM.

Understanding which DM universe we live in can come from looking at structure in the universe.  Because the structure
today came from fluctuations in the earliest moments of the Big Bang (\secref{subsec:cmb}) the cosmological layout
can illuminate the answer.

In HDM the relativistic particles are able to smooth out density perturbations, which is known as free
streaming.  In this case the first structures to form would be superclusters, followed by smaller-scale
features.  Observations show that this is not the case; galaxies have been around since before the universe
was 1 billion years old ($z \sim 6$) and superclusters are just forming today (\citeref{Ryden2003}).

CDM allows the early density perturbations to persist, causing smaller structures to materialize first,
consistent with galaxy surveys between ${\sim} 1$ Mpc to the horizon.  At scales $< 1$ Mpc and
$M \sim 10^{11} \ M_{\odot}$ there are discrepancies, including under-dense cores for many galaxies that are DM-dominated
and significantly fewer satellite dwarf and small galaxies than predicted (\citeref{Moore1999}, \citeref{Klypin1999}).  Possible
solutions to the latter may be that dwarf galaxies have not accumulated enough baryonic matter to be visible
(\citeref{Simon2007}), merged, or been stripped by tidal forces of larger galaxies.

WDM has received a lot of interest since the CDM problems were observed.  Simulations have shown that
WDM would result in fewer subhalos, though other model-observation contradictions have been less
successful (\citeref{Bullock2017}, \citeref{Ogiya2017}).  However, since neutrinos have mass demonstrate there
was at least some non-CDM in the early universe.

Despite the problems with CDM, it remains the most favorable model for dark matter.  One possible outcome is
there is a mix of CDM and WDM, but if that's the case CDM would make up the considerable bulk of dark matter.


%====================================
\section[WIMP Detection Methods][WIMP Detection Methods]{WIMP Detection Methods}
\label{sec:detection}

There are three methods to detect WIMPs.  The first is through particle colliders where standard model (SM) particles would interact to
produce DM.  The second is via indirect detection, where DM would annihilate into SM particles, with the hope that it would be detectable
on Earth.  The third method is by means of direct detection in which DM scatters off SM matter, producing a signal that would be
observable.  These methods are outlined in \figref{fig:detection_methods}.

\begin{figure}
\centering
\includegraphics[width=0.8\textwidth]{}
\caption{}
\label{fig:detection_methods}
\end{figure}

 %========
\subsection{Colliders} \label{subsec:colliders}
One possible mechanism through which we might observe WIMP dark matter is through particle-antiparticle
annihilation.  This could be observed at particle colliders where energies can exceed
several TeV, thereby producing DM particle-antiparticle pairs, which
would escape undetected.  From momentum conservation there would be missing transverse energy (MET),
which would be carried in the DM.  The Large Hadron Collider (LHC) is investigating the quark sector with energies
exceeding 10 TeV, while the Large Electron-Positron (LEP) Collider is doing so for leptons at
$\sim 200$ GeV.  Both experiments are competitive at lower energies than noble gas experiments, particularly
for spin-dependent searches (\citeref{Fox2011, Alpigiani2017}).

Annihilation alone would not prove the discovery of dark matter, and would need
indirect or direct experiments to validate the results with their own detectors.  But
it would still be useful in narrowing the search region.

\begin{figure}
\centering
\includegraphics[width=0.8\textwidth]{collider_diagram}
\caption{Diagram of a collision at the LHC.  Image credit: \citeref{CERN2018}.}
\label{fig:collider}
\end{figure}


 %========
\subsection{Indirect Detection} \label{subsec:indirect}
Indirect detection looks for signatures of DM by observing standard model particles.  Such observables
may come from dark matter annihilation, wherein two DM particles annihilate and produce standard model
gamma rays or other particle-antiparticle pairs.  Alternatively if DM is unstable it may decay into
standard model particles that can be detected.

Indirect experiments look towards regions where they expect a large number of interactions.  The local
dark matter density is estimated to be 0.2-0.56 GeV/cm$^{3}$ (\citeref{Read2014}) and scientists are observing
the Sun, where they hope to find an observable flux of high energy neutrinos (\citeref{Ellis1988}).  Other theories suspect
DM annihilation in the galactic halo would produce
antiprotons, positrons, and gamma rays that would be detectable on Earth.  Because the the galactic center
has a large flux of cosmic rays it is difficult to distinguish dark matter from other astrophysical
sources.  Nearby ($\sim 50$ kpc) dwarf spherical
galaxies have become an attractive target where star formation regions have an expected low $\gamma$-ray
background (\citeref{Zitzer2016}).

Measurements of $\gamma$-rays would have to be from space because for the necessary energy range (GeV to TeV) photons interact
with matter via $e^{+}e^{-}$ pair production, so would not be able to pass through Earth's atmosphere.  However,
they can look for signatures such as showers of secondary particles and their Cerenkov
light as they pass through the atmosphere (\citeref{Bertone2005}).

% for Ellis1988 reference listed above get their references 6 and 7 and maybe 8 for citations, need access to PRL


 %========
\subsection{Direct Detection} \label{subsec:direct}
Direct detection looks for low energy ($\sim 1-100$ keV) nuclear recoil (a few theories predict DM-lepton
but they will not be discussed) (\citeref{Kopp2009}).  Given that the majority of dark matter must be
non-relativistic (\secref{subsec:hot_vs_cold}), we can calculate the differential recoil spectrum as \citeref{Undagoitia2016}

\begin{equation} \label{eq:dr_de}
\frac{dR}{dE}(E, t) = \frac{\rho_{0}}{m_{\chi}m_{\mathrm{A}}} \int_{v_{\mathrm{min}}}^{v_{\mathrm{esc}}}
v f(\vectlett{v}, t) \frac{d\sigma_{\chi}}{dE}(E, t)\ d^{3}\vectlett{v}
\end{equation}

\noindent where $\rho_{0}$ is the local dark matter density of 0.2-0.56 GeV/cm$^{3}$ (\secref{subsec:indirect}), $m_{\chi}$ is
the mass of a
dark matter particle, $m_{\mathrm{A}}$ is the mass of the target element, $v_{\mathrm{esc}}$ is the escape velocity for WIMPs from the galaxy,
$f(\vectlett{v}, t)$ is the local velocity dispersion, and $\frac{d\sigma_{\chi}}{dE}(E, t)$ is the nucleon-DM differential
cross-section.  $v$ is the velocity of the DM in the rest frame of the detector.  The minimum velocity produce
a recoil of energy E is given by

\begin{equation}
v_{\mathrm{min}} = \sqrt{\frac{m_{\mathrm{A}} E}{2 \mu^{2}}}
\end{equation}

\noindent where the WIMP-nucleus reduced mass $\mu_{\mathrm{A}} = m_{\mathrm{A}} m_{\chi} /( m_{\mathrm{A}} + m_{\chi})$
and the velocity for
WIMPs to overcome the gravity of our galaxy has been measured to be $v_{\mathrm{esc}} = 533_{-41}^{+54}\ \mathrm{km\ s^{-1}}$
(\citeref{Piffl2014}).

Assuming the standard halo model (SHM)
Though there has been disagreement as to whether the DM velocity distribution can be described as Maxwell-Boltzmann
(\citeref{Diemand2004}, \citeref{Kuhlen2009}), we assume that this is the case.  For WIMP searches most experiments
are looking at spin-independent (SI) or spin-dependent (SD) interactions.  For spin-independent all nucleons
contribute equally.  For spin-dependent, atoms must have an odd number
of protons or neutrons since only unpaired nucleons contribute to the search.  The differential cross-section can then be written as

\begin{equation} \label{eq:diff_sigma_si}
\frac{d \sigma_{\chi}}{dE} = \frac{m_{\mathrm{A}}}{2 \mu_{\mathrm{A}}^{2} v^{2}} \big( \sigma_{0}^{\mathrm{SI}} F_{\mathrm{SI}}^{2}(E) +
\sigma_{0}^{\mathrm{SD}} F_{\mathrm{SD}}^{2}(E) \big)
\end{equation}

\noindent where $\sigma_{0}^{\mathrm{SI}}$ and $\sigma_{0}^{\mathrm{SD}}$ is the cross section at zero momentum for
spin-independent and spin-dependent DM.  $F_{\mathrm{SI}}^{2}(E)$ and $F_{\mathrm{SD}}^{2}(E)$ are the form factors, which
account for the cross-section decrease as energy increases (\citeref{Lewin1996}).  The SI form factor is
the Fourier transform of the mass density ground state, and for the parameterization given in \citeref{Helm1956} is given by

\begin{equation}
F_{\mathrm{SI}} = \frac{3 j_{1}(qr_{\mathrm{n}})}{qr_{\mathrm{n}}} e^{-(qs)^{2}/2}
\end{equation}

\noindent where $j_1(r_{\mathrm{n}}p)$ is a Bessel function of the first kind, $q = \sqrt{2m_{\mathrm{A}}E}$ is the momentum,
$r_{\mathrm{n}} = \sqrt{1.2A^{2/3} - 5s^{2}}$, and $s \sim 1$ fm is a measure of the nuclear skin (\citeref{Lewin1996},
\citeref{Engel1991}).  The cross section is then given by

\begin{equation} \label{eq:sigma_si}
\sigma_{0}^{\mathrm{SI}} = \sigma_{\mathrm{p}} \frac{\mu_{\mathrm{A}}^{2}}{\mu_{\mathrm{p}}^{2}} \frac{\big[ Z f^{p} +
(A - Z) f^{n} \big]^{2}}{(f^{p})^{2}}
\end{equation}

\noindent where $\sigma_{\mathrm{p}}$ is the cross section of a proton Z is the number of protons and $\mu_{\mathrm{p}}$ is
the WIMP-nucleon reduced mass. $f^{p/n}$ is the coupling strength for protons and neutrons, which are assumed to be
equivalent (see \citeref{Yaguna2017} for $f^{p} \neq f^{n}$).  Substituting \eqnref{eq:sigma_si} into \eqnref{eq:diff_sigma_si}
gives a spin-independent cross-section of

\begin{equation}
\frac{d \sigma_{\chi}}{dE} = \frac{m_{\mathrm{A}} \sigma_{\mathrm{p}}^{\mathrm{SI}}}{2 \mu_{\mathrm{p}}^{2} v^{2}}
 A^{2} \big| F(E) \big |^{2}
\end{equation}

\noindent This gives the differential rate as 

\begin{equation}
\frac{dR}{dE} = \frac{\rho_{0} A^{2} \sigma_{\mathrm{p}}^{\mathrm{SI}}}{2 m_{\mathrm{\chi}} \mu_{\mathrm{p}}^{2}}
  \big| F(E) \big |^{2} \int_{v_{\mathrm{min}}}^{v_{\mathrm{esc}}}
\frac{f(v)}{v}\ dv
\label{eq:dr_de_final}
\end{equation}

\noindent A direct detection experiment that is sensitive between energies $E_{\mathrm{min}}$ and $E_{\mathrm{max}}$ can count the number
of observed signals over a time $T$ for their target mass $M$ as

\begin{equation} \label{eq:counts}
N ( m_{\chi}, \sigma_{p}^{\mathrm{SI}}) = T \times M \times \int_{E_{\mathrm{min}}}^{E_{\mathrm{max}}} \frac{dR}{dE} dE
\end{equation}

\noindent \eqnref{eq:dr_de_final} and \eqref{eq:counts} state that the sensitivity of an experiment improves linearly with time and
target mass, but quadratically with $A$.  This makes heavier elements an important consideration when designing a direct detection
experiment.

%For SD interactions the cross-section is

%\begin{equation}
%\sigma_{0}^{\mathrm{SD}} = \frac{32}{\pi} \mu_{\mathrm{A}}^{2} G_{\mathrm{F}}^{2} \big[ a_{\mathrm{p}} \langle S^{\mathrm{p}} \rangle +
%a_{\mathrm{n}} \langle S^{\mathrm{n}} \rangle \big] \frac{J + 1}{J}
%\end{equation}

%\noindent where $G_{\mathrm{F}}$ is the Ferm coupling constant, $a_{\mathrm{p/n}}$ is the coupling constants, and $J$ is
%the total nuclear spin.




% https://journals.aps.org/rmp/abstract/10.1103/RevModPhys.28.214 for distribution of WIMP scatters assumed to be same
% as charge distribution derived from electron and muon scattering 


%correctly describes DM velocity
%as discussed in \secref{subsec:dynamics} assume the WIMP velocity the Maxwell-Boltzmann distribution



%Because current theory
%predicts the DM distribution to be in a halo around the galaxy (\secref{subsec:dynamics}), DM particles should
%be passing through Earth and - if they're WIMPS - interact with the nuclei of standard model atoms.



%The
%energy deposited in such a collision would be manifested as scintillation, excitation (nucleus or electron),
%ionization, or phonons.




%====================================
 \section[Direct Detection Experiments][Direct Detection Experiments]{Direct Detection Experiments}
 \label{sec:direct_detect}
 A moving particle that interacts with a detector will deposit some of its energy, causing different effects.  Some of the energy
 will be expelled as heat or propagated through the medium as phonons.  Additionally the atoms in the detector may become
 excited or ionized, producing scintillation and free electrons.  The different energy channels can be seen in
 \figref{fig:energy_channels}.  Experiments are probing all three of these observables, with many able to measure two
 simultaneously.
 
 Direct detection experiments look for some anomalous signal outside of their expected background.  This requires a comprehensive
 understanding of detector materials and physical location as both can produce radiation that will contaminate any signal.  For the
 former, major efforts are going into screening outsourced material and limiting the number of radioactive contaminates in
 production.  For on-site background radiation, shielding has been essential for many experiments who without would have orders
 of magnitude higher event rates.
 
\begin{figure}
\centering
\includegraphics[width=\textwidth]{EnergyChannels}
\label{fig:energy_channels}
\end{figure}
 
 Even with a priori screening and effective shielding background events are inevitable.  Identifying a signal may be too
 difficult if the signal-to-background ratio is small.  In 1986 it was proposed
 that due to the Earth's relative motion around the Sun there should be a modulation in signal (\citeref{Drukier1986}).  Thus
 detectors could look for a an annual variation in event rate with a maximum in May, when the Earth's velocity around the
 Sun aligned with to the Sun's velocity around the galaxy.

\begin{figure}
\includegraphics[width=0.6\textwidth]{wimp_wind}
\caption{Orientation of the Earth's rotation around the Sun with respect to a hypothetical WIMP wind.  As the Earth orbits the Sun
the component of its velocity parallel to the WIMP wind changes, varying the amount of dark matter passing through the Earth and
causing an annual modulation.  Note that while dark matter could be revolving around the galaxy, the term ``WIMP wind'' corresponds at
least to the relative motion of the Sun around the Milky Way in the case of an non-rotating DM halo.}
\label{fig:direct_detect_modulation}
\end{figure}

Direct detection experiments with low background offer the greatest sensitivity to detecting dark matter.  This is simply because
an experiment with a larger background will report an excess with less significance than one with a lower, i.e. a detector with a 10 event
background can make a stronger claim on an excess of 5 events than one with 100.  However, it is essential that the background is
well-understood, since if the background of the former was not complete they may falsely claim a discovery.
 
  
 
 %========
\subsection{Superheated Liquid Detectors}
\label{subsec:bubbles}
Superheated liquid detectors have peak sensitivity to WIMP masses in the tens of GeV range.  Their limits for spin-independent
WIMPs are not as stringent as dual-phase noble gas detectors (\secref{subsec:dual_phase}), but they have consistently provided the
strongest constraints in the spin-dependent sector (\citeref{PICO2017}).  Their strong sensitivity comes from using fluorinated
halocarbons, as $^{19}$F is preferred over other detector materials due to its isotopic abundance of 100\% unpaired proton
almost always carrying 1/2 spin (\citeref{Ellis1991}, \citeref{PICASSO2012}).

At ambient temperatures and pressures fluorinated halocarbons in a metastable state.  A particle will transfer some of its heat,
which can result in nucleation and is observed acoustically or optically.  By varying the temperature and pressure settings
the superheated detectors can reduce $\gamma$ and $\beta$ interactions by a factor of 10$^{9}$, making them a great candidate for
SD dark matter discovery.

Additional superheated detector collaborations include PICASSO (\citeref{PICASSO2017}), COUPP (\citeref{COUPP2012}),
PICO (\citeref{PICO2017}), SIMPLE (\citeref{SIMPLE2012}), and MOSCAB (\citeref{MOSCAB2017}).


\begin{figure}
 \centering
 \includegraphics[width=0.8\textwidth]{spin_dependent_limits}
 \caption{Spin-dependent WIMP-proton cross-section 90\% C.L. limits for PICO-60 C$_{3}$F${8}$ (thick blue), PICO-60 CF$_{3}$I (thick red,
 \citeref{PICO2016a}), PICO-2L (thick purple, \citeref{PICO2016b}), PICASSO (green, \citeref{PICASSO2017}), SIMPLE (orange,
 \citeref{SIMPLE2014}), PandaX-II (cyan, \citeref{PandaXII2016a}), IceCube (dashed and dotted pink, \citeref{IceCube2017}), and SuperK
 (dashed and
 dotted black, \citeref{SuperK2011, SuperK2015}).  The filled in purple region is constrained parameter space for minimal
 supersymmetric model from \citeref{Roszkowski2007}.
 Image credit: \citeref{PICO2017}.}
 \label{fig:sd_limits}
\end{figure}

%When superheated liquid detectors were first considered for WIMP detectors there were some crucial points
%that needed to be addressed.  The downtime of such a detector was high because the temperature
%For WIMP searches modifications had to be made from the original
%devices including increased stability for near-continuous operation and operation as a counting experiment
%(\citeref{Pullia2014}).

 %========
\subsection{Scintillation Crystals}
\label{subsec:crystals}
Another target for DM searches is highly radiopure scintillating crystals.  Typically sodium iodide (NaI) or cesium iodide (CsI) is chosen,
with NaI being the most common.  Their sensitivity to SI and SD, low energy threshold, can be run at room temperature,
and ability to run over long periods of time
make them an attractive option.  Doping with thallium causes an increase in wavelength providing high detection efficiency and large
crystal transparency (\citeref{Undagoitia2016}).  However, traces of $^{40}$K, and to a lesser extend $^{238}$U, $^{232}$Th,
and $^{87}$Rb in the NaI(Tl) crystals contribute a 3 \kevee peak (X-ray/Auger electron from $^{40}\mathrm{K} \rightarrow ^{40}\mathrm{Ar}$)
and flat background, which has made improved radiopurity an immediate goal (\citeref{Shields2015}).  Additionally, the scintillation
does not have unique features for different interactions, so particle discrimination is not possible, with the excepting of
multiple hits rejection.  Thus separating background from signal on and event by even basis is not possible.

The DAMA/NaI and its successor DAMA/LIBRA experiments are located underground at the Laboratori Nazionali del Gran Sassa (LNGS) in
Italy.  They use NaI(Tl) crystals that are sensitive.  For over 14 annual cycles they have observed an annual modulation in the 2-6 \kevee
range with 9.3$\sigma$ (\citeref{DAMA2013}).  This would give WIMP masses of 10-15 GeV for scattering off Na
and 60-100 GeV for I.  Their results are shown in \figref{fig:dama}.  Because NaI(Tl) cannot distinguish scatterings by different particles there is doubt on if their signal
is caused by WIMPs, or even dark matter.  Outside of spin-independent, spin-dependent or mixed coupling (\citeref{Bernabei2001})
and inelastic scattering (\citeref{Bernabei2002}) WIMPs 
have been considered.  Because DAMA/LIBRA cannot know if the interactions are with nuclei or electrons, alternative dark matter
electron-coupling models (\citeref{Bernabei2006}) have also been examined.  The results are controversial because other experiments
have surpassed the DAMA/LIBRA sensitivity and see no signal (\citeref{Aprile2017a}).  This has provided motivation to explain the annual modulation via
standard particles.  Hypotheses include known variation in muon flux due to changing stratosphere temperatures, which exhibits
annual modulation in similar phase with DAMA/LIBRA's findings (\citeref{Blum2011}), an incomplete understanding of neutron backgrounds
(\citeref{Ralston2010}), or using a combination of muon-induced neutrons and soloar neutrinos (\citeref{2014Davis}) (though this
has been received pushback in \citeref{Barbea2014} and \citeref{Klinger2015}).

\begin{figure}
\centering
\includegraphics[width=\textwidth]{DAMAModulation}
\caption{Oscillating event rate between 2-6 \kevee captured by DAMA and DAMA/LIBRA.  A fit of $A \cos \omega (t - t_{0})$ where
$t_{0} = \mathrm{June\ 2nd}$ and $A$ is the best-fit amplitude is overlaid, with verticale dashed lines corresponding to expected
maxima.  For 14 years they have observed an annual
modulation with $9.3\sigma$.  Claiming this modulation to be produced by dark matter is controversial however, as they are unable to
discriminate between electronic and nuclear recoils and a number of experiments have surpassed their sensitivity with no significant
findings.  Image credit: \citeref{Bernabei2013}.}
have surpassed 
\label{fig:dama}
\end{figure}

 %========
\subsection{Germanium Detectors}
\label{subsec:germanium}
High purity germanium (HPGe) detectors measure ionization and offer great energy resolution.  As with scintillation crystals,
they can reach very low energies (${\sim} 0.5\ \mathrm{keV}$), making them a promising candidate for low-mass WIMPs (${\sim} 10$ GeV).  It
is not possible to identify nuclear recoils from electronic, but p-type doped detectors have a dead layer on the surface and can use the
pulse rise time to reject background surface events.  In 2010 CoGeNT
announced it observed an annual modulation, similar to DAMA/LIBRA.  After continued observation showed a preference for the modulation
at 2.2$\sigma$ with a best-fit WIMP mass of $m_{\chi}\sim 8$ GeV (\citeref{CoGeNT2014}).  However, subsequent analyses with different
background assumptions showed that this preference is well below 1.7$\sigma$ (\citeref{Aalseth2014}, \citeref{Davis2014}).  Another
experiment, CDEX, uses the same setup as CoGeNT and found contradictory results (\citeref{CDEX2014}), as did CDMS II for NR only
(\citeref{CDMS2012}).


%========
\subsection{Cryogenic Bolometers}
\label{subsec:bolometers}
Cryogenic bolometers are typically cooled to $10-100$ mK so they can detect phonons, allowing a low energy threshold and
outstanding energy resolution.  The CDMS collaboration uses silicon and germanium detectors where they measure both phonons
and ionization.  The ionization-to-phonon ratio is used for discrimination where less than 1 in $10^{6}$ of all ER events
are rejected in the $10-100$ keV range (\citeref{CDMS2015}).  In 2013 an excess of events was reported with a best-fit
WIMP mass of 8.6 GeV (\citeref{Agnese2013}).  However, later measurements did not observe such an excess, nor did EDELWEISS,
which has an analogous setup (\citeref{EDELWEISS2016}).

The CREST-II experiment measured phonons as well as scintillation in CaWO$_{4}$ crystals.  They also observed an excess of
events at WIMP masses of 11.6 $\mathrm{GeV/c^2}$ at 4.2$\sigma$ and 25.3 $\mathrm{GeV/c^2}$ at 4.7$\sigma$ (\citeref{CRESST2012}), but in
a later publication ruled them out (\citeref{CRESST2015}).


 %========
\subsection{Liquid Noble Gas Detectors} \label{subsec:noble_gas}
Liquid noble gas detectors have led the field for spin-independent WIMP masses $\gtrsim 20$ GeV, as seen in
\figref{fig:si_sensitivity}.  Commissioned detectors have
used either liquid argon (LAr) or liquid xenon (LXe) and the DEAP/CLEAN collaboration is constructing a detector than can house
LAr and liquid neon (LNe) independently (\citeref{Kearns2012, CLEAN2015}).  Advantages of liquid noble gas detectors
include scaleability, particle discrimination, and self-shielding, among others.  As the target mass of these detectors passes
the 1-ton mark they are becoming sensitive to neutrinos, offering another branch of physics to study.  Predicted integral rates for xenon,
argon, and neon, as well as germanium are shown in \figref{fig:material_wimp_rate}.

\begin{figure}
\centering
\includegraphics[width=0.6\textwidth]{IntegralRateTargets}
\caption{Expected integral spectra for Xe, Ge, Ar, and Ne for spin-independent elastic scattering of a 100 GeV WIMP with cross section
$\sigma_{\chi}^{\mathrm{SI}} = 10^{-45}\ \mathrm{cm^{2}}$ per nucleon.  The rates assume perfect energy resolution and are calculated with
standard halo parameters.  The dots correspond to average thresholds for each element.  Image credit: \citeref{Cushman2013}.}
\label{fig:material_wimp_rate}
\end{figure}

Liquid noble gas detectors can be single- or dual-phase.  Dual phase use an electric field to drift electrons to the liquid surface, where
they are extracted across a gas gap generating a second scintillation.  Single-phase detectors measure only the scintillation and typically
spherical.  Because they do not drift electrons they benefit from 4$\pi$ photo-detection (in dual phase this would interfere with the
drift field.  Dual phase experiments are discussed discussed in detail in \chapref{chap:liquid_xe}.  \figref{fig:si_limits} shows the
current status of spin-independent limits.

\begin{figure}
\centering
\includegraphics[width=\textwidth]{si_limits}
\caption{Status of spin-independent WIMP cross section from $0.5 \mdash 1000\ \mathrm{GeV/c^2}$ before XENON1T run-combined analysis
(\chapref{chap:xenon1t}).  Possible regions of signal events from DAMA/LIBRA (shaded green, \citeref{DAMA2009}) and the silicon detectors
of CDMS II (shaded blue, \citeref{CDMS2013}) are shown.  Exclusion limits from PICO-60 (solid yellow, \citeref{PICO2017}), DEAP-3600
(dashed purple $> 20\ \mathrm{GeV/c^2}$,
\citeref{DEAP2017}), SuperCDMS (dashed blue, \citeref{SuperCDMS2017}), NEWS-G (dashed purple $< 20\ \mathrm{GeV/c^2}$,
\citeref{NEWSG2018}), CRESST (solid red, \citeref{CRESST2016}), CDMSLite2 (solid blue, \citeref{CDMSLite2016}), LUX
(solid orange, \citeref{LUX2017a}), 2016 (solid dark blue,
\citeref{PandaXII2016a}) and 2017 (dashed dark blue, \citeref{PandaXII2017}) PandaX-II, XENON100 (solid green, \citeref{Aprile2012c}), and
XENON1T First Results (dashed green, \citeref{Aprile2017f}) are plotted.  Four typical super-symmetric (SUSY) models (CMSSM, NUHM1, NUHM2,
and pMSSM10) with constraints from ATLAS Run
1 are shown for reference (\citeref{Bagnaschi2015}).  The cross section of neutrino coherent scattering is marked by the orange dashed
line and shaded region.  Image credit: \citeref{Patrignani2016}.}
\label{fig:si_limits}
\end{figure}


% read https://www.hindawi.com/journals/ahep/2014/387493/ to rewrite above section