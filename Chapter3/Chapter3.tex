%This is the third chapter of the dissertation

%The following command starts your chapter. If you want different titles used in your ToC and at the top of the page throughout the chapter, you can specify those values here. Since Columbia doesn't want extra information in the headers and footers, the "Top of Page Title" value won't actually appear.

\pagestyle{cu}
\graphicspath{{./Chapter3/Figures/}}
\chapter[Table of Contents Title][Top of Page Title]{Title of Chapter 3}

% https://arxiv.org/pdf/1801.07231.pdf for electron emission from wires

Here you can write some introductory remarks about your chapter.
I like to give each sentence its own line.

When you need a new paragraph, just skip an extra line.

\section*{New Section}

By using the asterisk to start a new section, I keep the section from appearing in the table of contents.
If you want your sections to be numbered and to appear in the table of contents, remove the asterisk.

\begin{figure}
\includegraphics[\width=0.8\textwidth]{AbsorptionSpectra}
\caption{\citeref{Ozone2005}}
\end{figure}


In an electric field $E$ an \electron that is freed but does not recombine with its parent or other ionized atoms will move anti-parallel
to the field at drift velocity $v_{d}$.  For $E \lesssim 100\ \mathrm{V\ cm^{-1}}$ \vd$\propto E$, $100 \lesssim E \lesssim 10^{3-4}$
\vd$\propto E^{1/2}$, and $E \gtrsim 10^{4}$ \vd plateaus at $\sim 3\ \mathrm{mm\ \mu s^{-1}}$ (\citeref{Miller1968}).

\begin{table}
 \centering
 \begin{tabular}{cc}
 \hline
 $E$ [V cm$^{-1}$] & \vd [mm $\mu$s$^{-1}$] \\
 \hline
 $\lesssim 100$ & \vd$\propto E$ \\
 $\sim 100-10^{3-4}$ & \vd$\propto E^{1/2}$ \\
 $\gtrsim 10^{4}$ & \vd$\sim 3$ \\
 \hline
 \caption{Drift velocity \vd as a function of electric field $E$ for LXe}
 \end{tabular}
\end{table}

\begin{figure}
\includegraphics[angle=0.5, width=0.8\textwidth]{DriftVelocity}
\caption{Drift velocity for solid and liquid xenon}
\label{fig:drift_velocity}
\end{figure}

As the electron cloud drifts it will diffuse both longitudinally (in the direction of $E$) and transversely (perpendicular to $E$).  The
diffusion coefficients $D_{L}$ and $D_{T}$ are dependent on the electric field with $D_{T}/D_{L} \sim 10$.  The electron spread can
be written as $\sigma_{D_{T}} = \sqrt{D_{T} t_{d}}$ where $t_{d} = d/v_{d}$ is the drift time and $d$ is the drift distance.

Extensive xenon distillation and purification occurs before it is used in a detector.  Nonetheless impurities outgas from detector
material and contaminate the LXe.  Electronegative impurities in particular present a problem since they will attach to a free \electron,
lowering the number that reach the top of the detector and decreasing the secondary scintillation as shown in \eqref{eq:impurity_attach}.

\begin{equation}
e^{-} + S \rightarrow S^{-}
\label{eq:impurity_attach}
\end{equation}

\noindent The amount of \electron captured is dependent on the time in the LXe.  Thus an advantage of larger \efields is a larger
\vd (up to a point) and thus less time in the liquid.  Doping LXe with organic materials such as butane can increase \vd at higher
\efields but they are not used in DM detectors due to difficulty in purifying (\citeref{Yoshino1976}).  By setting the rate at which
electrons are absorbed by impurities $dq/dt = -qk_{S}S$ where $S$ is the impurity concentration and $k_{S}$ is the attachment rate
constant we find

\begin{equation}
q(t) = q_{0}e^{-tk_{S}S} = q_{0}e^{-t/\tau_{e}}
\label{eq:lifetime_equation}
\end{equation}

\noindent where $\tau_{e} = (k_{S}S)^{-1}$ and is known as the electron lifetime.  $k_{S}$ is shown in \figref{fig:attachment_rate} for
O$_{2}$,
N$_{2}$O, and SF$_{6}$.  We see that for N$_{2}$O the attaching rate constant increases with \efield whereas \otwo and SF$_{6}$
decerase.  Typically impurity concentration is given in O$_{2}$-equivalent values - that is, the concentration of \otwo if it was solely
responsible for \electron attachment.  For modeling electron lifetime it turns out that using the \otwo curve in
\figref{fig:attachment_rate} gives a good approximation.  Removing such impurities will be discussed in detail in \secref{}.

\begin{figure}
\includegraphics[width=0.8\textwidth]{AttachmentRate}
\caption{Attaching rate constant $k_{S}$ from \citeref{Bakale1976} for \otwo, N$_{2}$O, and SF$_{6}$ with respect to electric field.  At
larger \efield $k_{S}$ increases for N$_{2}$O and decreases for \otwo and SF$_{6}$.}
\label{fig:attachment_rate}
\end{figure}

In a TPC a cathode at the bottom of the detector applies an electric field in the LXe.  The \electron drift towards the top where a
grounded gate rests a few millimeters below the LXe surface.  Directly above the gate by a couple centimeters is the anode, which
applies a strong electric field that extracts the electrons into the gas xenon (GXe).  An extracted electron will ionize and excite
GXe atoms, whose freed electrons will do so as well in what is known as electroluminescence.  The number of ionized and excited atoms
is proportional to the number of \electron extracted, hence it is also known as proportional scintillation.  The number of photons
$N_{\mathrm{ph}}$ produced traveling a distance $z$ is

\begin{equation}
\frac{dN_{\mathrm{ph}}}{dz} = \alpha \Big( \frac{E_{g}}{P} - \beta \Big) P
\label{eq:electronlum}
\end{equation}

\noindent where $\alpha = 70\ \mathrm{photons\ kV^{-1}}$, $\beta = 1.0\ \mathrm{kV\ cm^{-1}\ atm^{-1}}$, and $E_{g}$ and $P$ are the
GXe electric field and pressure, respectively (\citeref{Belogurov1995}).

For PMT use Fig. 1 of Aprile2015