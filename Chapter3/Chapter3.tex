%This is the third chapter of the dissertation

%The following command starts your chapter. If you want different titles used in your ToC and at the top of the page throughout the chapter, you can specify those values here. Since Columbia doesn't want extra information in the headers and footers, the "Top of Page Title" value won't actually appear.

\pagestyle{cu}
\graphicspath{{./Chapter3/Figures/}}
\chapter[The XENON1T Dark Matter Search][The XENON1T Dark Matter Search]{The XENON1T Dark Matter Search}

% https://arxiv.org/pdf/1801.07231.pdf for electron emission from wires


XENON1T is the third generation experiment of the XENON collaboration.  With a fiducial mass of $> 1000\ \mathrm{kg}$ it is the first
liquid xenon dark matter detector to reach the ton-scale era of DM detection.  Its large target mass and low radioactive background
makes it the most sensitive detector to spin-independent WIMPs.

In this chapter I describe the XENON1T experiment (\secref{sec:xenon1t_detector}) and give the results of the second science run
(\secref{sec:xenon1t_sr1}).

Lots of good info in Aprile2017b (instrument paper).

\section{The XENON1T Detector}
\label{sec:xenon1t_detector}




\subsection{PMTs}
\label{subsec:xenon1t_pmts}
A total of 248 Hamamatsu R11410-21 PMTs are installed in XENON1T.  The 127 PMTs in the top array are placed in a radial distribution to
maximize resolution of $r$ position reconstruction.  The 121 in the bottom array are packed as densely as possible to maximize light
collection.  The R11410 window is 76.2 mm in diameter and the photocathode yields an average QE to 178 nm of 34.5\% with 2.8\%
standard deviation (\citeref{Aprile2017b, Barrow2017}).  The high QEs result from preselecting PMTs with $\mathrm{QE} > 28\%$ for
screening.

PMTs with the highest QE are placed in the bottom array while those with the lowest are stationed along the outside of the
top.  The difference in arrangement is strategic.  Due to liquid xenon's relatively large dielectric constant (1.95) an S1 will
often reflect off the surface and be redirected towards the bottom of the TPC.  For low-energy events - the relevant range for WIMP DM
searches - a nuclear recoils may only emit a small number ($\lesssim 100$) of photons, many of which never reach the PMTs.  Thus it is
most advantageous to position those with the highest quantum efficiency in the region most likely to see scintillation from an
S1.  Likewise, S2s easily produce enough scintillation to be observed by both arrays.  Therefore the QE of the top PMTs is comparatively
unimportant, and may even be advantageous for larger S2s where saturation can occur.  The layout of the PMTs with respect to QE is shown
in \figref{fig:xenon1t_pmt_qe}.

\begin{figure}
\centering
\includegraphics[width=\textwidth]{PMTQuantumEfficiency}
\caption{Quantum efficiency of top (left) and bottom (right) PMT arrays.  PMTs with highest QE are placed in the center of the bottom
array to maximize light collection while those with the lowest are placed in the outer region of the top.  Image credit:
\citeref{Aprile2017b}.}
\label{fig:xenon1t_pmt_qe}
\end{figure}

The two PMT arrays are supported by oxygen-free high thermal conductivity (OFHC) copper, with holes in which the PMTs are placed.  The
copper is then covered with polytetrafluoroethylene (PTFE) on the TPC-facing sides.  Screening meshes are situated between the bottom
array and cathode as well as the anode and top array.  Each can be biased to minimize electrical interference between the
phototubes and the drift and extraction fields.  While \citeref{Baudis2013} showed normal operation of R11410 PMTs at
$\geq 11\ \mathrm{kV\ cm^{-1}}$ the expected voltage for the cathode was $10-100\ \mathrm{kV\ cm^{-1}}$ and decreasing the stress on the
phototubes is likely to be beneficial longterm.  The finished arrays are shown in \figref{fig:xenon1t_pmt_array}.

\begin{figure}
    \centering
    \begin{subfigure}[t]{0.45\textwidth}
        \centering
        \includegraphics[height=4.5cm]{PMTTopArray}
    \end{subfigure}%
    \begin{subfigure}[t]{0.45\textwidth}
        \centering
        \includegraphics[height=4.5cm]{PMTBottomArray}
    \end{subfigure}
    \caption{XENON1T top (left) and bottom (right) PMT arrays.  Top PMTs are installed inside the diving bell in a radial distribution
    to minimize uncertainty in radial position reconstruction.  Bottom PMTs are installed below the cathode and screening mesh that
    limits interference between the PMT and cathode electric fields, both of which can be seen.  They are packed tightly
    together to maximize lightcollection.  Image credit: \citeref{Aprile2017b}.}
	\label{fig:xenon1t_pmt_array}
\end{figure}

\begin{figure}
\centering
\includegraphics[width=0.8\textwidth]{PMTSchematic}
\caption{Schematic of the R11410-21 PMT.
\label{fig:xenon1t_hamamatsu_pmt}
\end{figure}

The R11410-21 has 12 dynodes following the focusing electrode disk.  The first dynode is the largest and extends to the electrode to
maximize the probability of capturing photoelectrons.  A schematic can be be seen in \figref{fig:xenon1t_hamamatsu_pmt}.  The electrode,
dynode, and shield are stainless steel and are insulated with L-shaped quartz plates.  The window is
also made of quartz, since it is transparent to vacuum ultraviolet (VUV) photons.  Deposited on it is a low-temperature bialkali
photocathode.  The window is fixed with an aluminum seal to the faceplate flange, which along with the stem flange is constructed from
Kovar.  Because of the PMT body's large mass (71\% of total Kovar, 35\% of total) a low-\ce{^{60}Co} Kovar is chosen.  Finally, to insulate
the connections to each dynode the stem is ceramic.

Because radioactivity limits the fiducial volume and increases the event rate, making accidental coincidence and outlier events more
likely, XENON and Hamamatsu worked together to develop a highly radio-pure PMT.  There were several iterations of the R11410 model before
the R11410-21 was determined to be adequate.  Nearly all the \ce{^{137}Cs} and \ce{^{60}Co} comes from the Kovar, though the \ce{^{137}Cs}
content is negligible and the \ce{^{60}Co} is 3-10 times lower than older models.  The remaining screened isotopes, \ce{^{238}U},
\ce{^{228}Th}, \ce{^{228}Ra}, \ce{^{226}Ra}, and \ce{^{40}K}, are dominated by the ceramic stem (\citeref{Aprile2015}).  Unfortunately a
material that is more radio-pure and can insulate the dynode connections has not been found.  Sapphire was used in an iteration but
ultimately showed any improvement was minimal.

The dark count rate, or the number of signals per second above a threshold without a light source, is an important property to
characterize.  At ambient temperatures the primary cause is thermal electrons that scale with PMT voltage.  This becomes subdominant at
cryogenic temperatures to electron field emission and radioactivity (internal and external) as well as cosmic rays.  In a detector such
as XENON1T higher dark count rates make accidental coincidence more likely, which produces fake additional background and in the worst case
can place fake events in the signal region.  Because the rate is dependent on the threshold it can effectively be tuned.  However, because
for DM search we would like as low of a threshold as possible, choosing PMTs with low dark count rate is essential.

Another problematic feature is light emission from the phototube itself, where light is created inside the PMT and escapes through the
window.  It has been observed to mainly occur in one of two ways.  The first is through a discharge of intense light that can last for
several seconds.  This so-called ``flash" is bright enough to be easily observable to itself and by PMTs that are facing its
window.  However, the intensity can be so strong that it can take anywhere from several minutes to several hours for them to
recover.  Because they seem to occur spontaneously and are not well understood it is impossible to predict when a flash will occur.

The second variety is a subtle but often continuous stream of light.  Known as ``micro light emission" it is considerably harder to
identify.  Doing so requires facing two phototubes towards one another and measuring the dark rate of each one with and without the other
on.  The level of emission increases with temperature and bias voltage.  Keeping the PMTs at cryogenic temperatures during DM runs reduces
such effects, and voltages can be lowered to help further.  Still, if micro light emission continues the PMT cannot be used as it risks
contaminating the detected light from the true event population.

Directly following the pulse of a PMT hit a secondary pulse may occur.  Known as afterpulses, they can occur within 10s of
nanoseconds.  This can make them difficult to resolve from true pulses - especially S2s - that can have widths of several
microseconds.  However, they are an inevitable side effect when using PMTs so characterizing them properly is important.  There are three
mechanisms known to cause afterpulses.

The first is elastic scattering of the photoelectron with the first dynode, freeing \electron that shortly return to the dynode.  This
prompts afterpulses in the range of a few to tens of nanoseconds.  A second kind is thought to stem
from dark noise and single electrons but is not well understood.  It has a relatively uniform distribution in delay time up to several
microseconds.  Both of these afterpulses have relatively small areas of $\lesssim 2\ \mathrm{PE}$.

A photoelectron may occasionally ionize residual gas inside the PMT along its trajectory to the first dynode.  The molecule then drifts
towards the photocathode, expelling additional electrons.  The number of newly ejected electrons (area of afterpulse) depends
on the ion and the position of ionization.  The responsible ion can be determined by calculating the time between the true pulse and
afterpulse.  For R11410-21 this gives

\begin{equation}
\delta t_{\mathrm{ap}} = \frac{\pi}{4} \sqrt{\frac{2 m}{q V_{0}}} L
\end{equation}

where $\delta t_{\mathrm{ap}}$ is the delay time, $m$ and $q$ are the ion's mass and charge, and $V_{0}$ and $L$ are the potential
difference and length between the photocathode and first dynode (see \citeref{Barrow2017} for details).  Note that $\delta t_{\mathrm{ap}}$
does not depend on where the ionization occurred.  Thus, if we know the time between the true and afterpulse the ion - or more specifically
charge to mass ratio - can be calculated.  Pulse time differences range from several hundred nanoseconds to several microseconds.  These
correspond to the ``lines" in \figref{fig:xenon1t_pmts_ap} that extend to larger afterpulses.  The short timescales of S1s make them
unlikely to be grouped with an afterpulse, though for S2s this is more likely.

\begin{figure}
\centering
\includegraphics[width=\textwidth]{Afterpulse}
\caption{Afterpulses for one of the PMTs from XENON1T.}
\label{fig:xenon1t_pmts_ap}
\end{figure}

Because it is not possible to remove all residual gas any PMT will suffer from some ionization-based afterpulsing.  A better vacuum
corresponds to fewer afterpulses and a healthier PMT in general.  If the concentration of gas in the PMT vacuum were to increase it would
escalate the afterpulse rate.  Therefore if when using in xenon the Xe peaks grow over time the PMT most likely has a leak and should
be removed, as continued worsening of the vacuum will lead to deterioration and inability to operate the PMT.  All PMTs were tested before
being installed in XENON1T.  73 were rejected and replaced: 12 due to high dark count rates, 53 for light emission, and 8 for
afterpulsing.

The transit time (TT) is the time between the freed photoelectron and the arrival of the electron avalanche at the anode.  Variations in
the photoelectron's initial position as well as emitted velocity and angle cause deviations in the TT, which is characterized by the
transit time spread (TTS).  Because an event is observed by many PMTs the TTS quantifies how close together there signals should be.  Thus
smaller TTSs lead to a smaller integration window, decreasing accidental coincidence.  The TTS for all R11410-21 PMTs was measured, giving
a mean of $9.1 \pm 1.3\ \mathrm{ns}$ - a fraction of an S1.

The charge that reaches the anode can change for a single phototube and varies between different.  Increasing the bias voltage amplifies
the gain, heightening sensitivity or conversely lowering can help prevent saturation.  Gains may also deviate from variations in
temperature (\textbf{check this}), overexposure to light, count rate, wear, and more.  Phototubes may differ from one another
for all the reasons just mentioned, as well as inevitable - however tiny - differences in parts.  Monitoring the single photoelectron (SPE)
response for each PMT then is essential for accurately reconstructing the number of photoelectrons and by extension the energy of the
event.  This is typically done with an LED inside the TPC.  Typically blue light is used since ultraviolet is unavailable, which despite
being far outside of the 178 nm range ejects a photoelectron some small fraction of the time.  Each time the LED flashes we can record the
PMTs (during normal data taking this requires a pulse larger than some threshold, \secref{}).  After a large ($\mathcal{O}(10^{5})$)
number of trials a histogram can be fit to find the SPE gain as shown in \figref{fig:xenon1t_pmt_spe}.  Note that technically the charge
is plotted but the gain is easily found as $g = \mu_{e} / e$ where $\mu_{e}$ is the mean number of \electron and $e$ is the electron charge.  We
see a large peak centered at 0 that corresponds to the baseline noise when a photoelectron is not released.  At larger gain a series of
smaller wider peaks are visible that represent integer numbers of photoelectrons, starting with 1 at the leftmost.  There is a ``shoulder"
between the baseline and first peak that cannot be explained by only considering signal from integer PEs.  As briefly mentioned in
\secref{subsec:tpcs_pmts} this comes from photons that pass through the photocathode and free an electron on the first dynode, causing an
under-amplified charge.  This is generally the most challenging aspect to model.

In \figref{fig:xenon1t_pmt_spe} the spectrum is fit assuming the noise baseline and PE peaks are gaussian and the under-amplification is
an exponential.  The PE gaussians are constrained by $\mu_{N} = N \mu_{e}$ and $\sigma_{N} = \sqrt{N} \sigma_{e}$ where $\sigma_{e}$ is the
standard deviation of the single photoelectron peak and $\mu_{N}$ and $\sigma_{N}$ are the mean and standard deviation for the peak with
$N$ photoelectrons.  As this model is simplistic and can add bias recent efforts have gone in to alternative methods
of characterization (\secref{Saldanha2017, 2017Anthony}).  Nonetheless, the combined fit (green) appears to agree will with the data.

\begin{figure}
\centering
\includegraphics[width=0.8\textwidth]{SPESpectrum}
\caption{(Left) Single photoelectron spectrum.}
\label{fig:xenon1t_pmt_spe}
\end{figure}

We can use the SPE spectrum to calculate the resolution $R = \sigma_{e} / \mu_{e}$.  As the gain grows the resolution should sharpen
initially until it eventually levels off.  For the R11410-21 models the plateau began around $2-3 \times 10^{6}$ at a resolution of
27\%.  Because the stress on the PMT raises with bias voltage and larger gains lead to greater saturation we decided there was no benefit
to exceeding this gain.

The relatively poor resolution is apparent in \figref{fig:xenon1t_pmt_spe} as the photoelectron peaks are hard to distinguish.  An
important metric of the SPE spectrum is the peak-to-valley ratio, which divides the first PE peak by the valley that sits between it and
the baseline noise.  As with $R$ it increases for low gains and eventually levels.  For gains of $2-3 \times 10^{6}$ the mean
peak-to-valley ratio is ${\sim} 3$.

An additional six Hamamatsu R8520 PMTs reside in LXe outside the TPC near the top electrode for studying calibrations.  These PMTs have
been used in a number of LXe TPCs including XENON100, the predecessor to XENON1T (\citeref{Goetzke2017}, see \citeref{Aprile2012a} for
details on XENON100).




\subsection{TPC}
\label{subsec:xenon1t_tpc}
The XENON1T time projection chamber is cylindrical with a 97 cm height and 96 cm diamater.  It encloses a target mass of 2.0 tons where
light and charge can be measured.  A schematic is shown in \figref{fig:xenon1t_tpc_tpc}.  The interior of the vertical wall consists of 24
PTFE panels that were treated with diamond tools to
maximize VUV reflectivity.  Each interlocks with adjacent panels to achieve light-tightness, and the system is designed so that despite
the high thermal expansion coefficient the radius does not contract when lowered to $-96^{\circ}\ \mathrm{C}$.  Outside the PTFE are 74
field shaping rings made of low-radioactivity OFHC copper, each with a cross section of ${\sim} 10 \times 5\ \mathrm{mm^{2}}$.  They are
supported by 18 PTFE pillars stationed around the circumference.  Two redundant
chains connect adjoining rings via $5\ \mathrm{G \Omega}$ resistors, each with a $25\ \mathrm{G \Omega}$ resistor between the bottom and
cathode.

\begin{figure}
\centering
\includegraphics[width=0.8\textwidth]{XENON1TTPC}
\label{fig:xenon1t_tpc_tpc}
\end{figure}

There are five TPC electrodes that control the electric fields: the cathode, gate, anode, and top and bottom screening meshes.  They have
wired diameters of $\mathcal{O}(100)\ \mathrm{\mu m}$ and were designed to maximize S1 light collection.  The cathode is connected to a
PNC150000-1 NEG high voltage supply and pre-filling tests successfully reached voltages beyond -100 kV.  48
cm below the cathode is the bottom screening mesh (mentioned in \secref{subsec:xenon1t_pmts}).  The mesh is 12 mm above the bottom PMT
array and can be biased to reduce unwanted effects from the PMT and cathode E-fields.  The cathode and bottom screening mesh consist of
parallel wires and are gold-plated stainless steel, the latter of which increases the workfunction.  The gate rests just below the
liquid-gas interface and defines $z = 0$.  The anode is
situated 5 mm above the gate and is connected to a CAEN A1526P unit.  The fifth and final electrode is the top screening mesh 58 mm above
the anode and 11 mm below the top PMT array, and serves the same function is the same as the bottom mesh.  The top three electrodes are
made of stainless steel and are hex-etched.  Details for each electrode can be seen in \tabref{tab:xenon1t_tpc_electrodes} and
\figref{fig:xenon1t_tpc_efield} shows the simulated electric field for the settings during the first science run,Science Run 0.

\begin{table}
\centering
\begin{tabular}{cccccc}
\hline
Electrode & Type & Wire Diamater & Material & Transparency & Position \\
\hline
Top screening & hex meshed & $178\ \mathrm{\mu m}$ & stainless steel & 96.5\% & 63 mm \\
Anode & hex meshed & $178\ \mathrm{\mu m}$ & stainless steel & 89.8\% & 5 mm \\
Gate & hex meshed & $127 \mathrm{\mu m}$ & stainless steel & 92.7\% & 0 mm \\
Cathode & parallel wires & $216\ \mathrm{\mu m}$ & gold-plated stainless steel & 97.2\% & -969 mm \\
Bottom screening & parallel wires & $216\ \mathrm{\mu m}$ & gold-plated stainless steel & 97.2\% & -1017 mm \\
\hline
\end{tabular}
\caption{Properties for TPC electrodes.  The cathode and bottom screening mesh have high transparency to optimize S1 light collection and
are gold-plated to increase workfunction.}
\label{tab:xenon1t_tpc_electrodes}
\end{table}

\begin{figure}
\centering
\includegraphics[width=0.6\textwidth]{ElectricField}
\caption{Finite element (COMSOL Multiphysics) simulation for E-field inside and around the TPC.  Field and equipotential lines are
shown.  The voltages for the cathode, gate, and anode are -12, 0, and 4 keV, respectively, or the settings for Science Run 0.  The field
is mostly uniform throughout the detector, minimizing possible biases that come from recombination, impurity attachment, etc.  Image
credit: \citeref{Aprile2017b}.}
\label{fig:xenon1t_tpc_efield}
\end{figure}

Four parallel-plate capacitors measure the liquid level.  They have a range of 10 mm with $30\ \mathrm{\mu s}$ precision and can be used
for observing tilts or raising (lowering) the TPC.  The liquid level's height is adjustable a gas-exhaust tube, and maintained with a
so-called ``diving bell" that uses controlled gas flow from purification to pressurize the GXe inside the TPC.  Two cylindrical levelmeters
with a range of 1360 mm extend from the bottom PMT array to above the diving bell and are used for filling and recovery.  Placed around the
field cage are cable trays that power and transfer signal from the PMTs.  They are made of PTFE and the cables are held
in place by PTFE spacers (\textbf{check if spacers is right word}).



\subsection{Cryogenics}
\label{subsec:xenon1t_cryo}
The TPC is stationed in the center of the water Cherenkov detector (\secref{}) and is encompassed by the inner cryostat.  The cryostat
is stainless steel and electropolished to reduce radon emanation (\secref{}), since it is in direct contact with the xenon.  It is 1960 mm
tall by 1100 mm diamater and is metal-sealed
with Helicoflex.  Surrounding it is the 2490 mm tall by 1620 mm diamater outer cryostat.  It is also composed of stainless steel but
because there is no contact with xenon electropolishing is unncessary.  Materials were screened prior to construction for low-radioactivity
selection.  The two are thermally isolated by Torlon polyamide-imide spacers and
vacuum between them.  Heat loss is further mitigated to ${\sim} 75\ \mathrm{W}$ by aluminized mylar foil wrapped around the inner vessel.

A 10 m high stainless steel support structure was built inside the water tank.  Attached are three M20 rods that suspend the
cryostat.  They can be adjusted independently to tilt the cryostat, thereby changing the inclination of the LXe level with respect to the
TPC.  A chain secures the bottom of the outer vessel to the water tank floor to counteract buoyancy forces when the cryostat is empty.  A
double-walled pipe with inner and outer diameters 254 and 406 mm, respectively, connects with purification (\secref{subsec:xenon1t_pur}),
cooling,
pressurization for diving bell, and emergency recovery, and carries the PMT and auxiliary cables.  The voltage for the cathode is guided
in a separate pipe.  A schematic of the cryostat is shown

\begin{figure}
\centering
\includegraphics[width=0.6\textwidth]{CryostatDiagram}
\caption{Diagram of the cryostat.  Image credit: \citeref{Aprile2017c}.}
\label{fig:xenon1t_cryo_cryostat_diagram}
\end{figure}

A total of 3.5 tons of xenon is stored in the cryostat.  In addition to the 2.0 inside the TPC, the extra 1.5 is LXe between the
TPC and inner vessel or GXe above the liquid level.  The nominal LXe temperature is $T_{0} = -96^{\circ}\ \mathrm{C}$.  Gas near the top of
the cryostat is liquified by pulse-tube refrigerators (PTRs) into a funnel and flows back to the TPC through a designated pipe and
deposited in the inner vessel beneath the TPC.  Because
the PTRs are higher than the TPC the flow is guided by gravity.  \figref{fig:xenon1t_cryogenics_schmatic} shows the layout of the cryogenic
system.  The two PTRs occupy independent cooling towers and can each deliver ${\sim} 250\ \mathrm{W}$ of cooling power, and with
a total heat xenon heat load of ${\sim} 150\ \mathrm{W}$ only one needs to be active at any time.  The cooling towers along with much of the
GXe are located in the service building outside of the water tank  Each PTR connects to a copper cold finger
inside the inner cryostat so they can be removed without exposing the xenon to air.  A proportional-integral-derivative (PID) controller
monitors and adjusts the temperature of a resistive heater on the coldfinger to maintain stable pressure and temperature.

A third cooling tower uses liquid nitrogen ($\mathrm{LN_2}$) and is used in the case of an emergency if both PTRs are being serviced or the
heat load becomes too great from e.g. loss of power or insulation vacuum.  The $\mathrm{LN_2}$ flows from the $10 \mathrm{m^3}$ tank that
is used by ReStoX (\secref{subsec:xenon1t_restox}).  The cooling power can regulated by adjusting the $\mathrm{LN_2}$ evaporation rate.  In the event the PTRs
lose power the $\mathrm{LN_2}$ cooling tower must react immediately to prevent rising pressure that can be damaging to PMTs among other
things.  Sensors and controllers monitoring such a power loss are powered with uninterruptible power supplies (UPSs) to ensure a successful
transition and operation of $\mathrm{LN_2}$.

\begin{figure}
\centering
\includegraphics[width=\textwidth]{Fig7Aprile2017b}
\caption{Layout of the cryogenic system.  Three cooling towers - two with PTRs and one backup $\mathrm{LN_2}$ - liquify GXe to return to
the TPC.  LXe is carried from the bottom of the cryostat to the purification system (\secref{subsec:xenon1t_pur}) through a heat exchanger
system, where
returning xenon is inserted into the bottom of the TPC with excess gas diffusing into the GXe.  A fraction of purified GXe is extracted
before the heat exchanger and used to maintain pressure inside the diving bell.  ReStoX (\secref{}) connections are also shown.  Image
credit: \citeref{Aprile2017b}.}
\label{fig:xenon1t_cryogenics_schematic}
\end{figure}

Connections to purification and ReStoX are shown in \figref{fig:xenon1t_cryogenics_schematic}.  The LXe carried to the purification system
passes through a heat exchanger system discussed in \secref{subsec:xenon1t_pur}.  This is not the case for recuperation since the xenon
does not need to be gaseous.

LXe from the beneath the TPC (near liquid line from PTRs) is removed for purification.  Once the xenon removed from the LXe it passes
through a two-phase heat exchanger system.  The function of the heat exchanger is for GXe (returning to the TPC) and LXe (leaving) to
exchange thermal energy.  This way the returning xenon is cooled before it reaches the TPC, and likewise removed xenon is heated before
the purification.  The first step of the system is the ``tube-in-tube" component where two concentric tubes carrying LXe (GXe) away from
(towards)
the TPC.  The second component is a plate heat exchanger and is closer to the purification system (i.e. returning xenon passes through
before tube-in-tube).

We can define the heat exchange
efficiency $\epsilon$ as the fraction of heat necessary for temperature change and vaporization that stays outside of the system.  A higher
efficiency results in greater thermal energy transferred between the GXe and LXe and thus decreases the heat load on the system.  This is
because instead of GXe entering the cryostat at roughly
room temperature, it returns at approximately that of LXe, with much of it having condensed to LXe, putting less stress on the
PTRs.  An essential ingredient of the heat transfer is the latent heat, which comprises ${\sim} 80\%$ of the total exchange.  The difference
in vaporization temperature of the outgoing xenon and condensation temperature of incoming xenon is given by

\begin{equation}
\Delta T_{\mathrm{ph}} = T_{\mathrm{gl}} (P_i) - T_{\mathrm{gl}} (P_o)
\label{eq:xenon1t_pur_latent}
\end{equation}

where $T_{\mathrm{gl}} (P)$ is the temperature of the gas-liquid phase transition that depends on pressure, and $P_i$ and $P_o$ are the
pressures of the incoming and outgoing xenon, respectively.  Because the conditions of the dynamical gas flow
(\secref{subsec:xenon1t_pur}) cause $P_i > P_o$, $\Delta T_{\mathrm{ph}} > 0$, making it effective at heat transfer
(\citeref{Aprile2012b}).  A study with the Demonstrator - the
experiment used to for research and development for XENON1T - showed a heat exchange efficiency of
two heat exchangers in series of $\geq 96\%$ (\citeref{Aprile2012b}).  Following the parallel-plate exchanger a heater provides additional
thermal energy to the xenon moving towards the purification system.



\subsection{Purification}
\label{subsec:xenon1t_pur}
Contamination in LXe can present a number of problems.  As mentioned in \secref{subsec:tpcs_working_principle} electronegative
impurities (e.g. O$_2$, H$_2$O, N$_2$O, etc.) will attach to drifting $e^-$, reducing the number that reach
the liquid surface and thus the S2.

Impurities will also attenuate scintillation, which pushes our energy threshold up as fainter signals become harder to detect.  This is
described by

\begin{equation}
I(x) = I_0 e^{-x / \lambda_{\mathrm{att}}}
\label{eq:xenon1t_pur_atten}
\end{equation}

where $I_0$ is initial number of photons, $I(x)$ is the number after traveling a distance $x$, and $\lambda_{\mathrm{att}}$ is the
attenuation length.  $\lambda_{\mathrm{att}}$ is dependent on the absorption and scatter lengths, $\lambda_{\mathrm{abs}}$ and
$\lambda_{\mathrm{scat}}$.  The absorption length describes the true loss of photons while the scatter refers to photons that elastically
scatter without energy loss.  The attenuation length can then be written
$1 / \lambda_{\mathrm{att}} = 1 / \lambda_{\mathrm{abs}} + 1 / \lambda_{\mathrm{scat}}$.  \figref{fig:xenon1t_pur_absorption_spectra} shows
the Xe scintillation spectrum along with the absorption coefficients for \otwo and \htwoo vapor at concentrations of 1 ppm.  We can see
there is large overlap at lower wavelengths - especially for H$_2$O - and that observations under these circumstances would lead to an
incorrect measurement of the scintillation spectrum.  In addition to electronegative there are intrinsic
impurities including Kr and Rn but these are discussed in \secref{xenon1t_kr_dist}.

\begin{figure}
\centering
\includegraphics[width=0.8\textwidth]{AbsorptionSpectra}
\caption{Xe scintillation spectrum along with absorption coefficients for oxygen and water vapor present at 1 ppm.  \otwo has a maximum
value of overlap
with Xe of ${\sim} 0.1$ around 166 nm.  \htwoo covers a larger range of xenon's spectrum and has a much more significant impact, with a
maximum value of $> 0.4$ at approximately 172 nm and ${\sim} 0.2$ at 178 nm.  Image credit: \citeref{Ozone2005}.}
\label{fig:xenon1t_pur_absorption_spectra}
\end{figure}

\begin{figure}
\centering
\includegraphics[width=\textwidth]{Fig4.1FromOzone2005}
\label{fig:xenon1t_pur_absorp_dist}
\end{figure}

Electronegative impurities come primarily from outgassing of detector materials.  They have different \electron attachment rates (see
\figref{fig:attachment_rate} for examples) so are measured in O$_2$-equivalent - that is, the analogous concentration of oxygen if it
were the only contaminant.  For LXe experiments the necessary concentration is $\mathcal{O}(10^{-9})$ parts per billion (ppb) or less.

In addition
to the GXe from the TPC (after passing through heat exchanger and heater as described in \secref{subsec:xenon1t_cryo}), connections are
made between
each of the cooling towers and the purification system.  Because impurities are lighter than xenon we can anticipate them having a larger
presence in the gas.  While the GXe impurity concentration has no direct effect on \electron loss, impurities should migrate between the
LXe and GXe.  Purifying the GXe then should largely reduce the impurities that pass into the liquid.

Xenon from ReStoX  (\secref{subsec:xenon1t_restox}) and bottles are also connected to the purification system.  A separate heat exchanger
sits between ReStoX and purification but is less effective when xenon is not incoming and outgoing (not during detector filling).  In
addition to cleaning the Xe as it is transfered to the from RestoX to the cryostat, the purification system connects ReStoX to the Kr
distillation column (\secref{subsec:xenon1t_kr_dist}).  The column returns the xenon after distillation to purification.  All systems that
feed into the purification system do so before the getters and are available as outputs as well.  A small tube between the purification
and cryostat before the heat exchanger siphons GXe to the TPC bell to maintain pressure, and is regulated by a flow controller.  The
numerous connections makes the purification system a hub for transferring xenon between systems.

\begin{figure}
\centering
\includegraphics[width=0.8\textwidth]{Fig8Aprile2017b}
\label{fig:xeno1t_pur_schematic}
\end{figure}

The purification system consists of two redundant purification loops, each of which can operate independently of the other.  The flow
is driven by CHART QDrives, with one on one loop and two on the other.  They were chosen for their high-capacity and hermetically sealed
pump volume, the latter of which was not true with KNF pumps that were used in XENON100 (\textbf{check this}).  This is important because
exposure to air could cause a significant drop in purity, and has occurred in the past with KNFs.  They are
magnetically-resonant and operate via a compression space with oscillating externally-driven pistons.  Motors, valves, and pistons are
not lubricated, making it an excellent choice for high-purity.  Unfortunately QDrives emanate \ce{^{222}Rn} can require frequent
maintenance if operated at too great a voltage (for larger flow).  For the former \ce{Rn} emanation measurements were made prior to
installation with the lowest being chosen.  For the latter they are operated at safe conditions and since the beginning of Science Run 0
no issues have arisen.  It turns out the limiting factor on our circulation speed is not a result of the pumps but the pressure difference
from tubes leading to the system caused by too small a radius.  With the upgrade to XENONnT these will be replaced with larger-radii.  One
QDrive is installed in one loop and two in the other.

The flow of the GXe is maintained by two MKA 1579A mass-flow controllers - one on either branch.  The assumption before the experiment went
online was these would limit the speed when necessary, but because we did not achieve our goal these have minimal impact.  Still, they can
be used for further restriction.  The xenon then passes through SAES PS4-MT50-R high-temperature rare-gas purifiers, or getters.  They
use heated ($400^{\circ}\ \mathrm{C}$) zirconium to form unbreakable bonds with carbide, nitride, and oxide, lowering the impurity content
to < 1 ppb.  A bypass valve allows the xenon to pass un-purified in the event that the it does not need to be cleaned.

A Tiger Optics HALO+ H$_2$O monitor measures the water content in the purified GXe to track the purification efficiency.  It can measure
concentrations down to 400 ppt.  As with the getters, it can be bypassed.  The Xe is then forwarded to the cryostat, bell, ReStoX, or
bottles (Kr column occurs after the purification loops before the HALO).

The components of the system are electropolished and able to be baked up to ${\sim} 120^{\circ}\ \mathrm{C}$ for decreased outgassing during
operation.  The purification system has a number of valves that allow the versatility described in this section.  The majority are pneumatic and can
be operated remotely with the slow control system.  The remainder are manual valves situated in select locations: to and from the cryostat,
inside the calibration box, and before the vacuum pump, which can be used to evacuate individual sections or all of the system.  These are
in place as a precaution against accidental openings of the actuated valves that could produce problems.


\subsection{ReStoX}
\label{subsec:xenon1t_restox}
The Recovery and Storage system for XENON1T (ReStoX) hosts xenon not in use by other systems.  With a volume of 4.95 m$^{3}$
(diameter of 2.1 m), wall thickness of 28 mm, and ability to withstand pressures up to 73 bar, it can support as much as 7.6 tons of xenon
as a gas, liquid, or super-critical fluid.  It is constructed from stainless steel and insulated with vacuum from an outer sphere, where
thermal conductance is minimized with superinsulation and limited contact to an external heat load of ${\sim}50\ \mathrm{W}$.

As discussed in \secref{xenon1t_pur} the
purification system doubles as a router between the different systems.  This offers an advantage over previous LXe experiments.

\begin{figure}
\centering
\includegraphics[width=0.8\textwidth]{Fig10Aprile2017b}
\label{fig:xenon1t_restox_pic}
\end{figure}



\subsection{Kr Distillation}
\label{subsec:xenon1t_kr_dist}




In an electric field $E$ an \electron that is freed but does not recombine with its parent or other ionized atoms will move anti-parallel
to the field at drift velocity $v_{d}$.  For $E \lesssim 100\ \mathrm{V\ cm^{-1}}$ \vd$\propto E$, $100 \lesssim E \lesssim 10^{3-4}$
\vd$\propto E^{1/2}$, and $E \gtrsim 10^{4}$ \vd plateaus at $\sim 3\ \mathrm{mm\ \mu s^{-1}}$ (\citeref{Miller1968}).

\begin{table}
 \centering
 \begin{tabular}{cc}
 \hline
 $E$ [V cm$^{-1}$] & \vd [mm $\mu$s$^{-1}$] \\
 \hline
 $\lesssim 100$ & \vd$\propto E$ \\
 $\sim 100-10^{3-4}$ & \vd$\propto E^{1/2}$ \\
 $\gtrsim 10^{4}$ & \vd$\sim 3$ \\
 \hline
 \caption{Drift velocity \vd as a function of electric field $E$ for LXe}
 \end{tabular}
\end{table}

\begin{figure}
\includegraphics[angle=0.5, width=0.8\textwidth]{DriftVelocity}
\caption{Drift velocity for solid and liquid xenon}
\label{fig:drift_velocity}
\end{figure}

As the electron cloud drifts it will diffuse both longitudinally (in the direction of $E$) and transversely (perpendicular to $E$).  The
diffusion coefficients $D_{L}$ and $D_{T}$ are dependent on the electric field with $D_{T}/D_{L} \sim 10$.  The electron spread can
be written as $\sigma_{D_{T}} = \sqrt{D_{T} t_{d}}$ where $t_{d} = d/v_{d}$ is the drift time and $d$ is the drift distance.

Extensive xenon distillation and purification occurs before it is used in a detector.  Nonetheless impurities outgas from detector
material and contaminate the LXe.  Electronegative impurities in particular present a problem since they will attach to a free \electron,
lowering the number that reach the top of the detector and decreasing the secondary scintillation as shown in \eqref{eq:impurity_attach}.

\begin{equation}
e^{-} + S \rightarrow S^{-}
\label{eq:impurity_attach}
\end{equation}

\noindent The amount of \electron captured is dependent on the time in the LXe.  Thus an advantage of larger \efields is a larger
\vd (up to a point) and thus less time in the liquid.  Doping LXe with organic materials such as butane can increase \vd at higher
\efields but they are not used in DM detectors due to difficulty in purifying (\citeref{Yoshino1976}).  By setting the rate at which
electrons are absorbed by impurities $dq/dt = -qk_{S}S$ where $S$ is the impurity concentration and $k_{S}$ is the attachment rate
constant we find

\begin{equation}
q(t) = q_{0}e^{-tk_{S}S} = q_{0}e^{-t/\tau_{e}}
\label{eq:lifetime_equation}
\end{equation}

\noindent where $\tau_{e} = (k_{S}S)^{-1}$ and is known as the electron lifetime.  $k_{S}$ is shown in \figref{fig:attachment_rate} for
O$_{2}$,
N$_{2}$O, and SF$_{6}$.  We see that for N$_{2}$O the attaching rate constant increases with \efield whereas \otwo and SF$_{6}$
decerase.  Typically impurity concentration is given in O$_{2}$-equivalent values - that is, the concentration of \otwo if it was solely
responsible for \electron attachment.  For modeling electron lifetime it turns out that using the \otwo curve in
\figref{fig:attachment_rate} gives a good approximation.  Removing such impurities will be discussed in detail in \secref{}.

\begin{figure}
\includegraphics[width=0.8\textwidth]{AttachmentRate}
\caption{Attaching rate constant $k_{S}$ from \citeref{Bakale1976} for \otwo, N$_{2}$O, and SF$_{6}$ with respect to electric field.  At
larger \efield $k_{S}$ increases for N$_{2}$O and decreases for \otwo and SF$_{6}$.}
\label{fig:attachment_rate}
\end{figure}

In a TPC a cathode at the bottom of the detector applies an electric field in the LXe.  The \electron drift towards the top where a
grounded gate rests a few millimeters below the LXe surface.  Directly above the gate by a couple centimeters is the anode, which
applies a strong electric field that extracts the electrons into the gas xenon (GXe).  An extracted electron will ionize and excite
GXe atoms, whose freed electrons will do so as well in what is known as electroluminescence.  The number of ionized and excited atoms
is proportional to the number of \electron extracted, hence it is also known as proportional scintillation.  The number of photons
$N_{\mathrm{ph}}$ produced traveling a distance $z$ is

\begin{equation}
\frac{dN_{\mathrm{ph}}}{dz} = \alpha \Big( \frac{E_{g}}{P} - \beta \Big) P
\label{eq:electronlum}
\end{equation}

\noindent where $\alpha = 70\ \mathrm{photons\ kV^{-1}}$, $\beta = 1.0\ \mathrm{kV\ cm^{-1}\ atm^{-1}}$, and $E_{g}$ and $P$ are the
GXe electric field and pressure, respectively (\citeref{Belogurov1995}).

For PMT use Fig. 1 of Aprile2015