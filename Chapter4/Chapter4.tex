%This is the third chapter of the dissertation

%The following command starts your chapter. If you want different titles used in your ToC and at the top of the page throughout the chapter, you can specify those values here. Since Columbia doesn't want extra information in the headers and footers, the "Top of Page Title" value won't actually appear.

\pagestyle{cu}
\graphicspath{{./Chapter4/Figures/}}
\chapter[Electronic and Nuclear Recoil Bands][Electronic and Nuclear Recoil Bands]{Electronic and Nuclear Recoil Bands}
\label{chap:er_nr_calibrations}

% https://arxiv.org/pdf/1801.07231.pdf for electron emission from wires


A band is the region in cS1-\cstwob (or some variation thereof) space where events of a certain type occur.  To
differentiate the nuclear from electronic recoil bands calibrations for each are performed and fit.  These are critical as
they define the reference region and discrimination power between potential WIMP signals and background.  \ce{^{220}Rn} calibrations are
used for the ER band and are are performed regularly.  For NR \ce{^{241}AmBe} and a deuterium-deuterium
neutron generator (NG) are used, both of which are positioned in the water tank outside the cryostat.  \ambe calibrations are done in
both science runs and a NG calibration is performed in SR1 only.

For this analysis all five sets of calibration data from both science runs were fit simultaneously.  This was
significant because it forced values that are common to some or all of the calibrations (e.g. $W$, $p_{\mathrm{dpe}}$ are shared
between all five, while $g_1$, $g_{2\mathrm{b}}$ are shared within science runs) to be described by a single
value.\footnote{for First Results the \radoncal and \ambe were independently fit.}

This chapter presents the electronic and nuclear recoil band fitting used in the run-combined analysis (\chapref{chap:xenon1t}).  It
begins with a summary of the importance of the calibrations (\secref{sec:er_nr_calibrations_purpose}) followed by a description of how
the parameters in the fit are determined (\secref{sec:er_nr_calibrations_parameter_determ}).  Lastly, the results of the fit are
presented (\secref{sec:er_nr_calibrations_results}).



\section{Purpose of Calibrations}
\label{sec:er_nr_calibrations_purpose}
For any event event we would like ot be able to calculate the probabilities of its possible origins.  Fitting the electronic and nuclear
recoil bands creates probability density functions (PDFs) for our largest background and region of interest.  Since they overlap
(the ER $-2\sigma$ is at roughly the NR median) it is important to carefully characterize their PDFs so events can be classified
as accurately as possible.  This approach improves sensitivity over
purely statistical approaches (\secref{sec:direct_detect}).  The method in this section requires a complete understanding of the TPC
background, which
in addition to electronic recoils includes wall leakage (\secref{subsec:backgrounds_detector_materials}) and accidental
coincidence (\secref{subsec:backgrounds_ac}).  It also demands that
the physical processes that occur starting from the exciton-to-ion ratio through the data acquisition are correctly modeled and can be
simulated on a reasonable time scale.

Following the ER and NR fits a signal PDF is generated.  For this analysis the signal is a spin-independent WIMP but in general it is not
constrained to this.  The shape of the PDF depends on the WIMP mass.  The data can be compared
to the PDF for a given WIMP mass and cross-section to see how well it agrees with a signal.  A limit is set where
cross-sections below some value match the data, while a discovery can be claimed if the data prefers a signal with background model.

\figref{fig:er_nr_calibrations_purpose_wimp_contours} shows the background model with contours for 6, 50 and $1000\ \mathrm{GeV/c^2}$
WIMPs overlaid from
First Results.  The ER band is the bright yellow region and the horizontal band around $\mathrm{cS2_b} = 150\ \mathrm{PE}$ is accidental
coincidence.  The overlap between signal and background emphasizes the importance of electronic and nuclear recoil band
fitting.  An incorrect fit might shift the positions of the bands, changing our understanding of the origin of events.

\begin{figure}
\centering
\includegraphics[width=\textwidth]{wimp_contours}
\caption{Total background model from First Results with 50, 90, and 99\% contours for 6 (left), 50 (center), and 1000 (right)
$\mathrm{GeV/c^2}$ WIMPs.  The bright yellow region is the
electronic recoil band, which was fit using a similar method to this section.  Accidental coincidence is visible as the horizontal strip
around $\mathrm{cS2_b} = 150\ \mathrm{PE}$.  The expected signal region changes with WIMP mass.}
\label{fig:er_nr_calibrations_purpose_wimp_contours}
\end{figure}



\section{Parameter Determination}
\label{sec:er_nr_calibrations_parameter_determ}
This approach compares the distributions of Monte Carlo events and data.  Because of the nearly flat distribution of events in the energy
region of interest and homogenous position distribution, Monte Carlo features $E_{\mathrm{true}}$, $x_{\mathrm{true}}$,
$y_{\mathrm{true}}$, and $z_{\mathrm{true}}$ for \ce{^{220}Rn} are each randomly drawn from uniform distributions.  \ce{^{241}AmBe} and
neutron generator Monte Carlo are simulated using GEANT4 \citeref{Agostinelli2003}, and additionally include the number of
scatters for each neutron.  The $\beta^-$ from \ce{^{220}Rn} should not scatter so it is fixed to 1.  These values represent the
``truth'' information - that is, these are the true interaction values and are inputs for the fits.  Reconstructed $x \mdash y$ values
will be referred to as $x_{\mathrm{rec}}$
and $y_{\mathrm{rec}}$ (energy and depth will be denoted $E$ and $z$ and are equivalent to the truth values).

\figref{fig:er_nr_calibrations_parameter_determ_flow_chart} shows the flow chart for the band matching.  Each Monte Carlo event is run
through a series of events that mimic our best understanding of the expected sequence as closely as possible.  The parameters in the fit
come from
a variety of sources including microphysics reported from literature and detector characterization measurements
(\secref{sec:det_char}).  The process takes the simulated energy from a Monte Carlo event and outputs observables cS1 and
$\mathrm{cS2_b}$.  These in turn are binned in cS1-$\mathrm{log_{10}(cS2_b / cS1)}$ space, and the likelihood for equivalent bins
between data and MC is calculated (\secref{subsec:er_nr_calibrations_parameter_determ_mc_match}).  To describe our energy region of
interest as well as possible only events with $0 \leq \mathrm{cS1} \leq 100$ and $0.5 \leq \mathrm{log_{10}(cS2_b / cS1)} \leq 3.5$ are
considered.  As the parameters vary the likelihood will increase or decrease and the minimizer will find the best-fit values.

\begin{figure}
\centering
\includegraphics[width=\textwidth]{FlowChart}
\label{fig:er_nr_calibrations_parameter_determ_flow_chart}
\end{figure}

The process of converting the truth MC to cS1 and \cstwob is referred to as a ``fast MC'' (random numbers are drawn from probability
distributions to select sets of parameters).  ``Truth MC'' refers to the simulated events before the fast MC ($E_{\mathrm{true}}$,
$x_{\mathrm{true}}$, $y_{\mathrm{true}}$, $z_{\mathrm{true}}$, and number
of scatters).  ``MC'' will denote the sum of all the fast MC results, usually in the form of a PDF or histogram, for
comparison with data.  Finally, a Markov Chain Monte Carlo (MCMC) is used to fit the data and Monte Carlo and the best-fit values
are derived from the posterior.  \secref{subsec:er_nr_calibrations_parameter_determ_er} (ER) and
\secref{subsec:er_nr_calibrations_parameter_determ_nr} (NR) discusses the physical steps that occur beginning with energy
deposition and ending with cS1 and $\mathrm{cS2_b}$, which are modeled in the fast MC.

The number of fast MC events simulated for each likelihood iteration is $\mathcal{O}(10^6)$.  A fast MC event randomly selects one of the
truth input events.  Such large statistics are necessary to reliably model the bands;
however, nominal running time is far higher than any sensible time-scale.  This was solved by using graphical processing units (GPUs),
which can run the events in parallel, providing a boost in speed by a factor of $10^{2 \mdash 3}$ and reducing the required time to a
reasonable level.

The calibrations were fit in the First Results FV ($-92.9 < z < -9\ \mathrm{cm}$, $r_{\mathrm{rec}} < 36.94\ \mathrm{cm}$
\figref{fig:calibrations_position_reconstruction}) despite the 1.3 t fiducial mass used for the dark matter analysis.  This was done to
minimize wall events and other contamination from materials at the top and bottom of the TPC.



\subsection{Electronic Recoils}
\label{subsec:er_nr_calibrations_parameter_determ_er}
The electronic recoil calibration is performed using \ce{^{220}Rn} (decay chain shown in the left panel of
\figref{fig:backgrounds_decay_chains}).  It is used because of the \ce{^{212}Pb} \betadecay (Q-value of 569.9 keV) and can
be performed as an internal calibration.  In the future we hope to use tritiated methane $\mathrm{C H_3 T}$, which undergoes \betadecay
with a maximum energy of 18.6 keV.  This would provide significantly more statistics in our region of interest in a shorter amount
of time.  However, as mentioned in \secref{subsubsec:xenon1t_calibrations_internal} $\mathrm{C H_3 T}$ has a half-life of 12.3 y and there
is concern it may attach to the cryostat.  For this reason we instead use \ce{^{220}Rn}, for which a total of six calibrations were
performed during SR0 and SR1.

When energy is deposited in LXe a number of quanta $n_q$ will be produced.  For electronic recoils this follows the normal distribution

\begin{equation}
n_q \sim \mathrm{Norm} \bigg( \mu = \frac{E}{W},\ \sigma^2 = \frac{E F}{W} \bigg)
\label{eq:er_nr_calibrations_parameter_determ_er_quanta}
\end{equation}

\noindent where $F$ is the Fano Factor \citeref{Fano1947} discussed in \secref{sec:er}.  Its derivation and subsequent measurements
demonstrates that fluctuations in $n_q$ are smaller than a Poisson distribution, and is estimated to be $F = 0.059$
\citeref{Doke1976}.  For the
fast MC $F$ is fixed.  Although a later measurement found different best-fit values, their result was
compatible with \citeref{Doke1976} when including uncertainty \citeref{Seguinot1995}.  $W$, the average energy to produce a single
quanta, is constrained by a Gaussian with $\mu = 13.7,\ \sigma = 0.2\ \mathrm{eV}$ following the measurement of \citeref{Dahl2009}.

For electronic recoils quenching is negligible so is not considered.  The quanta will be divided into excitons and electron-ion pairs
$n_q = n_{\mathrm{ex}} + n_{\mathrm{ion}}$ according to a binomial distribution

\begin{equation}
n_{\mathrm{ion}} \sim \mathrm{Binom} \Bigg(n = n_{\mathrm{q}},\ p = \frac{1}{1 + \frac{n_{\mathrm{ex}}}{n_{\mathrm{ion}}}} \Bigg)
\label{eq:er_nr_calibrations_parameter_determ_er_nions}
\end{equation}

\noindent where $n_{\mathrm{ex}} / n_{\mathrm{ion}}$ is constrained $0.06 \mdash 0.2$ and expected to be energy independent
\citeref{NEST2011}.  Some \electron will recombine with \ce{Xe^+} to form
excitons and emit photons as they decay to the ground state.  The recombination fraction $r$ depends on the field in the LXe and the
interaction energy, and has intrinsic fluctuations $\Delta r$ \citeref{LUX2016, Aprile2018a}.  A truncated Gaussian is assumed

\begin{equation}
r \sim \mathrm{Norm} \Big( \mu = \langle r \rangle,\ \sigma^2 = (\Delta r)^2 \Big)
\end{equation}

\noindent where $0 \leq r, \Delta r \leq 1$ and is parameterized using a modified Thomas-Imel box model \citeref{Thomas1987}

\begin{equation}
\langle r \rangle = \frac{1}{1 + e^{-(E - E_0) / E_1}}
\bigg( 1 - \frac{\mathrm{ln}(1 + n_{\mathrm{ion}} \varsigma / 4)}{n_{\mathrm{ion}} \varsigma / 4} \bigg)
,\ \ \varsigma = \gamma_{\mathrm{er}} e^{-E / \omega_{\mathrm{er}}} E_d^{-\delta_{\mathrm{er}}}
\label{eq:er_nr_calibrations_parameter_determ_ti}
\end{equation}

\noindent for the fit.  Here $\varsigma$ has been adapted from \eqnref{eq:ti_recomb} to include a power law field-dependence to allow
simultaneous fitting of SR0 and SR1, and an exponential energy term to extend compatibility to high energy
(${\sim} 20\ \mathrm{keV_{ee}}$).  An energy-dependent Fermi-Dirac suppresses recombination at
$\lesssim 2\ \mathrm{keV_{ee}}$ for better agreement with data.  The parameters do not have well-defined priors and are constrained to
$0 \leq \gamma_{\mathrm{er}} \leq 0.5$, $0 \leq E_0, E_1 < \infty$, and
$-\infty < \omega_{\mathrm{er}}, \delta_{\mathrm{er}} < \infty$.

Recombination fluctuations are modeled as

\begin{equation}
\Delta r = A(1 - e^{-E/B})
\label{eq:er_nr_calibrations_parameter_determ_er_rec_fluctuations}
\end{equation}

\noindent where parameters $A,B > 0$ are allowed to vary freely.  The number of electron-ion pairs that recombine is

\begin{equation}
n_{\mathrm{rec}} \sim \mathrm{Binom} \big(n = n_{\mathrm{ion}},\ p = r \big)
\end{equation}

\noindent yielding final numbers of photons and electrons

\begin{subequations}
\begin{align}
n_{\mathrm{ph}} &= n_{\mathrm{ex}} + r\, n_{\mathrm{ion}} \\
n_{\mathrm{e}} &= (1 - r) n_{\mathrm{ion}}
\end{align}
\end{subequations}

\noindent  As mentioned above \ce{^{220}Rn} is homogeneously spread throughout the TPC and has an energy spectrum that is nearly
flat between 0-30 keV so each event's truth energies and positions are randomly sampled from uniform distributions before the fit begins.



\subsection{Nuclear Recoils}
\label{subsec:er_nr_calibrations_parameter_determ_nr}
Two nuclear recoil sources were used for calibrations: americium beryllium (\ce{^{241}AmBe}) and a neutron generator
(NG).  \ambe was used once in both SR0 and SR1 and the NG was used in SR1.  \ambe decays to \ce{^{237}Np} by $\alpha$-emission and
the large \ce{^{9}Be} $\alpha$ cross-section prompts a second decay

\begin{equation}
\mathrm{^{9}Be} + \mathrm{^{4}He} \rightarrow \mathrm{^{12}C + n} + \gamma
\end{equation}

\noindent emitting a $< 11\ \mathrm{MeV}$ neutron.  The neutron generator (NSD Gradel Fusion NSD-35-DD-C-W-S) uses
deuterium-deuterium (D-D) fusion

\begin{equation}
\mathrm{^{2}D} + \mathrm{^{2}D} \rightarrow \mathrm{^{3}He} + \mathrm{n}
\end{equation}

\noindent where the \ce{^{3}He} and n are expelled at 0.82 and 2.45 MeV, respectively.  The energy spectrum can be seen in
\figref{fig:er_nr_calibrations_parameter_determ_nr_ng_energy}.  It shows two peaks - one at 2.2 MeV and the other at 2.7 MeV.  These
are the energies seen in the lab frame (2.45 MeV was in the deuteron's) and correspond to neutrons emitted at $180^{\circ}$ and
$0^{\circ}$, respectively.  The NG can operate at rates as low as $10\ \mathrm{n\ s^{-1}}$
and as high as $10^7\ \mathrm{n\ s^{-1}}$, which allows flexibility for our objectives.  A higher-energy neutron
population is produced by tritium via

\begin{figure}
\centering
\includegraphics[width=0.6\textwidth]{ng_energy_spectrum}
\caption{Simulated neutron energy spectra in fusion region of the NG (red dashed line) and at the pulse shape discriminator during
calibration before installation at LNGS (blue shaded region).  Deconvolution of data from calibration (black solid line) is also
shown.  Spectra are normalized by the 2.2-2.7 MeV range.  Image credit: \citeref{Lang2018}.}
\label{fig:er_nr_calibrations_parameter_determ_nr_ng_energy}
\end{figure}

\begin{subequations}
\begin{align}
\mathrm{^{2}D} + \mathrm{^{2}D} &\rightarrow \mathrm{^{3}T} + \mathrm{p} \\
\mathrm{^{3}T} + \mathrm{^{2}D} &\rightarrow \mathrm{^{4}He} + \mathrm{n}
\end{align}
\end{subequations}

\noindent where the neutron is expelled at 14.1 MeV.  The contribution of deuterium-tritium fusion was measured to be $3.5 \pm 0.2\%$
before installation at LNGS \citeref{Lang2018}.  Because the tritium is created in deuterium-deuterium fusion and has
$t_{1/2} = 12.3\ \mathrm{y}$ its fraction will increase with continued use.  For details on the characterization of the NG
please refer to \citeref{Lang2018}.

The positions for \ambe and NG events are shown in \figref{fig:er_nr_calibrations_parameter_determ_nr_ambe_positions} and
\figref{fig:er_nr_calibrations_parameter_determ_nr_ng_positions}.  Their ${\sim}10\ \mathrm{cm}$ mean free path causes clustering
closest to the calibration source.  Because we expect our
detector does not vary over $r$ and $\phi$ inside the 1 t FV the results of the asymmetric position distribution are applied to the entire
active volume.  Events in the region furthest from the source are mostly background.  Fewer background events are present in the NG data
because of the shorter calibration time.  SR0 \ambe is not shown but its distribution of NR events is similar to
\figref{fig:er_nr_calibrations_parameter_determ_nr_ambe_positions} because the location of the source was the same.  Many more ER events,
however, were present since it preceded the online distillation (\secref{subsec:xenon1t_kr_dist}).

\begin{figure}
\centering
\includegraphics[width=\textwidth]{ambe_positions}
\caption{Positions of events from SR1 \ambe calibration.  Events are clustered in the region of the detector closest to the \ambe
source.  The black line marks the 1 t fiducial volume.  Aside from the higher ER background, the SR0 distribution looks similar since the
source was placed in the same location inside the water tank.}
\label{fig:er_nr_calibrations_parameter_determ_nr_ambe_positions}
\end{figure}

\begin{figure}
\centering
\includegraphics[width=\textwidth]{ng_positions}
\caption{Positions of events from SR1 neutron generator calibration.  Events are densely distributed in the region of the TPC closest to
the NG source.  The black line shows the 1 t fiducial volume.}
\label{fig:er_nr_calibrations_parameter_determ_nr_ng_positions}
\end{figure}

The microphysical processes that follow a nuclear recoil differ from ER (\secref{subsec:er_nr_calibrations_parameter_determ_er}).  As
discussed in \secref{sec:nr} a sizable portion of the energy is lost to atomic motion.  To model this we use the Lindhard theory
in \eqnref{eq:er_nr_calibrations_parameter_determ_nr_lindhard}.

\begin{subequations}
\begin{align}
\epsilon &= 11.5 \bigg( \frac{E}{\mathrm{keV}} \bigg) Z^{-7/3} \\
g( \epsilon ) &= 3 \epsilon ^{0.15} + 0.7 \epsilon ^{0.6} + \epsilon \\
L( \epsilon ) &= \frac{k g( \epsilon ) }{1 + k g( \epsilon )}
\end{align}
\label{eq:er_nr_calibrations_parameter_determ_nr_lindhard}
\end{subequations}

$k$ is the proportionality constant between the electronic stopping power and recoiling nucleus velocity.  The Lindhard factor
$L(\epsilon )$ is the fraction of energy converted to excitons or electron-ion pairs

\begin{equation}
n_{\mathrm{q}} \sim \mathrm{P} \bigg( \mu = \frac{E L}{W} \bigg)
\label{eq:er_nr_calibrations_parameter_determ_nr_quanta}
\end{equation}

\noindent that uses a Poisson distribution as an approximation.  In reality the track structure of nuclear recoils makes the true
distribution more complicated.  The number of ions is found in the same way as electronic recoils

\begin{equation}
n_{\mathrm{ion}} \sim \mathrm{Binom} \Bigg(n = n_{\mathrm{q}},\ p = \frac{1}{1 + \frac{n_{\mathrm{ex}}}{n_{\mathrm{ion}}}} \Bigg)
\end{equation}

\noindent and $n_{\mathrm{ex}} = n_{\mathrm{q}} - n_{\mathrm{ion}}$.  Recombination is given by

\begin{equation}
n_{\mathrm{rec}} \sim \mathrm{Binom} \big(n = n_{\mathrm{ion}},\ p = r \big)
\end{equation}

\noindent where $r$ described by the Thomas-Imel model without the Fermi-Dirac tuning

\begin{equation}
r = 1 - \frac{\mathrm{ln} (1 + n_{\mathrm{ion}} \varsigma)}{n_{\mathrm{ion}} \varsigma}
\end{equation}

\noindent where $\varsigma$ is field-dependent.  Unlike ER, no recombination fluctuations have been observed for nuclear recoils so
$\Delta r$ is not included.  Biexcitonic quenching and Penning process, both of which decrease
$\mathrm{n_{ph}}$, must also be considered.  Biexcitonic quenching arises when two
\ce{Xe^{*}} interact to free an $e^-$, which quickly loses its kinetic energy
and recombines with a \ce{Xe^{+}}.  Thus what would have been two photons instead results in one.  The Penning process describes when
two excimers interact and result in an excited and ground state \citeref{Mei2008}.  Both of these depend on the exciton density, which
is proportional to the ionization density and therefore stopping power $dE / dx$.  They result in quenching and can be
described by Birks' saturation law, \eqnref{eq:er_nr_calibrations_parameter_determ_nr_birks} \citeref{Birks1951, Birks1964}.

\begin{equation}
f_b = \frac{1}{1 + k_{B}B \frac{dE}{dx}} = \frac{1}{1 + \eta \epsilon^{\lambda}}
\label{eq:er_nr_calibrations_parameter_determ_nr_birks}
\end{equation}

\noindent where $k_B$ is Birks' constant (calculated in \citeref{Mei2008} to be $2.015 \times 10^{-3}\ \mathrm{g\ MeV^{-1}\ cm^{-2}}$),
$B$ is the coefficient for the stopping power, and $\eta$ is defined to be their product.  The total quenching from biexcitonic
quenching and the Penning process is

\begin{equation}
n_{\mathrm{quench}} = \mathrm{Binom} \big( n = n_{\mathrm{ex}},\ p = f_b \big)
\end{equation}

\noindent reducing $n_{\mathrm{ex}} \rightarrow n_{\mathrm{ex}} - n_{\mathrm{quench}} \rightarrow n_{\mathrm{ex}}$.

The parameters in the nuclear recoil model were constrained by previous measurements of light and charge
yield.  This decision was made to prevent detector effects from compensating for changes
in the liquid xenon response model.  We used the results from \citeref{NEST2015}, which used nuclear recoil light and charge yield
measurements between $1 \mdash 300\ \mathrm{keV}$ and electric fields of $0 \mdash 4060\ \mathrm{V\ cm^{-1}}$ to fit the model
detailed in this section.  Although none of the measurements were made for light and charge yields simultaneously, \citeref{NEST2015}
performed a single fit of all data.

The model is parameterized by setting

\begin{equation}
\frac{n_{\mathrm{ex}}}{n_{\mathrm{ion}}} = \alpha E_{d}^{-\zeta} ( 1 - e^{-\beta \epsilon})
\label{eq:er_nr_calibrations_parameter_determ_nr_nex_nion}
\end{equation}

\begin{equation}
\varsigma = \gamma E_{d}^{- \delta}
\label{eq:er_nr_calibrations_parameter_determ_nr_recomb_sigma}
\end{equation}

\noindent where, $E_d$ is the drift field ($E$ is energy).  From \eqnref{eq:er_nr_calibrations_parameter_determ_nr_nex_nion}
$n_{\mathrm{ex}} / n_{\mathrm{ion}}$ has a power dependence on $E_d$,
which is suspected to be caused by geminate recombination: when electrons and parent ions recombine much more quickly than Thomas-Imel
recombination.  In addition $n_{\mathrm{ex}} / n_{\mathrm{ion}}$ depends exponentially on energy such that at higher $E$ a larger fraction
of excitons is favored.

We see in \eqnref{eq:er_nr_calibrations_parameter_determ_nr_recomb_sigma} that $\varsigma$ is proportional to a power of $E_d$.  The
Thomas-Imel theory predicts

\begin{equation}
\varsigma = \frac{\alpha^{\prime}}{4 a^2 u_- E_d}
\end{equation}

\noindent where $\alpha ^{\prime}$ is a recombination coefficient, $a$ is the dimension of the box, and $u_-$ is the electron
mobility.  While here $\delta$ is strictly 1, Dahl's model predicts $\delta \sim 0.1$ \citeref{Dahl2009}.

The marginalized posteriors from \citeref{NEST2015} of $k$ (\eqnref{eq:er_nr_calibrations_parameter_determ_nr_lindhard}), $\eta$ and
$\lambda$ (\eqnref{eq:er_nr_calibrations_parameter_determ_nr_birks}), $\alpha$, $\zeta$ and $\beta$
(\eqnref{eq:er_nr_calibrations_parameter_determ_nr_nex_nion}), and $\gamma$ and $\delta$
(\eqnref{eq:er_nr_calibrations_parameter_determ_nr_recomb_sigma}) are listed in
\tabref{tab:er_nr_calibrations_parameter_determ_nr_nest}.  They are used as gaussian priors for this analysis.

\bgroup
\def\arraystretch{1.2}
\begin{table}
\centering
\begin{tabular}{cccc}
\hline
\hline
Parameter & Best Fit & Equation \\
\hline
$\alpha$ & $1.240_{-0.073}^{+0.079}$ & \eqnref{eq:er_nr_calibrations_parameter_determ_nr_nex_nion} \\
$\zeta$ & $4.72_{-0.73}^{+0.88} \times 10^{-2} $ & \eqnref{eq:er_nr_calibrations_parameter_determ_nr_nex_nion} \\
$\beta$ & $239_{-8.8}^{+28}$ & \eqnref{eq:er_nr_calibrations_parameter_determ_nr_nex_nion} \\
$\gamma$ & $1.385_{-0.073}^{+0.058} \times 10^{-2}$ & \eqnref{eq:er_nr_calibrations_parameter_determ_nr_recomb_sigma} \\
$\delta$ & $6.20_{-0.64}^{+0.56} \times 10^{-2}$ & \eqnref{eq:er_nr_calibrations_parameter_determ_nr_recomb_sigma} \\
$k$ & $0.1394_{-0.0026}^{+0.0032}$ & \eqnref{eq:er_nr_calibrations_parameter_determ_nr_lindhard} \\
$\eta$ & $3.3_{-0.7}^{+5.3}$ &  \eqnref{eq:er_nr_calibrations_parameter_determ_nr_birks} \\
$\lambda$ & $1.14_{-0.09}^{+0.45}$ & \eqnref{eq:er_nr_calibrations_parameter_determ_nr_birks} \\
\hline
\hline
\end{tabular}
\caption{Best-fit values along with 68\% credible intervals for nuclear recoil microphysics parameters from \citeref{NEST2015}.  These
are used as gaussian priors for \ambe and NG band fitting.}
\label{tab:er_nr_calibrations_parameter_determ_nr_nest}
\end{table}
\egroup

Using this model the expected number of photons and electrons is

\begin{subequations}
\begin{align}
n_{\mathrm{ph}} &= L(E) f_b \frac{E}{W} \Bigg[1 - \frac{1}{1 + n_{\mathrm{ex}}/n_{\mathrm{ion}}} (1 - r) \Bigg] \\
n_{\mathrm{e}} &= L(E) \frac{E}{W} \frac{1}{1 + n_{\mathrm{ex}}/n_{\mathrm{ion}}} (1 - r)
\end{align}
\end{subequations}



\subsection{Detector Effects}
\label{subsec:er_nr_calibrations_parameter_determ_det_phys}
After the microphysics emission processes the properties of our detector become relevant as they convert
$n_{\mathrm{ph}} \rightarrow \mathrm{S1}$ and $n_{\mathrm{e}} \rightarrow \mathrm{S2}$.  This is in part why the
characterizations detailed in \secref{sec:det_char} are essential.  Although the \ce{^{220}Rn}, \ce{^{241}AmBe}, and NG calibrations
were performed at different times they were fit simultaneously.  In many cases the change in detector conditions between SR0 and SR1
affected parameters, so separate values are included.  The simultaneous fit ensures the detector parameters - none of which showed any
time-variation inside either run - have a single posterior, with two exceptions.  The first is the filter boxes were not installed in the
DAQ until after the SR0 \ambe calibration so the efficiency, bias, and smearing are different.  The second is the electron lifetime
$\tau_{e}$ since each calibration is performed with a different LXe purity.  However, to account for the possibility of possible bias in
electron lifetime measurements a systematic uncertainty is included in the fit that is shared across calibrations.

The first effect to consider is position reconstruction (\secref{subsec:det_char_position_reconstruction}).  The reason is that
any position-dependent correction is applied to the position we measure since we cannot know the truth.  The $x \mdash y$ position
reconstruction resolution $((x_{\mathrm{rec}} - x_{\mathrm{true}})^2 + (y_{\mathrm{rec}} - y_{\mathrm{true}})^2)^{1/2}$ is shown in
\figref{fig:calibrations_position_reconstruction_res}.  The poor resolution at
$n_e \lesssim 100\ e^-$ ($\mathrm{S2_b} \lesssim 1000\ \mathrm{PE}$) indicates that in our lowest dark matter energy region events inside
(outside) the FV will have a reasonable chance of being reconstructed outside (inside) - especially near the boundary.  Once
$x_{\mathrm{truth}} \rightarrow x_{\mathrm{rec}}$ and
$y_{\mathrm{truth}} \rightarrow y_{\mathrm{rec}}$ the truth values are no longer considered in the fast MC.


The number of photons from the interaction site that produce one or more photoelectrons is dependent on two effects.  The first is
the position-dependent light collection efficiency (LCE) $\mathcal{C}_{\mathrm{lce}}(x_{\mathrm{rec}}, y_{\mathrm{rec}}, z)$, which
accounts for detector effects such as $<100\%$ PTFE reflectivity and
absorption of light by impurities that reduce the amount of light that reaches the PMTs
(\secref{subsec:det_char_lce}).  The second is the probability of double photoelectron emission (DPE) $p_{\mathrm{dpe}}$, which was found
to be much higher than originally thought, ranging from 0.18-0.24 for the various PMTs tested by \citeref{Faham2015} and $0.225 \pm 0.01$
for the Hamamatsu R11410 model.  The parameter of interest therefore is not the quantum efficiency $\eta_{\mathrm{\mu}}$ but rather the
fraction of incident photons that generate at least one PE $\eta_{\mathrm{p}}$.  As shown in \eqnref{eq:xenon1t_pmts_dpe}
$\eta_{\mathrm{\mu}} / \eta_{\mathrm{p}} = 1 + p_{\mathrm{dpe}}$.  Accounting for both of these effects the number of photons that
generate $\geq 1\ \mathrm{PE}$ is shown in \eqnref{eq:er_nr_calibrations_parameter_determ_det_phys_npe}.

\begin{equation}
n_{\mathrm{p}} = \mathrm{Binom} \bigg( n = n_{\mathrm{ph}},\ p =
\frac{g_1 \mathcal{C}_{\mathrm{lce}}(x_{\mathrm{rec}}, y_{\mathrm{rec}}, z)}{1 + p_{\mathrm{dpe}}} \bigg)
\label{eq:er_nr_calibrations_parameter_determ_det_phys_npe}
\end{equation}

To calculate the total number of photons that produce DPE

\begin{subequations}
\begin{align}
n_{\mathrm{dpe}} &= \mathrm{Binom} (n = n_{\mathrm{p}},\ p = p_{\mathrm{dpe}} ) \\
n_{\mathrm{pe,S1}} &= n_{\mathrm{p}} + n_{\mathrm{dpe}}
\end{align}
\label{eq:er_nr_calibrations_parameter_determ_det_phys_num_pe}
\end{subequations}

\noindent where $n_{\mathrm{pe,S1}}$ is the total number of photoelectrons.

The signals from the PMTs are fed into and recorded by the DAQ (\secref{subsec:xenon1t_daq}), which introduces three effects:
efficiency, bias, and smearing - all of which are the most influential at low photon counts.  The efficiency is the probability that
the signal is found and recognized as an S1 by the
classification algorithm.  If at least 3 PMTs do not observe the S1 it is discarded.  Even if it meets this requirement, the processor
may not identify that there is a signal at low photon counts.  The processor efficiency cannot be modeled using data since it is
not possible to know when a signal was not found.  Instead we use simulated waveforms and pass them through the processor.  The
waveforms are generated with DAQ conditions that have been measured to replicate real waveforms as well as possible.  Separate
processor efficiencies are used for the SR0 \ambe calibration (performed before installation of filter boxes), SR0, and SR1.  Although its
performance is difficult to measure, it is expected to be reasonably accurate and is shown as the red shaded regions in
\figref{fig:er_nr_calibrations_parameter_determ_det_phys_inputs}.  For the fit the relative deviation from the median is assumed to be a
systematic effect so is shared between the three inputs.

\begin{figure}
\centering
\includegraphics[width=\textwidth]{bbf_inputs_sr0_sr1}
\caption{S1- and $\mathrm{S2_b}$-dependent detector parameter 68\% credible intervals for SR0 (left) and SR1 (right)
(right).  Processor efficiency (red) is plotted with respect to the
S1 axis but in units of hits.  Filter boxes were installed on the DAQ between the SR0 \ambe calibration and
dark matter data taking so separate processor efficiencies and S1 and S2 biases and smearings are used (shown as transparent in same
color).  The filter boxes led to improvement, particularly for the processor efficiency and S1 bias and smearing.  The uncertainty on the
processor efficiency also decreased.}
\label{fig:er_nr_calibrations_parameter_determ_det_phys_inputs}
\end{figure}

Bias and smearing (\secref{subsec:det_char_bias_smearing}) are the change in area of the scintillation recorded due to PMT and DAQ
effects.  As with the efficiency, it is impossible to measure directly from data since the true area is not known beforehand.  By using
waveform simulations however, we can input
``truth'' values and compare with the processed value.  Slices in $(\mathrm{S1_{rec}} - \mathrm{S1_{truth}}) / \mathrm{S1_{truth}}$ where
$\mathrm{S1_{rec}}$ and $\mathrm{S1_{truth}}$ are the reconstructed and truth S1s are fit with gaussians, calling the mean
the fractional bias $\mu_{b, \mathrm{S1}}$ and standard deviation the fractional smearing $\sigma_{b, \mathrm{S1}}$ for each bin.  They
are shown as the blue and green shaded regions in
\figref{fig:er_nr_calibrations_parameter_determ_det_phys_inputs}.  As with the processor efficiency the bias and smearing changed after
the filter boxes were installed, and the three inputs share an uncertainty that accounts for systematic effects (bias and smearing have
independently parameters).  For the fit they are constrained to the shaded regions and applied
in the fast MC via \eqnref{eq:er_nr_calibrations_parameter_determ_det_phys_s1_bias_smear}.

\begin{equation}
\mathrm{S1} = n_{\mathrm{pe,S1}} \Big( 1 + \mathrm{Norm} \big( \mu = \mu_{b, \mathrm{S1}},\ \sigma^2 = \sigma_{b, \mathrm{S1}}^2 \big)
\Big)
\label{eq:er_nr_calibrations_parameter_determ_det_phys_s1_bias_smear}
\end{equation}

Because multiple scatters happen so close in time the S1s are typically not resolvable.  Thus in the fast MC all S1s are summed, though
this only affects \ambe and NG data (\betadecay of \ce{^{212}Pb} should not scatter).  The LCE map is applied

\begin{equation}
\mathrm{cS1} = \frac{\mathrm{S1}}{\mathcal{C}_{\mathrm{lce}}(x_{\mathrm{rec}}, y_{\mathrm{rec}}, z)}
\label{eq:er_nr_calibrations_parameter_determ_det_phys_cs1}
\end{equation}

\noindent to obtain the corrected S1.  In the case of multiple scatters the S1 will be corrected by a single
$\mathcal{C}_{\mathrm{lce}}(x_{\mathrm{rec}}, y_{\mathrm{rec}}, z)$ with $x_{\mathrm{rec}}, y_{\mathrm{rec}}, z$ calculated from the S2
(\secref{subsec:det_char_position_reconstruction}), though in reality the various interactions have
likely occurred in different regions of the detector.  The cS1 then is not fully representative of the total light yield.  However,
a single scatter cut is applied that requires the energy of additional depositions to be below a threshold
(\secref{subsec:er_nr_calibrations_parameter_determ_cuts}).  In addition, to be
consistent we apply an identical correction in the fast MC, so (assuming our GEANT4 MC is sufficiently valid) this does not impact our
results.

The \electron from the event move from the event site towards the liquid-gas interface.  As they drift electronegative
impurities will bind to them, reducing the total number that reach the surface and are extracted across the GXe.  The attachment rate
is field-dependent, and for \ce{O_{2}} - expected to be the largest contributor - decreases with larger $E_d$
(\secref{subsec:tpcs_working_principle}).

Inhomogeneities in $E_d$ influence the charge yield in two ways.  The first is the impurity attachment rate will change depending on the
position in the TPC.   This causes the number of \electron that reach the surface from events in different regions to be
biased.  The second is the light and charge yield vary with electric field (shown in
\figref{fig:tpcs_signals_drift_field}).  Therefore events at the same energy may have
different yields depending on where in the TPC they occur.  This also affects the light yield, and therefore cS1.  Neither of these
effects are considered
in the fast MC, introducing a small level of uncertainty.  However, beacause $E_d$ is extremely uniform inside the FV as shown in
\figref{fig:xenon1t_tpc_efield} the error should be small.

The electron lifetime used is randomly selected from the collection of lifetimes measured for each calibration.  Any global
systematic error in measured lifetimes \eqnref{eq:det_char_elifetime} is accounted for with a normal distribution

\begin{equation}
\tau_{e, \mathrm{true}} = \tau_{e, \mathrm{rec}} \Big( 1 + \mathrm{Norm} \big( \mu = \tau_{e, \mathrm{rec}},\ \sigma^2 = \sigma_{e}^2
\big) \Big)
\label{eq:er_nr_calibrations_parameter_determ_det_phys_elife_true}
\end{equation}

\noindent where $\tau_{e, \mathrm{rec}}$ and $\tau_{e, \mathrm{true}}$ refer to the reconstructed (measured) and true electron
lifetimes, and $\sigma_e$ is the uncertainty on $\tau_{e, \mathrm{rec}}$.  We set the systematic uncertainty to 
$\sigma_{e, \mathrm{SR0}} = 0.04$ and $\sigma_{e, \mathrm{SR1}} = 0.5 \sigma_{e, \mathrm{SR0}}$.  The difference is because we are
more confident in the electron lifetimes for SR1 (\figref{fig:det_char_elifetime_evolution}).  The true and reconstructed electron
survival probabilities are given in
\eqnref{eq:er_nr_calibrations_parameter_determ_det_phys_prob_elifetime}. The number of \electron that survive to the surface
$n_{\mathrm{surv}}$ is calculated using \eqnref{eq:er_nr_calibrations_parameter_determ_det_phys_drift_electrons} where $p_{\mathrm{wall}}$
refers to the probability an \electron attaches to the wall; however, because this is only relevant in the $1 \mdash 2\ \mathrm{cm}$
next to the PTFE we can ignore its effects and set $p_{\mathrm{wall}} = 0$.

\begin{subequations}
\begin{align}
p_{\mathrm{rec}} (t_d) &= e^{-t_d / \tau_{e, \mathrm{rec}}} \\
p_{\mathrm{true}} (t_d) &= e^{-t_d / \tau_{e, \mathrm{true}}}
\end{align}
\label{eq:er_nr_calibrations_parameter_determ_det_phys_prob_elifetime}
\end{subequations}

\begin{equation}
n_{\mathrm{surv}} = \mathrm{Binom} \Big( n = n_{\mathrm{e}},\ p = p_{\mathrm{surv}} \Big) ,\ \ \
p_{\mathrm{surv}} = p_{\mathrm{true}} \times (1 - p_{\mathrm{wall}})
\label{eq:er_nr_calibrations_parameter_determ_det_phys_drift_electrons}
\end{equation}

The number of electrons extracted from the liquid $n_{\mathrm{extr}}$ depends only on $n_{\mathrm{surv}}$ and the probability of
extraction $\eta_{EE}$

\begin{equation}
n_{\mathrm{extr}} = \mathrm{Binom} \Big( n = n_{\mathrm{surv}},\ p = \eta_{EE} \Big)
\end{equation}

\noindent where $\eta_{EE} = 0.936$ and $0.933$ for SR0 and SR1 and is fixed in the fast MC (the change in notation from
\secref{subsec:det_char_photon_charge_efficiencies} is to avoid confusion with $\eta$ in the the nuclear recoil model,
\eqnref{eq:er_nr_calibrations_parameter_determ_nr_birks}).  The single electron gain is given by

\begin{equation}
G (x, y) = \frac{g_{2\mathrm{b}} \mathcal{C}_{\mathrm{S2_b}}(x_{\mathrm{rec}}, y_{\mathrm{rec}})}{\eta_{EE}} =
G_e \mathcal{C}_{\mathrm{S2_b}}(x_{\mathrm{rec}}, y_{\mathrm{rec}})
\label{eq:er_nr_calibrations_parameter_determ_det_phys_gg}
\end{equation}

\noindent where $G_e$ is the nominal position-averaged gas gain (\secref{subsec:det_char_single_electron_gain}) and
$\mathcal{C}_{\mathrm{S2_b}}(x, y)$ is the \stwob $x \mdash y$ correction map
(\secref{subsec:det_char_s2_position_correction}).  Approximating the amplification as a gaussian the number of photoelectrons is

\begin{equation}
n_{\mathrm{pe,S2_b}} =  \mathrm{Norm} \Big( \mu = n_{\mathrm{extr}} G,\
\sigma^2 = n_{\mathrm{extr}} \sigma_{G}^2 G^2 \Big)
\label{eq:er_nr_calibrations_parameter_determ_det_phys_s2_num_pe}
\end{equation}

\noindent where $G = G(x, y)$ to simplify notation and $\sigma_G$ is the gas gain resolution and is 0.24 and 0.25 for SR0 and
SR1.  Similar to the S1 there will be bias and smearing.  Using the same assumptions we find

\begin{equation}
\mathrm{S2_b} = n_{\mathrm{pe,S2_b}} \Big( 1 + \mathrm{Norm}(\mu = \mu_{b, \mathrm{S2_b}},\ \sigma^2 = \sigma_{b, \mathrm{S2_b}}^2) \Big)
\label{eq:er_nr_calibrations_parameter_determ_det_phys_s2_bias_smear}
\end{equation}

\noindent where $\mu_{b, \mathrm{S2_b}}$ and $\sigma_{b, \mathrm{S2_b}}$ are the \stwob fractional bias and smearing,
respectively.  They are shown as the pink and yellow bands in \figref{fig:er_nr_calibrations_parameter_determ_det_phys_inputs}, and
deviations in the fit are parameterized by a single variable.  In
practice \eqnref{eq:er_nr_calibrations_parameter_determ_det_phys_gg,
eq:er_nr_calibrations_parameter_determ_det_phys_s2_num_pe, eq:er_nr_calibrations_parameter_determ_det_phys_s2_bias_smear} are applied
for the total S2 - however, in the fast MC the treatment is slightly different.  After
\eqnref{eq:er_nr_calibrations_parameter_determ_det_phys_s2_num_pe} we calculate
$\mathrm{S2} = \mathrm{S2_b} / (1 - f_{\mathrm{aft}})$ where $f_{\mathrm{aft}} = 0.627$ (fixed) is the mean fraction of light from an
S2 that is observed by the top PMT array.  Bias and smearing are then applied.  While in reality the total bias and smearing
can differ from that of the bottom array, to save memory we use $\mu_{b, \mathrm{S2_b}}$ and $\sigma_{b, \mathrm{S2_b}}$.  Any
difference is too small to affect the results so can be disregarded.  $\mu_{b, \mathrm{S2_b}}$ and $\sigma_{b, \mathrm{S2_b}}$ are shown
in the right panel of \figref{fig:er_nr_calibrations_parameter_determ_det_phys_inputs} and as with the S1, are constrained to the
shaded regions for the fit.

Lastly the electron lifetime and S2 position corrections are applied to calculate the corrected S2.

\begin{equation}
\mathrm{cS2_b} = \frac{\mathrm{S2_b}}{\mathcal{C}_{\mathrm{S2_b}} p_{\mathrm{rec}}}
\end{equation}



\subsection{Cuts}
\label{subsec:er_nr_calibrations_parameter_determ_cuts}
Each event in the data must pass a series of cuts.  Because the cuts were developed to select good events - that is, events that are
clearly the result of an interaction in the LXe - most are not necessary for the fast-MC since ``bad'' events are not simulated.  An
example is a cut that requires the fraction of the S1 observed by the top PMT array to fall within a range determined by analysis, in
part to reject events in the GXe.  This is irrelevant to the fast-MC since only events in the LXe are included as input.

There are a couple of cuts that, assuming the data quality is good, select which events to keep.  The first is a requirement that
$\mathrm{S2} > 200\ \mathrm{PE}$, referred to as the S2 threshold cut.  The second cut is introduced to remove events that have more than
one scatter by rejecting those where the second-largest \stwob falls above some $\mathrm{S2_b}$-dependent threshold.  However, it can be
difficult to determine if multiple scatters occurred - particularly at low energy, and for nuclear
recoils, which have a long mean free path and low light and charge yields.  Thus some multi-scatter events are inevitably included,
with the restriction that the ratio of the first to second scatters is large.  This is not a concern for \ce{^{220}Rn}.

For the remainder of the cuts S1 and S2 acceptances are computed using the data to understand what fraction of good events are lost
as a result of each cut.  The product of each cut's acceptance is included in the fast MC for consistency and are used to accept events
with a fraction equivalent to the data.  They are shown as the light blue (S1) and
black (S2) bands in \figref{fig:er_nr_calibrations_parameter_determ_det_phys_inputs}.



\subsection{Backgrounds}
\label{subsec:er_nr_calibrations_parameter_determ_additional_components}
While the majority of events recorded during a calibration are from the source a subset comes
from background.  These are considered in the analysis to avoid fitting events that are not described by the models.

Electronic recoils (\secref{subsec:backgrounds_electronic}) are by far the largest background.  For \ce{^{220}Rn}
they do not present a problem but need to be included in the NR fits.  The ER band computed in parallel is used in the likelihood
analysis for NR calibration fits.

The higher event rate will cause an increase in accidental coincidence (\secref{subsec:backgrounds_ac}).  This is more dramatic
for \ambe and NG data where neutrons with small scattering angles may produce a very small S1 or S2.  AC distributions are included
in all five calibrations, with SR0 and SR1 \ambe sharing one since the location and activity of the source were the same.  AC is not
simulated in the band fitting so the amplitude of the distribution is set by a gaussian-constrained parameter.

Wall events were not included because they are far from the fiducial volume.



\subsection{Monte Carlo Matching}
\label{subsec:er_nr_calibrations_parameter_determ_mc_match}
The fit is performed using a binned likelihood approach using Bayesian inference.  Fast MC events (typically $\mathcal{O}(10^6)$) and
data are each binned in cS1 vs. $\mathrm{log_{10}(cS2_b / cS1)}$.  The likelihood between equivalent bins \li is computed as
\begin{subequations}
\begin{align}
\mathcal{L}_i &=\frac{\hat{b}_{i}^{b_i} e^{-\hat{b}_{i}}}{b_{i}!} \\
\mathcal{L} &= \prod_i \mathcal{L}_i \\
\mathrm{ln}\, \mathcal{L} &= \sum_i \mathrm{ln}\, \mathcal{L}_i = \sum_i b_i\, \mathrm{ln} (\hat{b}_i) - \hat{b}_i - \mathrm{ln} (b_i !)
\end{align}
\end{subequations}

\noindent where \bhi and $b_i$ is are the expected (MC) and true (data) number of events in a bin, respectively.  The parameters
responsible for the fast Monte Carlo ($\hat{b}_i$) were detailed earlier in this section and are listed in
\tabref{tab:er_nr_calibrations_parameter_determ_mc_match}.  As discussed, each parameter is constrained by our understanding before
the fit using a prior distribution.  For the cases where we expect the error to be either negligible or contained in a dependent parameter
the value is fixed.  Components where we have good knowledge on the uncertainty (e.g. $W$, $g_1$, $g_{2\mathrm{b}}$ etc.) are constrained
with a gaussian.  The S1 and S2 cut acceptances as well as processor reconstruction acceptance are modeled by a gaussian distribution with
mean $\mu = 0$ corresponding to the median value and $\sigma = \pm 1$ to the lower and upper bounds in the credible interval.  Parameters
that have a range of preferred values (e.g. $n_{\mathrm{ex}} / n_{\mathrm{ion}}$, $p_{\mathrm{dpe}}$,
bias and smearing, etc.) are restricted to within these bounds.  Finally, in cases where there is little knowledge the priors
are left free - though this is only applied to parameters in the electronic recoil recombination model
(\secref{subsec:er_nr_calibrations_parameter_determ_er}).  Applying more stringent constraints on well-understood parameters provides
more information on those that are less known.

In Bayesian inference a posterior density exists based on Bayes' formula
\begin{equation}
p(\vect{\theta}|\vectlett{x}, \vect{\alpha}) = \frac{p(\vectlett{x}|\vect{\theta})
p(\vect{\theta}|\vect{\alpha})}{p(\vectlett{x}|\vect{\alpha})} = \frac{\mathcal{L}(\vect{\theta})
p(\vect{\theta}|\vect{\alpha})}{p(\vectlett{x}|\vect{\alpha})}
\end{equation}

\noindent where $p(\vect{\theta}|\vect{\alpha})$ captures the prior understanding of the model and is known as the
\textit{prior probability}.  Constraints on $\vect{\theta}$ - generally from physical restrictions or previous measurements or
knowledge - are stored in hyperparameters $\vect{\alpha}$.  $p(\vect{\theta}|\vectlett{x}, \vect{\alpha})$ is the probability of
parameters $\vect{\theta}$ of dimension $n$ given the
data $\vectlett{x}$ and is known as the \textit{posterior probability density}, or \textit{target
density}.  $p(\vectlett{x}|\vect{\theta})$ is the probability of the data $\vectlett{x}$ given the parameters
$\vect{\theta}$ and is
commonly referred to as the \textit{likelihood} $\mathcal{L}(\vect{\theta})$.  $p(\vectlett{x}|\vect{\alpha})$ is as the
\textit{marginal likelihood}
and can be computed using \eqnref{eq:er_nr_calibrations_parameter_determ_mc_match_p_data}.  Because it does not depend on
$\vect{\theta}$ it is model-independent so does not contribute to the relative likelihoods between different
$\vect{\theta}$.
\begin{equation}
p(\vectlett{x}|\vect{\alpha}) = \int_{\Theta} p(\vectlett{x}|\vect{\theta}) p(\vect{\theta}|\vect{\alpha})\, \vectlett{d}^n \vect{\theta}
\label{eq:er_nr_calibrations_parameter_determ_mc_match_p_data}
\end{equation}

Unfortunately for nearly all models - including those of interest to this analysis - $p(\vectlett{x}|\vect{\alpha})$ is not
calculable.  \secref{subsec:er_nr_calibrations_parameter_determ_mcmc} discusses a solution that allows us to estimate the posterior
without needing to solve \eqnref{eq:er_nr_calibrations_parameter_determ_mc_match_p_data}.



\subsection{Markov Chain Monte Carlo}
\label{subsec:er_nr_calibrations_parameter_determ_mcmc}
To estimate the values $\vect{\theta}$ that are best fit by $\vectlett{x}$ a Markov Chain Monte Carlo (MCMC) is used.  MCMCs
are a class of algorithms that sample a probability distribution, and are in part desirable because they
give complete knowledge of the full posterior probability
distribution.  An MCMC consists of $k \geq 1$ ``walkers'' with each walker representing an independent set of parameters
$\vect{\theta}_i$.  A large
subpopulation of MCMC algorithms - including the method used in this analysis - use random walk Monte Carlos algorithms, meaning the
walkers move about randomly.

A MCMC runs for $T$ iterations or ``steps'' so the total number of samples is $k \times T$.    An array of $k$ walkers of
$\vect{\theta}$ over $T$ iterations forms a ``chain'' that contains the history of the walkers.  Each sample contributes its
integrand to the total integral.  In looking for its next move a walker may make a number of trial steps around its perimeter looking for
a point with a high integrand.  A step is dependent solely on the current positions of all $\vect{\theta}$.  Sometimes referred to as
``memorylessness'' this means that with respect to knowing only the walker's current state, knowledge of its entire history would not
improve predictions of its future state, i.e. a walker's past and future states are independent.

The next state in a Markov Chain is computed by

\begin{equation}
f^{(t + 1)} (\theta) = \int f^{(t)} (\theta^{\prime}) p(\theta|\theta^{\prime}) d\theta^{\prime}
\label{eq:er_nr_calibrations_parameter_determ_mcmc_marginal}
\end{equation}

\noindent where $f^{(t)}, f^{(t + 1)}$ are the marginal distributions at steps $t$ and $t + 1$ (to simplify notation $\theta$
is used in place of multi-dimensional $\vect{\theta}$, but the results are similar).  Therefore, beginning from step $t$, any state
at $m$ steps into the future can be computed.  In the case when initial distribution $f^{(0)}$ is given every state in the Markov Chain
can be calculated.

A Markov Chain is \textit{reversible} if there exists a probability density $\pi (\theta^{\prime})$ such that

\begin{equation}
\pi (\theta^{\prime}) p(\theta^{\prime}|\theta) = \pi (\theta) p(\theta|\theta^{\prime})
\label{eq:er_nr_calibrations_parameter_determ_mcmc_detailed_balance}
\end{equation}

\noindent for some transition kernel probability density $p(\theta^{\prime}|\theta)$ that defines the probability of
moving to state $\theta^{\prime}$ from $\theta$ in one step.  In other words, 
\eqnref{eq:er_nr_calibrations_parameter_determ_mcmc_detailed_balance} states that being in state $\theta$ and transitioning to
$\theta^{\prime}$ must have the same probability as being in state $\theta^{\prime}$ and transitioning to  $\theta$.  When
this is true for all pairs of $\theta, \theta^{\prime}$ (all states can communicate with one another) the Markov Chain
is said to be \textit{irreducible}.  A chain that does not cycle to any state $\theta$ in a predictable way is
\textit{aperiodic}.  \eqnref{eq:er_nr_calibrations_parameter_determ_mcmc_detailed_balance}
is known as the \textit{detailed balance condition} and cannot be solved for all Markov Chains.  For those that can a simple integration
gives

\begin{equation}
\begin{aligned}
\int \pi (\theta^{\prime}) p(\theta|\theta^{\prime}) d\theta^{\prime} &=
\int \pi (\theta) p(\theta^{\prime}|\theta) d\theta^{\prime} \\
&= \pi (\theta) \int p(\theta|\theta^{\prime}) d\theta^{\prime} \\
&= \pi (\theta)
\end{aligned}
\label{eq:er_nr_calibrations_parameter_determ_mcmc_stationary}
\end{equation}

\noindent which reveals $\pi (\cdot)$ is a
\textit{stationary distribution}.  \eqnref{eq:er_nr_calibrations_parameter_determ_mcmc_stationary} is a special case of
\eqnref{eq:er_nr_calibrations_parameter_determ_mcmc_marginal} in that $\pi( \cdot )$ is invariant across all iterations (for this reason
it is also referred to as an \textit{invariant distribution}).  When $p(\theta^{\prime}|\theta)$ is irreducible and
aperiodic it will have a single stationary distribution and is \textit{ergodic}, or guaranteed to converge as
$t \rightarrow \infty$ regardless of initial distribution.

\begin{equation}
\lim_{t \rightarrow \infty} f^{(t)}(\theta) \rightarrow \pi (\theta)\ \ \forall f^{(0)}
\eqnref{eq:er_nr_calibrations_parameter_determ_mcmc_converge}
\end{equation}

A frequent objective in developing MCMC algorithms is to create Markov Chains that are reversible, ergodic, homogeneous
(no variation in $p(\vect{\theta}^{\prime}|\vect{\theta})$ in $t$), and has its target distribution as its stationary distribution.

$\vect{\theta}$ may contain parameters that are necessary for the model but of little interest,
known as \textit{nuisance parameters}, $\vect{\theta}^{(N)}$ ($\vect{\theta} = [\vect{\theta}^{(I)}, \vect{\theta}^{(N)}]$ with
$\vect{\theta}^{(I)}$ representing parameters of interest).  It can often be the case that we wish to \textit{marginalize}, or integrate
over $\vect{\theta}^{(N)}$
\begin{equation}
p(\vect{\theta}^{(I)}|\vectlett{x}, \vect{\alpha}) = \int_{\Theta^{(N)}}
p(\vect{\theta}^{(I)}, \vect{\theta}^{(N)}|\vectlett{x}, \vect{\alpha}) \vectlett{d}^{n^{(N)}}\vect{\theta}^{(N)}
\end{equation}

\noindent though this in general may be difficult to compute.  However, sampling from the MCMC posterior joint distribution
$p(\vect{\theta}^{(I)}, \vect{\theta}^{(N)}|\vectlett{x}, \vect{\alpha})$ naturally provides values for $\vect{\theta}^{(I)}$ from the
marginalized posterior $p(\vect{\theta}^{(I)}|\vectlett{x}, \vect{\alpha})$.  This technique can be done in general for any number of
parameters in $\vect{\theta}$.  Often it is interesting to know the posterior of a single parameter, in which case all others would
be marginalized over.  This is an advantage over many other techniques.

MCMCs have become increasingly popular in recent years as advances in methods and computer processing speeds have made them
a powerful tool that can be run in reasonable timescales.  For this analysis an implementation by \citeref{Foreman2013} of
the Affine-Invariant Ensemble Sampler proposed by \citeref{Goodman2010} modified to allow stretch move update step parallelization is
used as outlined below.  A Differential Evolution Markov Chain (DEMC) is used in \secref{subsec:elifetime_fit_mcmc} for the electron
lifetime analysis.

\begin{enumerate}
\item Initialize $k$ walkers of $n$-dimensional parameter space to some state $\vect{\theta}(t = 0)$ (for this analysis samples are
randomly drawn from $p(\vect{\theta}|\vect{\alpha})$).

\item \label{itm:divide} Divide the ensemble into subsets $S^{(0)} = \{\vect{\theta}_i, \forall i = 1, . . ., k/2\}$, and
$S^{(1)} = \{\vect{\theta}_i, \forall i = k/2 + 1, . . ., k\}$.

\item \label{itm:newstate} For each walker in $S^{(0)}$ randomly select a walker $\vect{\theta}_j^{(1)}$ from $S^{(1)}$ and propose a new
state
\begin{equation}
\vect{\theta}_p = \vect{\theta}_j^{(1)} + z \big[ \vect{\theta}_i(t) - \vect{\theta}_j^{(1)} \big]
\label{eq:er_nr_calibrations_parameter_determ_mcmc_walker_update}
\end{equation}

\noindent where $z$ is randomly drawn from distribution $g(z)$. This stretch move is affine-invariant, though some other MCMC
methods are not.  Choosing
$g(z^{-1}) = z g(z)$ keeps \eqnref{eq:er_nr_calibrations_parameter_determ_mcmc_walker_update} symmetric - that is, the proposal
distributions for $\vect{\theta}_i \rightarrow \vect{\theta}_p$ and $\vect{\theta}_p \rightarrow \vect{\theta}_i$ are
equivalent.  Choosing the
acceptance probability of the proposed stretch move update step as
\begin{equation}
q = \mathrm{min} \bigg( 1, z^{n - 1} \frac{\mathcal{L}(\vect{\theta}_p)p(\vect{\theta}_p|\vect{\alpha}_p)}
{\mathcal{L}(\vect{\theta}_i)p(\vect{\theta}_i|\vect{\alpha}_i)} \bigg)
\label{eq:er_nr_calibrations_parameter_determ_mcmc_prob}
\end{equation}

\noindent ensures the chain will satisfy detailed balance.  \citeref{Foreman2013} uses
\begin{equation}
g(z) \propto
\begin{cases}
\dfrac{1}{\sqrt{z}} & \mathrm{if}\ z \in \bigg[ \dfrac{1}{a}, a \bigg], \\
0 & \mathrm{otherwise}
\end{cases}
\end{equation}

\noindent with $a$ as a scalable parameter as recommended by \citeref{Goodman2010}.

\item \label{itm:rand} Generate a random number from a uniform distribution $r \in [0, 1]$.  If $r \leq q$ then accept proposed state
$\vect{\theta}_i(t + 1/2) = \vect{\theta}_p$, otherwise keep present state $\vect{\theta}_i(t + 1/2) = \vect{\theta}_i(t)$.

\item \label{itm:update}  Set $t = t + 1/2$.

\item \label{itm:rerun_second_half} Repeat steps \cref{itm:newstate,itm:rand,itm:update} for $S^{(1)}$ using the updated $S^{(0)}$.

\item Repeat \cref{itm:newstate,itm:rand,itm:update,itm:rerun_second_half} for $T$ iterations.
\end{enumerate}

The advantage of the above implementation is the computationally expensive stretch move update steps
(\cref{itm:newstate,itm:rand,itm:update}) are run
for each walker in parallel, saving enormous amounts of time.  Running all $k$ walkers in parallel would break detailed balance,
but splitting into two groups satisfies it.

Because the Affine-Invariant Ensemble Sampler satisfies \eqnref{eq:er_nr_calibrations_parameter_determ_mcmc_converge} it is guaranteed
to converge as $t \rightarrow \infty$.  However, having a finite number of samples allows us to test for convergence,
though it can never be proved.  An important metric for evaluating the results is the acceptance fraction
$\mathcal{F}$.  This is the fraction of proposed stretch move update steps that are accepted by a walker (step \cref{itm:rand}).  There is
no consensus on an optimal value.  $\mathcal{F} \sim 0$ would mean nearly all steps are rejected so there would be few independent samples
and the target density would be poorly explored.  $\mathcal{F} \sim 1$ means nearly every proposal is accepted in which case
the chain is in a random walk with little regard for the posterior.  A reasonable range is considered to be $0.2 \mdash 0.5$.

Thus $\mathcal{F}$ is
the fraction of proposals that are accepted over the course of the fit.  Even if the posterior of $\vect{\theta}_p$ is less than
that of $\vect{\theta}_i$ the proposal may be accepted.  We do not need to know the marginal
likelihood (\eqnref{eq:er_nr_calibrations_parameter_determ_mc_match_p_data}) to perform an Affine-Invariant Ensemble Sampler MCMC
fit.

A metric to assess convergence is the autocorrelation time, which measures the number
of evaluations to produce independent samples of the target density and is recommended by \citeref{Foreman2013}.  The Affine-Invariant
Ensemble Sampler has been shown to have a smaller autocorrelation time than the popular Metropolis-Hastings algorithm
\citeref{Goodman2010}.  The autocorrelation time is affine-invariant, which makes it a reasonable measurement to quantify
the convergence of samplers with varying levels of density anisotropy.

A second metric and the main one used in
this analysis is the Gelman-Rubin statistic $\hat{R}$ \citeref{Gelman1992}.  It compares the average variance of the individual chains
with the variance between chains, with the idea being the two should be nearly equivalent when the fit has converged.

For this analysis the electronic and nuclear recoil bands are fit with  $n = 44\ \mathrm{parameters}$ and an Affine-Invariant Ensemble
Sampler with $k = 200$ walkers over 11,000 steps.



\section{Results}
\label{sec:er_nr_calibrations_results}
With the parameters and procedure outlined in \secref{sec:er_nr_calibrations_parameter_determ} the results are now
presented.  The posterior of the MCMC fit is defined as the final $1000\ \mathrm{iterations}$ for the 200 walkers.  Marginalized
posteriors in \tabref{tab:er_nr_calibrations_results_er}, \tabref{tab:er_nr_calibrations_parameter_determ_mc_match} and
\tabref{tab:er_nr_calibrations_results_backgrounds} are calculated using these 200,000 samples.  Medians and credible regions shown in
figures are computed using 400 samples randomly drawn from the posterior.  The Gelman-Rubin statistic in this region is
shown in \figref{fig:er_nr_calibrations_results_gr} and is ${\sim}2$.  Ideally it would be closer to 1 but time restrictions did not allow
continued fitting.  Regardless, $\hat{R} \sim 2$ indicates the fit is likely close to the best-fit PDF, though as mentioned above
convergence can never be proved for a finite number of steps.

\begin{figure}
\centering
\includegraphics[width=0.8\textwidth]{gelman_rubin}
\caption{Gelman-Rubin test statistic for the electronic and nuclear recoil band fitting
(\secref{sec:er_nr_calibrations}).}
\label{fig:er_nr_calibrations_results_gr}
\end{figure}



\subsection{Parameter Posteriors}
\label{subsec:er_nr_calibrations_results_par_post}
The microphysics for electronic recoils (\secref{subsec:er_nr_calibrations_parameter_determ_er}) depends on the mean energy
per quanta $W$, Fano Factor $F$, and exciton-to-ion ratio $n_{\mathrm{ex}} / n_{\mathrm{ion}}$.  Recombination is computed using
a the modified Thomas-Imel box model (\eqnref{eq:er_nr_calibrations_parameter_determ_ti}) and recombination fluctuations
(\eqnref{eq:er_nr_calibrations_parameter_determ_er_rec_fluctuations}).

The microphysics for nuclear recoils is described by $\alpha$, $\zeta$, $\beta$, $\gamma$, $\delta$, $k$, $\eta$, and $\lambda$
(\secref{subsec:er_nr_calibrations_parameter_determ_nr}).  It shares only one
parameter, $W$, with the electronic recoil model.  The exciton-to-ion ratio is parameterized differently than ER.  To date
recombination fluctuations (RF) have not been observed in nuclear recoils so they are omitted from the model.

The priors and marginalized posteriors for the electronic and nuclear recoil model microphysics parameters are shown in
\tabref{tab:er_nr_calibrations_results_er}.  Because we
do not use asymmetric gaussian distributions the uncertainties on the NR parameters differ slightly from the values
found by NEST in \tabref{tab:er_nr_calibrations_parameter_determ_nr_nest}, with generally the larger of the two selected
\citeref{NEST2015}.

\bgroup
\def\arraystretch{1.2}
\begin{table}
\centering
\resizebox{\textwidth}{!}{
\begin{tabular}{cccccc}
\hline
\hline
Parameter & Prior & Prior Distribution & Posterior & Source & Comment \\
\hline
$W$ & $13.7 \pm 0.2\ \mathrm{eV}$ & Normal & $13.78_{-0.21}^{+0.21}$ & \ce{^{220}Rn}, \ce{^{241}AmBe}, NG &
\eqnref{eq:er_nr_calibrations_parameter_determ_er_quanta}, \eqnref{eq:er_nr_calibrations_parameter_determ_nr_quanta},
\citeref{Dahl2009} \\
\hline
\multicolumn{6}{c}{Electronic Recoils} \\
\hline
$F$ & 0.059 & Fixed & $-$ & \ce{^{220}Rn} & \citeref{Doke1976}, \eqnref{eq:er_nr_calibrations_parameter_determ_er_quanta} \\
$n_{\mathrm{ex}} / n_{\mathrm{ion}}$ & 0.06-0.2 & Uniform & $0.150_{-0.053}^{+0.034}$ & \ce{^{220}Rn} &
\eqnref{eq:er_nr_calibrations_parameter_determ_er_nions}, see
\secref{subsec:er_nr_calibrations_parameter_determ_nr} for NR \\
$E_0$ & $\geq 0$ & Free & $1.133_{-0.332}^{+0.231}$ & \ce{^{220}Rn} & Recombination, \eqnref{eq:er_nr_calibrations_parameter_determ_ti} \\
$E_1$ & $\geq 0$ & Free & $0.473_{-0.141}^{+0.171}$ & \ce{^{220}Rn} & Recombination, \eqnref{eq:er_nr_calibrations_parameter_determ_ti} \\
$\gamma_{\mathrm{er}}$ & $0 \mdash 0.5$ & Free & $0.125_{-0.028}^{+0.029}$ & \ce{^{220}Rn} & Recombination,
\eqnref{eq:er_nr_calibrations_parameter_determ_ti} \\
$\omega_{\mathrm{er}}$ &  & Free & $30.5_{-3.3}^{+4.1}$ & \ce{^{220}Rn} & Recombination,
\eqnref{eq:er_nr_calibrations_parameter_determ_ti} \\
$\delta_{\mathrm{er}}$ &  & Free & $-0.243_{-0.055}^{+0.056}$ & \ce{^{220}Rn} & Recombination,
\eqnref{eq:er_nr_calibrations_parameter_determ_ti} \\
$A$ & $\geq 0$ & Free & $0.0408^{+0.0058}_{-0.0056}$ & \ce{^{220}Rn} & RF,
\eqnref{eq:er_nr_calibrations_parameter_determ_er_rec_fluctuations} \\
$B$ & $\geq 0$ & Free & $1.735_{-1.148}^{+1.320}$ & \ce{^{220}Rn} & RF,
\eqnref{eq:er_nr_calibrations_parameter_determ_er_rec_fluctuations} \\
\hline
\multicolumn{6}{c}{Nuclear Recoils} \\
\hline
$\alpha$ & $1.240 \pm 0.079$ & Normal & $1.283_{-0.063}^{+0.070}$ & \ce{^{241}AmBe}, NG &
\eqnref{eq:er_nr_calibrations_parameter_determ_nr_nex_nion} \\
$\zeta$ & $0.0472 \pm 0.0088$ & Normal & $0.0451_{-0.0083}^{+0.0085}$ & \ce{^{241}AmBe}, NG &
\eqnref{eq:er_nr_calibrations_parameter_determ_nr_nex_nion} \\
$\beta$ & $239 \pm 28$ & Normal & $271_{-18}^{+22}$ & \ce{^{241}AmBe}, NG & \eqnref{eq:er_nr_calibrations_parameter_determ_nr_nex_nion} \\
$\gamma$ & $0.01385 \pm 0.00073$ & Normal & $0.01415_{-0.00060}^{+0.00059}$ & \ce{^{241}AmBe}, NG &
\eqnref{eq:er_nr_calibrations_parameter_determ_nr_recomb_sigma} \\
$\delta$ & $0.0620 \pm 0.0064$ & Normal & $0.0615_{-0.0064}^{+0.0054}$ & \ce{^{241}AmBe}, NG &
\eqnref{eq:er_nr_calibrations_parameter_determ_nr_recomb_sigma} \\
$k$ & $0.1394 \pm 0.0032$ & Normal & $0.138_{-0.003}^{+0.003}$ & \ce{^{241}AmBe}, NG & \eqnref{eq:er_nr_calibrations_parameter_determ_nr_lindhard} \\
$\eta$ & $3.3 \pm 0.7$ & Normal & $3.283_{-0.525}^{+0.675}$ & \ce{^{241}AmBe}, NG & \eqnref{eq:er_nr_calibrations_parameter_determ_nr_birks} \\
$\lambda$ & $1.14 \pm 0.45$ & Normal & $1.139_{-0.246}^{+0.354}$ & \ce{^{241}AmBe}, NG & \eqnref{eq:er_nr_calibrations_parameter_determ_nr_birks} \\
\hline
\hline
\end{tabular}
}
\caption{Median and 68\% credible intervals for electronic and nuclear recoil microphysics models using the final
1000 steps of the fit.  The NR model is adopted from \citeref{NEST2015}.}
\label{tab:er_nr_calibrations_results_er}
\end{table}
\egroup

Once the microphysics of the interaction has concluded, the photons and \electron are influenced by detector effects
(\secref{subsec:er_nr_calibrations_parameter_determ_det_phys}).  In addition an S2 threshold requirement, processor efficiency, and S1
and S2 cut acceptances remove some events (\secref{subsec:er_nr_calibrations_parameter_determ_cuts}).  Parameters are shared within
science runs and listed in \tabref{tab:er_nr_calibrations_parameter_determ_mc_match}.

\begin{figure}
\centering
\includegraphics[width=\textwidth]{er_sr0_cs1}
\caption{cS1 for SR0 \ce{^{220}Rn} calibration.  The median and 68\% credible region from the posterior are shown as the solid lines and
shaded regions, respectively, and matches the data well.  Aside from \ce{^{220}Rn}, the only contributor to
data is accidental coincidence, which has a small contribution and is visible in pink at the bottom.  The sum of AC and \ce{^{220}Rn} is
shown in blue.}
\label{fig:er_nr_calibrations_results_er_sr0_cs1}
\end{figure}

\begin{figure}
\centering
\includegraphics[width=\textwidth]{er_sr1_cs1}
\caption{cS1 for SR1 \ce{^{220}Rn} calibration.  The median and 68\% credible region match the data well.  AC is shown in pink at the
bottom of the plot.  The sum of \ce{^{220}Rn} and AC are shown in blue.}
\label{fig:er_nr_calibrations_results_er_sr1_cs1}
\end{figure}

\begin{figure}
\centering
\includegraphics[width=\textwidth]{er_sr0_cs2}
\caption{\cstwob for SR0 \ce{^{220}Rn} data with median and 68\% credible interval in slices of cS1.  The model matches the data well,
with the possible exception of the $50 < \mathrm{cS1} < 70\ \mathrm{PE}$ slice where the model may predict slightly lower
$\mathrm{cS2_b}$, but limited statistics make it difficult to tell.}
\label{fig:er_nr_calibrations_results_er_sr0_cs2}
\end{figure}

\begin{figure}
\centering
\includegraphics[width=\textwidth]{er_sr1_cs2}
\caption{\cstwob for SR1 \ce{^{220}Rn} calibration in slices of cS1.  The results from the band fitting match the data well, though in
the $0 < \mathrm{cS1} < 10\ \mathrm{PE}$ slice the data may follow a slightly narrower distribution, and the
$50 < \mathrm{cS1} < 70\ \mathrm{PE}$ slice, which similar to the SR0 results (\figref{fig:er_nr_calibrations_results_er_sr0_cs2})
suggests the model predicts lower $\mathrm{cS2_b}$.}
\label{fig:er_nr_calibrations_results_er_sr1_cs2}
\end{figure}

\bgroup
\def\arraystretch{1.2}
\begin{table}
\centering
\resizebox{\textwidth}{!}{
\begin{tabular}{cccccc}
\hline
\hline
Parameter & Value & Units & Prior Distribution & Posterior & Comment \\
\hline
$\eta_{EE, \mathrm{SR0}}$ & $0.936$ & $-$ & Fixed & $-$ & SR0 extraction efficiency \\
$\eta_{EE, \mathrm{SR1}}$ & $0.933$ & $-$ & Fixed & $-$ & SR1 extraction efficiency \\
$\sigma_{G, \mathrm{SR0}}$ & $0.240$ & $-$ & Fixed & $-$ & SR0 gas gain resolution \\
$\sigma_{G, \mathrm{SR1}}$ & $0.250$ & $-$ & Fixed & $-$ & SR1 gas gain resolution \\
$v_{d, \mathrm{SR0}}$ & $0.144$ & $\mathrm{mm\ \mu s^{-1}}$ & Fixed & $-$ & SR0 drift velocity \\
$v_{d, \mathrm{SR1}}$ & $0.134$ & $\mathrm{mm\ \mu s^{-1}}$ & Fixed & $-$ & SR1 drift velocity \\
$f_{\mathrm{aft}}$ & $0.627$ & $-$ & Fixed & $-$ & S2 fraction by top PMTs\\
$E_{d, \mathrm{SR0}}$ & $120$ & $\mathrm{kV\ cm^{-1}}$ & Fixed & $-$ & \\
$E_{d, \mathrm{SR1}}$ & $82$ & $\mathrm{kV\ cm^{-1}}$ & Fixed & $-$ & \\
$g_1$ & $0.1424 \pm 0.0062$ & PE/ph & Normal & $0.142_{-0.005}^{+0.005}$ & \eqnref{eq:er_nr_calibrations_parameter_determ_det_phys_npe} \\
$g_{2\mathrm{b}}$ & $11.44 \pm 0.20$ & $\mathrm{PE/e^-}$ & Normal & $11.38_{-0.18}^{+0.18}$ &
\eqnref{eq:er_nr_calibrations_parameter_determ_det_phys_gg} \\
$p_{\mathrm{dpe}}$ & $0.18 \mdash 0.24$ & $-$ & Uniform & $0.219_{-0.024}^{+0.015}$ &
\eqnref{eq:er_nr_calibrations_parameter_determ_det_phys_npe}, \eqnref{eq:er_nr_calibrations_parameter_determ_det_phys_num_pe},
\citeref{Faham2015} \\
$\sigma_{e, \mathrm{SR1}}$ & $0 \pm 0.02$ & $-$ & Normal & $0.006_{-0.011}^{+0.012}$ &
$\sigma_{e, \mathrm{SR0}} = 2\sigma_{e, \mathrm{SR1}}$, \eqnref{eq:er_nr_calibrations_parameter_determ_det_phys_elife_true} \\
Multiscatter Cut & $0 \mdash 1$ & $-$ & Uniform & $0.52_{-0.38}^{+0.34}$ & \\
$\mu_{b, \mathrm{S1}}$ & $0 \mdash 1$ & $-$ & Uniform & $0.63_{-0.40}^{+0.25}$ &
\eqnref{eq:er_nr_calibrations_parameter_determ_det_phys_s1_bias_smear} \\
$\sigma_{b, \mathrm{S1}}$ & $0 \mdash 1$ & $-$ & Uniform & $0.324^{+0.394}_{-0.245}$ &
\eqnref{eq:er_nr_calibrations_parameter_determ_det_phys_s1_bias_smear} \\
$\mu_{b, \mathrm{S2}}$ & $0 \mdash 1$ & $-$ & Uniform & $0.45_{-0.29}^{+0.37}$ &
\eqnref{eq:er_nr_calibrations_parameter_determ_det_phys_s2_bias_smear} \\
$\sigma_{b, \mathrm{S2}}$ & $0 \mdash 1$ & $-$ & Uniform & $0.52^{+0.35}_{-0.34}$ &
\eqnref{eq:er_nr_calibrations_parameter_determ_det_phys_s2_bias_smear} \\
S2 Threshold & 200 & PE & Fixed & $-$ & Cut \\
Processor Efficiency & $0 \pm 1$ & $-$ & Normal & $-0.25_{-0.98}^{+0.93}$ & \\
S1 Cut Acceptance & $0 \pm 1$ & $-$ & Normal & $0.08_{-1.17}^{+0.92}$ & \\
S2 Cut Acceptance & $0 \pm 1$ & $-$ & Normal & $-0.1_{-1.0}^{+1.1}$ & \\
\hline
\hline
\end{tabular}
}
\caption{Median and 68\% credible intervals for detector, processor, and analysis effects using the final 1000 iterations of the
fit.  Biases and smearings are constrained between their lower (0) and upper (1) limits.  Processor efficiency and
cut acceptances are constrained by a normal distribution centered at the 50th percentile, with standard deviation ($\pm 1$) representing
the 16\% and 84\% percentiles.  Biases, smearings, cut acceptances, and processor efficiencies are expected to be impacted by systematic
effects so the relative deviation is shared between the three inputs (\ambe SR0, SR0, SR1) for each parameter.}
\label{tab:er_nr_calibrations_parameter_determ_mc_match}
\end{table}
\egroup

\begin{figure}
\centering
\includegraphics[width=0.7\textwidth]{er_sr0_cs1_log_cs2_cs1}
\caption{cS1-$\mathrm{log}_{10}(\mathrm{cS2_b / cS1})$ distribution for the SR0 \ce{^{220}Rn} calibration.  Electronic (red)
and nuclear (blue) recoil (calculated using SR0 \ce{^{241}AmBe} fit) medians and $\pm 2\sigma$ are marked in solid and dashed
lines.  Accidental coincidence is shown in the bottom panel.}
\label{fig:er_nr_calibrations_results_er_sr0_cs1_log_cs2_cs1}
\end{figure}

\begin{figure}
\centering
\includegraphics[width=0.7\textwidth]{er_sr1_cs1_log_cs2_cs1}
\caption{cS1-$\mathrm{log}_{10}(\mathrm{cS2_b / cS1})$ distribution for the SR1 \ce{^{220}Rn} calibration.  Electronic recoil
median and $\pm 2\sigma$ are shown as red solid and dashed lines.  Nuclear recoil are computed using the SR1 \ambe calibration fit and
shown in blue.}
\label{fig:er_nr_calibrations_results_er_sr1_cs1_log_cs2_cs1}
\end{figure}

\begin{figure}
\centering
\includegraphics[width=\textwidth]{ambe_sr0_cs1}
\caption{cS1 for SR0 \ambe calibration.  The total number of events (blue) agrees reasonably well with the data.  Single- (red) and
multiple-scatter (green) neutrons are shown.  Multiple-scatters make up a larger fraction of events
at low cS1.  Electronic recoils (gold) make up a decent fraction of the total events, especially at $\mathrm{cS1} \gtrsim 50\ \mathrm{PE}$
where they are $> 30\%$.  Accidental coincidence is barely visible along the bottom and is expected to be responsible for ${\sim}2$
events.}
\label{fig:er_nr_calibrations_results_ambe_sr0_cs1}
\end{figure}

\begin{figure}
\centering
\includegraphics[width=\textwidth]{ambe_sr1_cs1}
\caption{cS1 for SR1 \ambe calibration.  The total number of events from the band fitting (blue) is compared with data.  Single- (red)
and multiple-scatter (green) nuclear recoils are shown.  The fraction of multiple-scatter events increases at low cS1.  The low-ER
background (gold) and AC (pink) contribute ${\sim}60$ and $\lesssim 2$ events, respectively.}
\label{fig:er_nr_calibrations_results_ambe_sr1_cs1}
\end{figure}

Finally accidental coincidence and electronic recoil backgrounds during calibrations (due to
$r_{\mathrm{rec}} < 36.94\ \mathrm{cm}$ wall leak is negligible if not absent,
\secref{subsec:er_nr_calibrations_parameter_determ_additional_components}) must be considered.  The method
described in \secref{subsec:er_nr_calibrations_parameter_determ_additional_components} tightly constrains the AC.  The AC
distributions and expectations are computed independently for each source.  The ER background - only relevant for \ambe and NG
(for \ce{^{220}Rn} it simply contributes to the ER band) - is left free.  The initial values, ranges, and posteriors are given in
\tabref{tab:er_nr_calibrations_results_backgrounds}.  \figref{fig:er_nr_calibrations_results_ambe_sr1_cs1} shows the AC (pink) and ER
(gold) spectra.

\bgroup
\def\arraystretch{1.2}
\begin{table}
\centering
\begin{tabular}{rcccc}
\hline
\hline
\multicolumn{1}{c}{Parameter} & Value & Prior Distribution & Posterior & Comment \\
\hline
Science Run 0 \ce{^{220}Rn} & $\geq 0$ & Free & $996^{+89}_{-70}$ & \\
AC & $1.59 \pm 0.32$ & Normal & $1.62_{-0.33}^{+0.31}$ & \\
\hline
Science Run 0 \ce{^{241}AmBe} & $\geq 0$ & Free & $1980_{-118}^{+114}$ & \\
AC & $1.6 \pm 0.32$ & Normal & $1.701_{-0.325}^{+0.270}$ & \\
ER fraction & $\geq 0$ & Free & $0.248_{-0.033}^{+0.047}$ & \\
\hline
Science Run 1 \ce{^{220}Rn} & $\geq 0$ & Free & $7528^{+211}_{-278}$ & \\
AC & & & & \\
\hline
Science Run 1 \ce{^{241}AmBe} & $\geq 0$ & Free & $2020^{+131}_{-131}$ & \\
AC & $1.6 \pm 0.32$ & Normal & $1.69_{-0.32}^{+0.36}$ & \\
ER fraction & $\geq 0$ & Free & $0.031_{-0.011}^{+0.014}$ & \\
\hline
Science Run 1 NG & $\geq 0$ & Free & $2058_{-125}^{+137}$ & \\
AC & $7.94 \pm 1.588$ & Normal & $8.6_{-1.7}^{+1.5}$ & \\
ER fraction & $\geq 0$ & Free & $0.0127_{-0.007}^{+0.010}$ & \\
\hline
\hline
\end{tabular}
\caption{Accidental coincidence and electronic recoil background distributions for \ce{^{220}Rn}, \ce{^{241}AmBe}, and NG
calibrations calculated using the posterior of the band fits.  AC is tightly constrained via the method outlined in
\secref{subsec:er_nr_calibrations_parameter_determ_additional_components}.  ER background in \ambe and NG data is left free.}
\label{tab:er_nr_calibrations_results_backgrounds}
\end{table}
\egroup

\begin{figure}
\centering
\includegraphics[width=\textwidth]{ambe_sr0_cs2}
\caption{\cstwob in slices of cS1 for SR0 \ambe data.  The band fitting results (blue) match the data nicely. The ratio of multi- (green)
to single-scatter (red) neutrons is higher at low $\mathrm{cS2_b}$.  The electronic recoil band (gold) is easily visible.  AC (pink)
is too low to observe.}
\label{fig:er_nr_calibrations_results_ambe_sr0_cs2}
\end{figure}

\begin{figure}
\centering
\includegraphics[width=\textwidth]{ambe_sr1_cs2}
\caption{\cstwob for SR1 \ambe calibration data in slices of cS1.  The median and 68\% credible interval of the fit (blue) is compared
to the data.  Single-scatter (red) and multiple-scatter (green) nuclear recoils, ER (gold), and AC (pink) are shown.  The ER background
contributes only ${\sim}3\%$ of all events.  AC is too small to be visible.}
\label{fig:er_nr_calibrations_results_ambe_sr1_cs2}
\end{figure}

\begin{figure}
\centering
\includegraphics[width=0.7\textwidth]{ambe_sr0_cs1_log_cs2_cs1}
\caption{cS1-$\mathrm{log}_{10}(\mathrm{cS2_b / cS1})$ distribution for the SR0 \ambe calibration.  The top panel shows the
total while the rest show the contributions from different components normalized with respect to the total.  Medians and $\pm 2\sigma$
from electronic (red) and nuclear (blue) recoil bands are shown as solid and dashed lines.  The ER lines are taken from SR0 ER band fit.}
\label{fig:er_nr_calibrations_results_ambe_sr0_cs1_log_cs2_cs1}
\end{figure}

\begin{figure}
\centering
\includegraphics[width=0.7\textwidth]{ambe_sr1_cs1_log_cs2_cs1}
\caption{cS1-$\mathrm{log}_{10}(\mathrm{cS2_b / cS1})$ distribution for the SR1 \ambe calibration.  The total is shown
in the top panel with the separate contributors displayed below.  The electronic recoil band median and $\pm 2\sigma$ are derived
from the SR1 ER band from the fit.  Nuclear recoil lines are blue.}
\label{fig:er_nr_calibrations_results_ambe_sr1_cs1_log_cs2_cs1}
\end{figure}

\begin{figure}
\centering
\includegraphics[width=\textwidth]{ng_sr1_cs1}
\caption{cS1 for NG calibration.  The fit agrees with the data.  Single-scatter (red) and multiple-scatter (green) neutrons are shown.  ER
background (gold) is responsible for $\lesssim 25$ events.  Accidental coincidence (pink) is too small to be observed but makes up
7-10 events.}
\label{fig:er_nr_calibrations_results_ng_sr1_cs1}
\end{figure}

\begin{figure}
\centering
\includegraphics[width=\textwidth]{ng_sr1_cs2}
\caption{\cstwob for SR1 NG calibration in slices of cS1.  The fit (blue) matches the data nicely, though in the
$10 < \mathrm{cS1} < 20\ \mathrm{PE}$ slice the model predicts a lower $\mathrm{cS2_b}$.  Similar to \ambe the ratio of multi-
(green) to single-scatter neutrons (red) is larger at low $\mathrm{cS2_b}$.  ER background (gold) and AC (pink) have too few events
to be visible.}
\label{fig:er_nr_calibrations_results_ng_sr1_cs2}
\end{figure}

\begin{figure}
\centering
\includegraphics[width=0.7\textwidth]{ng_sr1_cs1_log_cs2_cs1}
\caption{cS1-$\mathrm{log}_{10}(\mathrm{cS2_b / cS1})$ distribution for the SR1 neutron generator calibration band fitting.  Medians and
$\pm 2\sigma$ for the electronic (red) and nuclear (blue) recoil bands are shown as solid and dashed lines, respectively.  Electronic
recoil lines are taken from the SR1 ER model from the fit.}
\label{fig:er_nr_calibrations_results_ng_sr1_cs1_log_cs2_cs1}
\end{figure}

Another parameter of interest is the influence of neutrons that scatter more than once in the TPC.  As mentioned in
\secref{subsec:er_nr_calibrations_parameter_determ_cuts} a single scatter cut is applied to remove multi-scatter events.  However,
it is difficult in some cases to determine if in fact additional scatters occurred - particularly at low energy.  The fast MC
framework applies the same cut used on the data to replicate this effect as closely as possible.  In extracting the spectra of single and
multiple scatters separately we gain a powerful tool to examine this physics.  The cS1 and \cstwob \ambe and NG figures show the single
and multiple scatters separately in red and green.  Additionally, their contributions to the total nuclear recoil band are shown in
the cS1-$\mathrm{log}_{10}(\mathrm{cS2_b/cS1})$ plots.

The fraction of \ambe data that comes from single scatters is $f_{\mathrm{^{241}AmBe, ss}} = 0.6742_{-0.0081}^{+0.0096}$
(fraction of multiple scatters is $f_{\mathrm{^{241}AmBe, ms}} = 0.3258_{-0.0096}^{+0.0081}$).  For NG data it is
$f_{\mathrm{NG, ss}} = 0.6483_{-0.0079}^{+0.0092}$ ($f_{\mathrm{NG, ms}} = 0.3517_{-0.0092}^{+0.0079}$).  Despite the single scatter cut
roughly one third of multi-scatter events end up in the final data.  From the energy spectra in
\figref{fig:er_nr_calibrations_parameter_determ_nr_ambe_spectrum} the vast majority of events occur at low energy, which is where
we know the cut is least effective.  Because we consider multi-scatter neutrons in the fast-MC this should not affect our final
result.

The fraction of nuclear recoil events that scatter more than once and pass the multiple scatter (and other) cut for the SR1 \ambe
calibration is shown in
\figref{fig:er_nr_calibrations_results_ms_fraction}.  Because S1s cannot be resolved events with multiple scatters will appear to have an
inflated S1.  The resulting decrease in $\mathrm{log}_{10}(\mathrm{cS2_b / cS1_{sum}})$ will become more dramatic for events with a
greater number of scatters.  In addition the positition used for the LCE map correction
(\eqnref{eq:er_nr_calibrations_parameter_determ_det_phys_cs1}) is from the highest-energy event.  This
increases the spread in the cS1 distribution since additional scatters in lower or higher LCE region will be under- or over-corrected,
respectively.  As the highest-energy event increases, the relative effect becomes smaller since the multiple scatter cut prevents large
secondary \stwob and therefore large S1s.

\begin{figure}
\centering
\includegraphics[width=\textwidth]{ms_fraction}
\caption{Fraction of nuclear recoil events in SR1 \ambe calibration that result from multiple scatters.  In the main panel red and
blue lines mark the median and $\pm 2\sigma$ for single and multiple scatters, respectively (at $\mathrm{cS1} > 60\ \mathrm{PE}$ the
number of multiple scatter events is too low for calculation).  Multiple scatters will be pushed to smaller
$\mathrm{log}_{10}(\mathrm{cS2_b / cS1})$.  The effect becomes more dramatic at lower
$\mathrm{log}_{10}(\mathrm{cS2_b / cS1})$ due to a couple effects.  The first is a larger number of scatters will lead to a
smaller $\mathrm{cS2_b / cS1}$.  Additionally the S1s are corrected by the LCE map
according the position of the largest scatter, so S1s in a higher-LCE region
would be over-corrected.  This also leads the magnitude of the dip between SS and MS percentiles extending to higher $\mathrm{cS1}$
at lower $\mathrm{log}_{10}(\mathrm{cS2_b / cS1})$.  At sufficiently high $\mathrm{cS1}$ ($\gtrsim 40\ \mathrm{PE}$) the
\cstwob must be large enough that S1s from other scatters do not have much effect, since a secondary scatter with a large S1 should also
contain a large \stwob and fail the multiple scatter cut.  The top and right panels are integrated over the main panel and show the
multiple scatter fraction with respect to $\mathrm{cS1}$ (top) and $\mathrm{log}_{10}(\mathrm{cS2_b / cS1})$ (right).}
\label{fig:er_nr_calibrations_results_ms_fraction}
\end{figure}



\subsection{Light and Charge Yields}
\label{subsec:er_nr_calibrations_results_ly_qy}
In \secref{subsec:er_nr_calibrations_parameter_determ_er} and \secref{subsec:er_nr_calibrations_parameter_determ_nr} the number
of photons and electrons produced were calculated for electronic and nuclear recoils using models that describe
the microphysics of xenon interactions.  Because dark matter searches rely on energy reconstruction, having dependable models is
crucial.  Values are quoted as light (or photon) yield, $L_y$, and charge (or electron) yield $Q_y$ and are defined as
$n_{\mathrm{ph}}/E$ and $n_{\mathrm{e}} / E$, respectively.  \figref{fig:er_nr_calibrations_results_ly_qy_er} shows the light and
charge yields for electronic recoils for this result along with values from previous experiments.  Because the ER model parameters were
not constrained (aside from physical restrictions) our result is an independent measurement of $L_y$ and $Q_y$.  Our
$82\ \mathrm{V\ cm^{-1}}$ seems to be higher (lower) in $L_y$ ($Q_y$) at $E \gtrsim 2\ \mathrm{keV}$ than
other points, while the $200\ \mathrm{V\ cm^{-1}}$ agree well with all but the PIXeY \ce{^{37}Ar}.  However, we should be careful when
comparing measurements since different electric fields change the $L_y/Q_y$ ratio.

\begin{figure}
\centering
\includegraphics[width=\textwidth]{ER_LYQY_final_logx}
\caption{Photon and charge yields electronic recoils.  This result is shown in blue at $82\ \mathrm{V\ cm^{-1}}$ (SR1) and its
extrapolation to $200\ \mathrm{V\ cm^{-1}}$ is shown in green.  The purple band represents XENON100 tritium calibration
\citeref{Aprile2018a}, red triangles and orange squares represent
LUX \ce{^{127}Xe} \citeref{LUX2017c} and tritium \citeref{LUX2016}, respectively, and grey triangles represent PIXeY \ce{^{37}Ar}
\citeref{PIXeY2017}.  The black line shows the NEST v2.0 beta at $82\ \mathrm{V\ cm^{-1}}$ prediction.  This analysis has $> 10\%$
acceptance for events with energies higher than the dashed blue line (\secref{subsec:dark_matter_results_selection}).}
\label{fig:er_nr_calibrations_results_ly_qy_er}
\end{figure}

Historically nuclear recoil light yield measurements have been quoted as
$\mathcal{L}_{\mathrm{eff}}$, the ratio of scintillation yields of nuclear recoils to that of the \ce{^{57}Co} 122 keV \gammaray at zero
electric field.  This was done in part to account for detector effects so long as a second measurement with \ce{^{57}Co} was
performed.  In the era of ton-scale detectors, 122 keV photons, with an attenuation length of just 3 mm, can probe only the very
outermost region of the TPC.  Therefore the absolute light yield $L_y = \mathcal{L}_{\mathrm{eff}} \times 63\ \mathrm{ph\ keV^{-1}}$ has
become the standard notation.  \figref{fig:er_nr_calibrations_results_ly_qy_nr} shows the light and charge yields for nuclear
recoils.  This result was calculated using expected WIMP data rather than \ambe or NG to avoid effects that would incorrectly modify
our results (e.g. summing S1s from multiple-scatter events that pass our cut or a non-homogeneous energy distribution across the
TPC).  Unlike our electronic recoil result, this is not an independent measurement of $L_y$ and $Q_y$ since our emission model parameters
are constrained by \citeref{NEST2015}.  We
see our $L_y$ agrees well at $\gtrsim 3\ \mathrm{keV}$ but has larger disagreement in $Q_y$.  Strangely, it diverges significantly at
$E > 40\ \mathrm{keV}$ where our acceptance is high (${\sim} 80\%$).  As with \figref{fig:er_nr_calibrations_results_ly_qy_er} we should
exercise caution when contrasting yields at different fields, though the effect for nuclear recoils is known to be less stark.

\begin{figure}
\centering
\includegraphics[width=\textwidth]{NR_LYQY_final_logx}
\caption{Photon and charge yields for nuclear recoils.  This result at $82\ \mathrm{V\ cm^{-1}}$ is shown in blue.  It is not
an independent measurement since the model is constrained by previous work.  $L_y$ measurements are represented by purple triangles
\citeref{Aprile2005}, grey triangles \citeref{Aprile2009b}, light green triangles \citeref{Plante2011}, and magenta triangles
\citeref{Manzur2010}.  $Q_y$ are represented by purple diamonds \citeref{Aprile2006b}, red band \citeref{Aprile2013}, black squares
\citeref{Sorenson2009} and light blue triangles \citeref{Manzur2010}.  $L_y$ and $Q_y$ from the same measurement are shown by orange
squares
\citeref{LUX2016b}.  The NEST v1.0 model was used to constrain our fit and is shown as the green band \citeref{NEST2015}.  This
analysis has $> 10\%$ acceptance for events with energies higher than indicated by the dashed blue line
(\secref{subsec:dark_matter_results_selection}).}
\label{fig:er_nr_calibrations_results_ly_qy_nr}
\end{figure}



\subsection{Acceptances}
\label{subsec:dark_matter_results_selection}
Events during dark matter data-taking that are below the $-2 \sigma$ quantile of the ER $\mathrm{cS1} \mdash \mathrm{cS2_b}$ band are
removed before analysis.  Known as ``blinding'' this prevents bias that might occur by adjusting analyses to better fit or reject
these events.  Data selection includes a valid S1 and S2 pair with $\geq 3$ PMTs observing the S1 in $< 100\ \mathrm{ns}$.  A position
reconstruction cut requires the difference between the neural network and likelihood-fit algorithms to be less than 2 (larger S2s) to 5
cm (smaller).

\begin{figure}
\centering
\includegraphics[width=\textwidth]{acceptances_bbf_inputs}
\caption{Acceptances calculated using 400 samples randomly drawn from the posterior of the MCMC band fit.  Best-fit values are shown for
S1 (pink) and S2 (light blue) cut acceptances, S2 threshold (green), processor efficiency (red), and cS1 threshold
(dark blue).  Dashed lines represent medians from SR0 while solid lines and shaded regions show SR1 medians and 68\% credible regions.}
\label{fig:dark_matter_results_selection_acc_components}
\end{figure}

The acceptances are calculated using 400 randomly-drawn samples from the posterior from the electronic and nuclear recoil band
fitting using an expected WIMP spectrum.  While the 1.3 t fiducial mass is used in this analysis, the
acceptances are calculated using the 1 t mass from the band fitting with differences expected to be small.  Each acceptance is shown in
\figref{fig:dark_matter_results_selection_acc_components} for SR0 and SR1.  The S1 and S2 cut
acceptances (\secref{subsec:er_nr_calibrations_parameter_determ_cuts}) are shown in pink and light blue.  They show little variation
across energy.  The S2 threshold acceptance
is only relevant at low energy ($\lesssim 5\ \mathrm{keV_{nr}}$) and is highlighted in green.  The processor efficiency is marked in
red.

\begin{figure}
\centering
\includegraphics[width=\textwidth]{xenon_1ty_paper_acceptance_linear_linear}
\caption{Nuclear recoil acceptances for WIMPs from this analysis.  Processor efficiency is shown in green, processor efficiency with S1
and S2 cuts is shown in blue, total acceptance across all \cstwob in black, and total acceptance between the $-2\sigma$ and NR
median.  Dashed and solid acceptance lines correspond to SR0 and SR1 acceptances,
respectively.  The shaded regions are the 68\% credible interval for SR1.  Spectra are shown for for 10 (dashed), 50 (dotted), and
200 (dash-dotted) $\mathrm{GeV/c^2}$ WIMPs.}
\label{fig:dark_matter_results_acceptances_band_fitting}
\end{figure}

The energy ROI is selected by requiring cS1 threshold $3 \leq \mathrm{cS1} \leq 70\ \mathrm{PE}$ (blue in
\figref{fig:dark_matter_results_selection_acc_components}), corresponding to roughly
$[1.4, 10.6]\ \mathrm{keV_{ee}}$ and $[4.9, 40.9]\ \mathrm{keV_{nr}}$.  The total acceptance is the product of the acceptances shown
in \figref{fig:dark_matter_results_selection_acc_components} - processor, S2 threshold, cS1 threshold, and
cut acceptances.  Events that are gained or lost at the fiducial volume edge but such effects are
expected to be small so are not considered.

Acceptances are calculated using the posterior from the calibration band fitting using an expected
WIMP spectrum.  While we use a 1 t fiducial mass for the band fitting rather than the 1.3 t run-combined
result (\figref{fig:calibrations_position_reconstruction}) we expect deviations to be
small.  \figref{fig:dark_matter_results_acceptances_band_fitting} is redrawn from \secref{subsec:dark_matter_results_selection} and
shows acceptances for 1) processor (``Detection''), 2) processor and S1
and S2 cuts (``Selection''), 3) total for SR0 and SR1 across all $\mathrm{cS2_b}$, and 4) same as (3) but in reference region.

The acceptances are used in the likelihood calculation in \secref{subsec:dark_matter_results_background}.