%Dissertation Template for Columbia University Ph.D. programs
%By Charles McNamara, 2016
%I posted this document under a CC0 public domain license -- do whatever you want with it!
%It's probably a good idea to review the university guidelines just so you know what you want your dissertation to look like. You can read about those guidelines at this site: http://gsas.columbia.edu/content/formatting-guidelines.
%Good luck writing your dissertation!




% This is the main document file for the dissertation. You should not include any of your actual chapters or other substantive writing in this file.
%First we have to set up the style and formatting of the pages.

\documentclass[letterpaper,12pt]{memoir} %The memoir class is great for longer works that use separate chapters. The Dissertation Office recommends 10-pt Arial or 12-pt Times New Roman. I use 12-pt for readability.

%Below are some LaTeX packages to include to make sure that your Unicode characters render correctly. This is especially important if your dissertation includes polytonic Greek!

%\RequireXeTeX %XeTeX allows you to use Unicode characters like polytonic Greek in your writing.
%\usepackage{fontspec} %Allows you to load fonts in XeTeX.
%\defaultfontfeatures{Mapping=tex-text} %Allows you to get pretty TeX ligatures in your writing.
%\usepackage{xunicode} %You need this for Unicode fonts to work properly.
%\usepackage{xltxtra} %Some extra font capabilities for XeTeX
\DisemulatePackage{setspace} %You need to use this package for "true MS Word" double-spacing.
\usepackage{setspace} %Allows you to set different spacing (double, etc.) throughout your writing.
% \usepackage{hyperref} % Use if you want hyperlinks in your table of contents



%\setmainfont{Times New Roman}
%\setmainfont{Linux Libertine O} %This is a really readable serif font. It renders polytonic Greek better than any other OpenType font I've found, including Times New Roman. I now use it for everything, including class handouts. Highly recommended for classicists.

\usepackage{enumitem} % To fix spacing of bullet point lists
\setlist{noitemsep} % or \setlist{nosep} for space around whole list

%\pagestyle{fancy}
\usepackage[utf8]{inputenc}
\usepackage{amsmath}
\usepackage{amsfonts}
\usepackage{amssymb}
\usepackage[ocgcolorlinks, linktocpage]{hyperref}
\hypersetup{
    linkcolor =[RGB]{0,130,200},
    citecolor =[RGB]{60,180,75}
}
\usepackage{graphicx}
\usepackage{geometry}
\newsavebox{\largestimage}
\usepackage{floatrow}
\usepackage{caption}
\usepackage{subcaption}
%\usepackage{subfig}
\usepackage{sidecap}
\usepackage{bm}
%\usepackage{subcaption}
\usepackage{siunitx}
%\usepackage[sort&compress,numbers]{natbib}
\usepackage{multirow}
\newcommand{\RNum}[1]{\uppercase\expandafter{\romannumeral #1\relax}}

\usepackage{cleveref}
\crefrangeformat{enumi}{#3#1#4--#5#2#6}
\crefformat{footnote}{#2\footnotemark[#1]#3}

\usepackage{float}
\usepackage{amsbsy}
\usepackage{mhchem}
\usepackage{xfrac}
\usepackage{booktabs}

\DeclareUnicodeCharacter{00A0}{ }
\setsecnumdepth{subsubsection}

\newcommand\figref[1]{Fig.\,\ref{#1}}
\newcommand\eqnref[1]{Eq.\,\ref{#1}}
\newcommand\tabref[1]{Tab.\,\ref{#1}}
\newcommand\secref[1]{Sec.\,\ref{#1}}
\newcommand\appref[1]{App.\,\ref{#1}}
\newcommand\citeref[1]{\cite{#1}}
\newcommand\chapref[1]{Chap.\,\cite{#1}}
\newcommand{\radon}{\ce{^{222}Rn} }
\newcommand{\uranium}{\ce{^{238}U}}
\newcommand{\thorium}{\ce{^{232}Th}}
\newcommand{\krypton}{\ce{^{85}Kr} }
\newcommand{\radioxenon}{\ce{^{136}Xe}}
\newcommand{\doublebeta}{2$\nu \beta \beta$}
\newcommand{\dru}{$\textrm{events} / (\textrm{kg} \cdot \textrm{day} \cdot \textrm{keV})$}
\newcommand{\csbottom}{$\textrm{cS2}_{\textrm{b}}$}
\newcommand{\csquared}{$\textrm{c}^2$}
\renewcommand*{\cftappendixname}{Appendix\space}
\newcommand{\titde}{\ce{TiD_2}}
\newcommand{\cesium}{\ce{^{137}Cs} }
\newcommand{\sodium}{\ce{^{22}Na} }
\newcommand{\cobaltsixty}{\ce{^{60}Co} }
\newcommand{\nimbar}{$\overline{\textrm{NIM}}$}
\newcommand{\mr}{\mathrm}
\newcommand{\kevee}{keV$_{\mathrm{ee}}$ }
\newcommand{\kevnr}{keV$_{\mathrm{nr}}$ }
\newcommand{\electron}{$e^{-}$ }
\newcommand{\positron}{$e^{+}$ }
\newcommand{\gammaray}{$\gamma$-ray }
\newcommand{\gammarays}{$\gamma$-rays }
\newcommand{\vd}{$v_{d}$ }
\newcommand{\efield}{$E$-field }
\newcommand{\efields}{$E$-fields }
\newcommand{\otwo}{O$_{2}$ }
\newcommand{\htwoo}{H$_2$O }
\newcommand{\vece}{$\mathbf{E}$ }
\newcommand{\nex}{$n_{\mathrm{ex}}$ }
\newcommand{\nion}{$n_{\mathrm{ion}}$ }
\newcommand{\nquant}{$n_{\mathrm{q}}$ }
\newcommand{\energyer}{$E_{\mathrm{ER}}$ }
\newcommand{\energynr}{$E_{\mathrm{NR}}$ }
\newcommand{\nphot}{$N_{\mathrm{ph}}$ }
\newcommand{\nelect}{$N_{\mathrm{e}}$ }
\newcommand{\metakr}{$\mathrm{^{83m}Kr}$}
\newcommand{\radoncal}{\ce{^{220}Rn} }
\newcommand{\betadecay}{$\beta$-decay }
\newcommand{\betadecays}{$\beta$-decays }
\newcommand{\leadtwofourteen}{\ce{^{214}Pb} }
\newcommand{\bismuthtwofourteen}{\ce{^{214}Bi} }
\newcommand{\poloniumtwofourteen}{\ce{^{214}Po} }
\newcommand{\alphadecay}{$\alpha$-decay }
\newcommand{\alphadecays}{$\alpha$-decays }
\newcommand{\poloniumtwoeighteen}{\ce{^{218}Po} }
\newcommand{\utwothirtyfive}{\ce{^{235}U} }
\newcommand{\stwob}{$\mathrm{S2_b}$}
\newcommand{\ustwob}{$\mathrm{uS2_b}$}
\newcommand{\cstwob}{$\mathrm{cS2_b}$ }
\newcommand{\ambe}{\ce{^{241}AmBe} }
\newcommand{\ed}{$E_{\mathrm{d}}$ }
\newcommand{\li}{$\mathcal{L}_i$ }
\newcommand{\bhi}{$\hat{b}_i$ }
\newcommand{\vectlett}[1]{\mathbf{#1}}
\newcommand{\vect}[1]{\boldsymbol{#1}}
\renewcommand{\meta}[2]{$\mathrm{^{{#1}}{#2}}$}
\newcomand{\pc}{$P_{\mathrm{C}}$}
\newcommand{\il}{$I_{\mathrm{L}}$}
\newcommand{\ig}{$I_{\mathrm{G}}$}
\newcommand{\fg}{$F_{\mathrm{G}}$}
\newcommand{\rncal}{\ce{^{220}Rn}}
\newcommand{\rnbkg}{\ce{^{222}Rn}}

\mathchardef\mdash="2D

\newcommand{\aap}{A\&A}
\newcommand{\aj}{Astro. J.}
\newcommand{\amstat}{The American Statistician}
\newcommand{\ap}{Astropart. Phys.}
\newcommand{\apj}{ApJ}
\newcommand{\apjl}{ApJ}
\newcommand{\app}{Astropart. Phys.}
\newcommand{\araa}{ARA\&A}
\newcommand{\arxiv}{arXiv e-prints}
\newcommand{\asr}{Adv. in Space Research}
\newcommand{\camcs}{Comm. App. Math. Comp. Sci.}
\newcommand{\cjc}{Canadian J. Chem.}
\newcommand{\cp}{Chem. Phys.}
\newcommand{\cpl}{Chem. Phys. Lett.}
\newcommand{\epja}{Eur. Phys. J. A}
\newcommand{\epjc}{Eur. Phys. J. C}
\newcommand{\hts}{High Temp. Sci.}
\newcommand{\ieeetns}{IEEE Trans. Nucl. Sci.}
\newcommand{\ieeenssmicr}{Nuc. Sci. Symp. and Med. Imag. Conf. Record}
\newcommand{\ijmpa}{Int. J. of Mod. Phys. A}
\newcommand{\jap}{J. of Applied Physics}
\newcommand{\jcap}{JCAP}
\newcommand{\jcp}{J. of Chem. Phys.}
\newcommand{\jetp}{JETP}
\newcommand{\jhep}{J. of High Energy Phys.}
\newcommand{\jinst}{JINST}
\newcommand{\jjap}{Jpn. J. Appl. Phys.}
\newcommand{\josa}{J. Opt. Soc. Am.}
\newcommand{\jpc}{J. of Phys. Chem.}
\newcommand{\jpcm}{J. of Phys. Condensed Matter}
\newcommand{\jpdap}{J. Phys. D}
\newcommand{\jpgnp}{J. Phys. G}
\newcommand{\mfmdvs}{Mat. Fys. Medd. Dan. Vid. Selsk}
\newcommand{\mnras}{MNRAS}
\newcommand{\na}{Nature Astronomy}
\newcommand{\nat}{Nature}
\newcommand{\nim}{Nuc. Instr. and Method}
\newcommand{\nimpra}{Nuc. Instr. and Meth. in Phys. Research A}
\newcommand{\pasp}{Pub. of the Astro. Soc. of the Pacific}
\newcommand{\pdu}{Phys. of Dark Universe}
\newcommand{\physpro}{Physics Procedia}
\newcommand{\physrep}{Phys. Rept.}
\newcommand{\plb}{Phys. Lett. B}
\newcommand{\pqdt}{ProQuest Dissertations and Theses}
\newcommand{\pr}{Phys. Rev.}
\newcommand{\pra}{Phys. Rev. A}
\newcommand{\prb}{Phys. Rev. B}
\newcommand{\prc}{Phys. Rev. C}
\newcommand{\prd}{Phys. Rev. D}
\newcommand{\prl}{Phys. Rev. Lett.}
\newcommand{\ppsa}{Proc. Phys. Soc. A}
\newcommand{\rmp}{Rev. of Mod. Phys.}
\newcommand{\rnaas}{Research Notes of the American Astronomical Society}
\newcommand{\rpc}{Rad. Phys. and Chem.}
\newcommand{\rsi}{Rev. of Sci. Instr.}
\newcommand{\statsci}{Statistical Science}
\newcommand{\universe}{Universe}


% look at (22; 31; 49; 71; 142; 169; 203) in Aprile2009 for info on purification methods

%Here is some stuff on the bibliography. You want to keep your bibliography file in the same directory as this file.


%\usepackage[american]{babel} %Enables hyphenation and date formats according to American conventions. Change "american" to "british" (or another value) if you work outside the US.
%\usepackage[backend=bibtex8,style=authoryear,texencoding=utf8,bibencoding=utf8]{biblatex} %The command here uses biblatex to render your bibliography, and it tells biblatex to use Unicode fonts. You can change the style of your citations easily by changing the value for "style" -- I use authoryear-icomp which takes care of the id/ibid citations automatically. There are many styles available if you want to change it.
%I have tried to use biber instead of biblatex many times, but it's never worked properly for me. I use biblatex, but feel free to try biber instead. 

\usepackage[backend=bibtex,style=ieee]{biblatex} 
\addbibresource{./bibliography.bib} %Your bibliography file. I use JabRef to keep track of my bibliography. Highly recommended, and free! You can use Zotero if you want, but I've had trouble exporting to .bib files from my Zotero database.
\setlength{\bibitemsep}{\baselineskip} %Skip lines between bibliography entries. Columbia requires that you skip a line between entries.



%Here you can set margins and other page formatting

\setlrmarginsandblock{3.175cm}{3.175cm}{*} %Left and right margin -- the dissertation office requires at least 1-inch margins
\setulmarginsandblock{2.54cm}{2.54cm}{*}  %Upper and lower margin -- same thing, at least 1-inch margins
\checkandfixthelayout %A function of the memoir class that finds the right number of lines per page and apparently tidies up the formatting in other mysterious ways...?

% penalize footnotes that run onto next page
\interfootnotelinepenalty=10000

% get rid of whitespace
\raggedbottom


% Below we start to set up the document itself, including how to use spacing throughout the dissertation.

\begin{document}

\sloppy %If I don't include sloppy, then Greek and Latin words screw up margins all over. If you don't include weird languages in your dissertation, you can probably leave this one out.
\chapterstyle{default} % Nice formatting for chapter headings. Check out the documentation for the memoir class for other options.
\footnotesep\baselineskip % Footnotes need to have a space between each one for Columbia's Dissertation Office. 
\setstretch{1.5} %Set line spacing to "true MS Word" double-spacing
%\setstretch{1.}
%\DoubleSpace % Use LaTeX double-spacing (more like 1.666 spacing)

\expandafter\def\expandafter\quote\expandafter{\quote\SingleSpace} %Keep all block quotes single-spaced regardless of body text spacing.
%\pagestyle{plain} %Put page numbers at the bottom-center for the whole dissertation. Columbia's dissertation office requires that the numbers appear at this location on the page throughout the document.
%\makeevenhead{plain}{\sffamily\leftmark}{}{\sffamily\rightmark}
%\makeoddhead{plain}{\sffamily\rightmark}{}{\sffamily\leftmark}

\makepagestyle{cu}
\copypagestyle{cu}{ruled}
	\makeevenfoot{cu}{}{\thepage}{}
	\makeoddfoot{cu}{}{\thepage}{}



%% Here's the Title Page
%\include{./FrontMatter/TitlePage}

%% Here's the copyright page. I use a Creative Commons license.
%\include{./FrontMatter/Copyright}

%% Here's the Abstract
%% This is the abstract of my dissertation.

\pagestyle{empty} % No page number in entire abstract
\begin{center}
  ABSTRACT

The XENON1T Spin-Independent WIMP Dark Matter Search Results and a Model to Characterize the Reduction of Electronegative Impurities in
its 3.2 Tonne Liquid Xenon Detector

Zachary Schuyler Greene
\end{center}

Over much of the last century evidence has been building for a new component of our universe that interacts primarily through
gravitation.  Known as cold dark matter, this non-luminous source is predicted to constitute 83\% of matter and 26\% of mass-energy in the
universe.  Experiments are currently searching for dark matter via its possible creation in particle colliders, annihilation in
high-density regions of the universe, and interactions with Standard Model particles.  So far dark matter has eluded detection so its
composition and properties remain a mystery.

Weakly interacting massive particles (WIMPs) are hypothetical elementary particles that interact on the scale of the weak nuclear
force.  They naturally satisfy predictions from extensions of the Standard Model, and are one of the most favored dark matter
candidates.  A number of direct detection experiments dedicated to measuring their predicted interactions with atomic nuclei have been
constructed over the last 25 years.

Liquid xenon dual phase time projection chambers (TPCs) have led the field for spin-independent WIMP searches at WIMP masses of
${>}\, 10\ \mathrm{GeV/c^2}$ for most of the last decade.  XENON1T is the first tonne-scale TPC, and with 278.8 days of dark matter data
has set the strictest limits on WIMP-nucleon interaction cross sections above WIMP masses of $6\ \mathrm{GeV/c^2}$, with a minimum of
$4.1 \times 10^{-47}\ \mathrm{cm^2}$ at $30\ \mathrm{GeV/c^2}$.  XENON1T and the analysis that led to this
result are discussed, with an emphasis on electronic and nuclear recoil calibration fits, which help discriminate between background and
WIMP-like events.

Interactions in liquid xenon produce light and charge that are measured in TPCs.  These signals are attenuated by electronegative
impurities including \ce{O_2} and \ce{H_2O}, which are homogeneously distributed throughout the liquid xenon.  The decrease in observables
enlarges the uncertainty in our analysis, and can decrease our sensitivity.  Methods on measuring the
charge loss are presented, and a physics model that describes the behavior of the electronegative impurity concentration over the
lifetime of XENON1T is derived.  The model is shown to successfully explain the more than two years of data.

\maxtocdepth{subsubsection}

%\frontmatter
%\tableofcontents % You need to put your ToC after frontmatter so that it will get lowercase Roman numerals.
\cleardoublepage

%A list of graphs and illustrations should go here if you use any.
%\listoffigures
\cleardoublepage

% Acknowledgements
%% This is the acknowledgments page of my

\cleartorecto % A memoir-class command for moving the acknowledgments to a recto page, not verso.
\chapter{Acknowledgements} % For the heading on the page, also registers in the table of contents
\thispagestyle{plain} %This page should have numbers.

First and foremost, I would like to thank my advisor, Elena Aprile, who welcomed me into her research group in 2014.  I owe this dissertation
and my remarkable research experience over the past four years to your guidance and encouragement.  You exemplify scientific excellence and
innovation, and are a role model for young academics.

I would like to give a special thanks to postdoctoral fellows Qing Lin, Patrick de Perio, and Fei Gao.  You committed your time and energy to
help me succeed.  So much of what I have learned over the years is thanks to you.  I feel fortunate to have had you as mentors.

I would also like to thank postdoctoral fellows and research scientists Guillaume Plante, Marc Weber, Luke Goetzke, Marcello Messina, and
Alfio Rizzo.  Your patience and encouragement, especially at the beginning, shaped my four years in the group.

I have had a lot of fun working alongside graduate students Matt Anthony, Yun Zhang, Joey Howlett, and Tianyu Zhu.  The experiments,
analyses, and trips I've shared with you were among the most memorable of my time at Columbia.  I am also happy to have had the
opportunity to know Damon Daw, who worked in our group for a couple of years.

It has been a pleasure knowing Marcela Stern.  You are one of the kindest and most caring people I know, and am grateful for your support
on so many occasions.  It was always a joy talking with you.

I feel extremely fortunate to have gotten to know Jeremy Dodd these last six years.  You are both a mentor and friend, and were there for me
in times of uncertainty.  I am confident that without you I would not be here today.  You had a tremendous impact on my time at Columbia, and
I am forever grateful.

I am especially thankful to my friends and fellow PhD students Felix Clark, Matt Anthony, Laura Havener, Ryne Carbone, Geoff Iwata, and
Russell Smith.  I feel privileged to have made such remarkable, long-lasting friends during this experience.  Your support during
difficult times - both personally and professionally - has been indispensable.  I want to thank Rex Brown, who became a good friend
during his time in New York.  I also want to thank Felix Clark, Rex Brown, and Michael Hammett for being great roommates.

Sarah Goldberg is one of the most remarkable people in my life.  Silly or serious, happy or sad, dumb or intellectual,
being myself around you is so natural.  Your positivity, uniqueness, and kind-heartedness has given our friendship its affluence.

My family has supported and believed in me my entire life.  My parents, Pam and Brock, give me the
confidence and love that motivates me.  My brother, Spencer, is the bravest and most inspiring person I know.  You all have taught me how to
confront adversity, the power of hard work, and importance of joie de vivre.  I am lucky to have such strong relationships with my aunt,
Diana, uncle, Mark, and cousins, Marissa and Tyler, who encourage, love, support me.  I believe people are largely shaped by those closest to
them, so I share this achievement with all of you.  I love you.


% Dedication
%\include{./FrontMatter/Dedication}

% Preface if you have one




%What follows is the main text of your dissertation. You can comment out lines if you want to exclude them from your document for drafts. Everything after \mainmatter will get Arabic numbers centered on the bottom of the page.

%\mainmatter

%I use subdirectories for each part of my dissertation just to keep the files tidy. LaTeX generates a lot of different files for output, and using subdirectories allows you to find your .tex files more easily.

%%This is the first chapter of the dissertation

%The following command starts your chapter. If you want different titles used in your ToC and at the top of
%the page throughout the chapter, you can specify those values here. Since Columbia doesn't want extra
%information in the headers and footers, the "Top of Page Title" value won't actually appear.

%By using the asterisk to start a new section, I keep the section from appearing in the table of contents.
%If you want your sections to be numbered and to appear in the table of contents, remove the asterisk.



\pagestyle{cu}
\graphicspath{{./Chapter1/Figures/}}
\chapter[Dark Matter][Dark Matter]{Dark Matter}

For much of the last century evidence of two non-luminous components of our universe has been developing

For much of the last century evidence has been building for two non-luminous components of our universe that are not yet understood.
Dark Energy has been derived to explain the accelerating expansion of the universe and is predicted to compose $\sim70$\% of
matter throughout the universe.  Dark Matter is used to satisfy the need for an additional massive component and is expected to
constitute $\sim26$ \% of the universe.  The term "dark" is a relic from their early history and today both are understood to be
invisible.



%====================================
\section[$\Lambda$ CDM Model][$\Lambda$ CDM Model]{$\boldsymbol{\Lambda}$CDM Model}
The cosmological model describes the evolution of the universe since its inception of the Big Bang.  It validity is dependent
on the premise that our universe is homogeneous and isotropic over large ($\sim 100 Mpc$) - consistent with observations,
Einstein's General Relativity, and a path connected universe.  This can be solved for the Robertson-Walker metric, which in
polar coordinates gives

\begin{equation}
ds^{2} = -c^{2}dt^{2} + a(t)^{2} \bigg( \frac{dr^{2}}{1 - kr^{2}} + r^{2}d\Omega^{2} \bigg)
\end{equation}

\noindent where ds is the distance traversed in space-time, $a(t)$ is the scale factor ($a=1$ today), and $k$ is the curvature of the
universe and can be -1, 0, or 1 for open, flat, or closed, respectively.  These values are also known as hyper-spherical,
Euclidean, or spherical.

Solving Einstein's equations with this metric gives the Friedman Equations

\begin{equation}
\Big(\frac{\dot{a}}{a}^{2}\Big) = \frac{8\pi G \rho}{3} - \frac{k}{a^{2}}
\label{eq:friedman1}
\end{equation}

\begin{equation}
\Big(\frac{\ddot{a}}{a}\Big) = -\frac{4\pi G \rho}{3}(\rho + 3p)
\label{eq:friedman2}
\end{equation}

\noindent where $\rho$ is the average energy density of the universe, $\frac{\dot{a}}{a}$ is defined as the Hubble
constant H ($H_{0}$ today), and $p$ is the pressure from components.  From \eqref{eq:friedman1} the critical
density, defined as the density for a flat universe, is $\rho_{crit} = \frac{3H^{2}}{8\pi G}$.  A useful
notation then is the ratio of the density to the critical density, $\Omega = \frac{rho}{rho_crit}$
$\Omega = 1$ is a flat universe.  Substituting
$\Omega$ into \eqref{eq:friedman1} gives the density parameter

\begin{equation}
\Omega - 1 = \frac{k}{H^{2}{a^{2}}}
\end{equation}

\noindent  $\Omega = \sum \Omega_{i}$ where $\Omega_{i}$ is a component of the universe.  Current measurements give density
parameters of $\Omega_{b} ~ 0.05$ for baryonic matter, $\Omega_{dm} = 0.26$ for non-baryonic matter, $\Omega_{r} \sim 0$
for radiation, and $\Omega_{\Lambda} \sim 0.68$ for dark energy (see \ref{subsec:cmb}).  Because cold dark matter
and dark energy dominate $\Omega$ this is referred to as the $\Lambda CDM$ model.



%====================================
\section[Evidence for Dark Matter][Evidence for Dark Matter]{Evidence for Dark Matter}
%========
\subsection{Dynamical Contraints} \label{subsec:dynamics}
The first evidence for Dark Matter came in 1933 when Swiss astronomer Fritz Zwicky was observing the Coma Cluster
and noticed the velocity of the galaxies was larger than the total luminous matter.  He argued that if there were
some additional mass that only acted gravitationally this could explain his observations, and estimated this
to be $\sim 400$ times greater than the luminous matter (\citeref{Zwicky1933}).  Today this factor is much
smaller as the observed luminous matter is larger, but the discrepancy persists.  Zwicky coined this
\textit{dunkel Materie} or ``dark matter".  Since then other galaxy clusters have been observed and supported
this claim.

In 1970 Vera Rubin and Kent Ford used $H\alpha$ from H II regions to measure the rotation of stars around the
center of the Andromeda galaxy \citeref{Rubin1970}.  The virial theorem predicts $v(r) = \sqrt{GM(r)/r}$ so
that at large radii the velocity of the galaxies should decrease as $r^{-1/2}$.  Rubin and Ford's observations
contradicted the value $M(r)$ derived from luminous mass.  As with Zwicky's measurement, the results could be
explained by introducing a non-luminous feature - in this case one with $M(r) \sim r$.  Observations of over
one hundred thousand galaxies have since shown similar results, with DM halos containing several times the
luminous mass.

One example is NGC 6501.  The dark halo mass density in \figref{fig:ngc_6501} is estimated as
$\rho (r) = \rho_{0} \Big[ 1 + \Big( \frac{r}{r_{c}} \Big)^{2} \Big]^{-1}$ where $\rho_{0}$ and $r_{c}$
are the central halo mass density and radius, respectively, by fitting a least square to the velocity
distribution \citeref{Begeman1991}.  The left panel shows the mass of luminous matter decreases after
peaking at $\sim2$ kpc, but the DM increases steadily and at $>10$ kpc is proportional to $r$ causing
the rotational velocity to flatten.  Unfortunately, there is no independent model for determining the
masses of a galaxy's disk, bulge, and halo, which makes disassociating the mass of baryonic matter from
DM very difficult.  Most models assume the luminous matter to compose the galaxy's disk and bulge with
the DM constituting its halo \citeref{Sofu2001}.



\begin{figure}[t]
\centering
\captionsetup[subfigure]{labelformat=empty}
\ffigbox[\linewidth]{%
  \begin{subfloatrow}
\ffigbox[\FBwidth][]{\caption{}}{\includegraphics[height=5cm]{ngc_6503}}
  \end{subfloatrow}
  \hspace*{\columnsep}
  \begin{subfloatrow}
\vbox to 5.3cm{
  \ffigbox[\FBwidth]{\caption{}}
  {\includegraphics[trim={2cm, 2cm, 2cm, 2cm}, clip, height=4.4cm]{ngc_6503_image}}
}
  \end{subfloatrow}
  }{
  \vspace*{-10mm}
  \caption{(left) Rotational velocity vs. radius for NGC 6503.  Visible components represented by dashed line,
  gas by dotted, and dark halo by dashed-dotted.
Image credit: (left) \citeref{Begeman1991} (right) Robert Gendler, the Subaru Telescope (NAOJ) and the Hubble Legacy Archive}
	\label{fig:ngc_6501}
  }
\end{figure}[t]

%for figure trim info
%https://tex.stackexchange.com/questions/57418/crop-an-inserted-image





\begin{figure}
	\centering
	\includegraphics[width=0.5\textwidth]{coma_cluster}
	\label{fig:coma_cluster}
	\caption{Coma Cluster as seen by Sloan Digital Sky Survey and Spitzer Space Telescope.  Image credit: NASA/JPL-Caltech/GSFC/SDSS.}
\end{figure}


%========
\subsection{Big Bang Nucleosynthesis}



%========
\subsection{Gravitational Lensing}
A gravitational lens is a distribution of matter capable of bending electromagnetic radiation between a luminous source and
an observer.  The deflection is caused by the gravitational distortion of space-time by the mass of the lens, which
while in theory can be anything with mass-energy, the effects are typically most noticeable for high-density objects
such as galaxies, galaxy clusters, or a stars.  A source that is gravitationally lensed will have two distinct
features. The convergence, $\kappa$, describes the focusing of the light rays.  The shear, denoted by $\gamma$ and $\phi$, which
represent the ellipticity and position ange, characterizes the distortion of the source.  One important feature of lensing
is a magnification of the source given by

\begin{equation}
\mu = \frac{1}{(1-\kappa)^{2} + \gamma^{2}}
\end{equation}

\noindent that produces an increase in brightness when the source and lens are close to one another.  This has given astronomers
a tool to see objects that previously were thought to be too faint to observe.

In the case of strong lensing an observer will see a misshapen source in the form of arcs.  Because the magnitude of the
deflection is depending on the proximity to the lens, it is possible to see multiple instances of the same source, with
independent distances traveled by each.  Thus a time-delay between images is also present, and their location to the observer
is incorrect.  In the special case
when the lens is directly between the source and observer an Einstein ring will be produced such that the source
appears as a circular distortion of the source around the object, and no time delay will occur.  \figref{fig:lensing}
shows a an image of a strong lens and a diagram depicting the path of the light.  Originally Einstein
thought gravitational lensing to be useless having only considered what today is known as micro lensing (deflection
about a star), but Fritz Zwicky promptly predicted galaxies could provide stronger lensing as well as
magnification.  When the arcs are sufficiently large in size and flux the source's luminosity can be determined.

\begin{figure}
 \centering
 \begin{subfigure}[t]{0.5\textwidth}
  \centering
  \includegraphics[trim={0cm, 2cm, 0cm, 0cm}, clip, height=4.5cm]{lensing_horseshoe}
 \end{subfigure}%
 \begin{subfigure}[t]{0.5\textwidth}
  \centering
  \includegraphics[trim={6cm, 0cm, 0cm, 0cm}, clip, height=4.5cm]{lensing_diagram}
 \end{subfigure}
 \caption{(left) an Einstein ``horseshoe" ring of a red galaxy distorting a blue galaxy behind it.  (right)
 A diagram of strong lensing.  The red source is bent by the blue lens, with the solid orange lines that connect
 to it representing the light's true trajectory.  The observer incorrectly views the source's
 position as indicated by the dashed orange lines, and the image behind them.  Image credit: (left) ESA/Hubble
 \& NASA (right) ALMA (NRAO/ESO/NAOJ)/Luis Calada (ESO)}
 \label{fig:lensing}
\end{figure}


In many astrophysical instances the lens' deflection of the source is much more subtle.  This is known as weak
lensing, and because of its minor effects is much more difficult to observe.  Astronomers must only consider a source's
shear because its luminosity is too low to be understood, and since the deviation in shear is small, systematic effects of
observing (i.e. atmosphere, instrument point spread function, noise, etc.) must be very small and well
understood \citeref{Paolis2016}.  Whereas strong lensing results from radiation passing around the lens, weak lensing
is the passage of light through a gravitational field where tidal effects distort the shape of the image.  Furthermore,
because galaxies are generally elliptical, deduction of the shear can be difficult to impossible.  Thus the gravitational
field is calculated statistically by randomly sampling galaxies from the known ellipticity distribution.  Since the
number of samples must equal the number of galaxies, this method is most effective when the number of galaxies is
large.

If the mass of the lens is well known, the luminous portion can be subtracted, yielding the fraction of DM.  This can typically
be done for strong lenses as shown in \figref{fig:lensing} but is more difficult for weak.

A galaxy collision provides a unique setting to study dark matter.  1E0657-56 (Bullet Cluster) is two galaxies that passed
through one another $\sim150$ million years ago at $4500_{-800}^{+1100}\ \mathrm{km\ s^{-1}}$ \citeref{Markevitch2004}.  \figref{bullet_cluster} shows the x-ray map in pink and the mass
distribution from gravitational lensing in blue in the left panel, along with mass contours in the right.  The separation of
mass from baryonic matter can be explained by incorporating dark matter.  As the galaxies collide intergalactic dust interacts
and heats up, creating x-rays and slowing the speed at which they pass.  The dark matter does not interact and passes
through affected by only gravity.  In addition to providing evidence of dark matter, 1E0657-56 sets a limit on the
cross-section of dark matter self-interaction of $<1\ \mathrm{cm^{2}\ g^{-1}}$ \citeref{Markevitch2004}.  It also provides evidence
against modified gravity, an alternative hypothesis to dark matter, since the observed mass distributions should now lay
outside of the luminous content.



\begin{figure}[t!]
    \centering
    \begin{subfigure}[t]{0.45\textwidth}
        \centering
        \includegraphics[height=4.5cm]{chandra_bullett_preview}
    \end{subfigure}%
    \begin{subfigure}[t]{0.45\textwidth}
        \centering
        \includegraphics[height=4.5cm]{bullet_cluster_paper}
    \end{subfigure}
    \caption{X-ray emission from hot gas (pink) and mass centroids (blue) from gravitational lensing after cluster
	collision of 1E 0657-558.  The white bar in the right panel represents 200 kpc at cluster.  The separation
	between the colors provides evidence for Dark Matter.
	Image credit: (left) X-ray: NASA/CXC/CfA/M.Markevitch et al.; Optical: NASA/STScI; Magellan/U.Arizona/D.Clowe et al.;
	Lensing Map: NASA/STScI; ESO WFI; Magellan/U.Arizona/D.Clowe et al. (right) NASA, ESA, CXC, M. Brada\u{c}
	(University of California, Santa Barbara), and S. Allen (Stanford University)}
	\label{fig:bullet_cluster}
\end{figure}




%========
\subsection{Cosmic Microwave Background} \label{subsec:cmb}
The Cosmic Microwave Background (CMB) was accidentally discovered by Penzias and Wilson, for which they received
the Nobel Prize (\citeref{Penzias1965}).  It
is a remnant from shortly after the Big Bang ($\sim380,000\ \mathrm{yrs}$, $z\sim1100$, $T\sim3000\ \mr{K}$) and a near-perfect
blackbody at $2.725\ \mr{K}$ today (most precise measurement at $2.72548 \pm 0.00057\ \mr{K}$ \citeref{Fixsen2009}).  A CMB
photon is considered to have originated at the time of last scattering $t_{ls}$ and proceeded unperturbed since.

Deviations in the blackbody are small (root-mean square of $\delta T/T \sim 10^{-5}$) and the results of several
mechanisms at $t_{ls}$ that varied throughout space.  The dipole anisotropy results from
our motion with respect to the comoving rest frame of the CMB (\citeref{Smoot1991}).  Furthermore
energy density perturbations would cause fluctuations in gravitational potential $\delta \Phi$.  A
photon at larger $\delta \Phi$ at $t_{ls}$ would become blueshifted as its
potential decreases, while one at lower $\delta \Phi$ would be redshifted.  This effect is known as the
Sachs-Wolfe effect (\citeref{Sachs1967}).  Additional effects, including intrinsic fluctuations and acoustic
oscillations, are not discussed.

\begin{figure}
	\centering
	\includegraphics[width=0.8\textwidth]{PlanckFig_map_columbi1_IDL_HighDR_colbar_1000px_CMB_moll}
	\label{fig:planck_map}
\end{figure}


The CMB provides the most precise measurements on $\Omega_{b}$, $\Omega_{dm}$, $\Omega_{\Lambda}$, $\Omega_{r}$,
$H_{0}$, and many other cosmological parameters.  It has been precisely charted by several satellites since its
discovery, with the most recent being Planck (\citeref{Plack2011}).  The CMB is shown in \figref{fig:planck_map},
where fluctuations are on the order of $\sim 10^{\pm 3} \mu K$.  These fluctuations were caused by the number and
amount of each component - thus, by finding the correlation function

\begin{equation}
%C(\theta) = \Big \langle \frac{\delta T}{T}( \hat{n}) \frac{\delta T}{T} ( \hat{n}^\prime) \Big \rangle
C(\theta) = \Big \langle \delta T( \hat{n}) \delta T(\hat{n}^\prime) \Big \rangle
\end{equation}

\noindent we can identify the correction cosmological makeup.  To do this we make use of

\begin{equation}
\delta T = \sum\limits_{l=0}^{\infty} \sum\limits_{m=-l}^{l} a_{lm}Y_{lm}(\theta, \phi)
\end{equation}

\noindent where $\delta T = T(\theta, \phi) - \langle T \rangle$ for a given $\theta , \phi$
on the map, $Y_{lm}(\theta, \phi)$ corresponds to spherical harmonics, and $a_{lm}$ the preceding coefficients
with the condition that $\sum\limits_{l,m} |a_{lm}|^{2} = 1$.  This gives a correlation function of

\begin{equation}
C(\theta) = \frac{1}{4 \pi} \sum\limits_{l=0}^{\infty} (2l + 1) C_{l} P_{l}(cos \theta)
\end{equation}

\noindent where $P_{l}$ are the Legendre polynomials and $C_{l} = \frac{1}{2l + 1} \sum\limits_{m=-l}^{l} a_{lm}$.  The
only unknown is $C_{l}$, which is typically shown in an angular power spectrum with
$D_{l}^{TT} \equiv l(l+1)C_{l}/2\pi$ such as in \figref{fig:cmb_power_spectrum}.

\begin{figure}
%	\centering
	\includegraphics[width=0.8\textwidth]{cmb_power_spectrum}
	\centering
	\caption{Angular ower spectrum for CMB from Planck measurements, fit using the $\Lambda$CDM model.  Residuals
	are shown in bottom panel.  Image credit: \citeref{Planck2016}}
	\label{fig:cmb_power_spectrum}
\end{figure}

The first peak in the spectrum is related to the curvature of the universe, while the ratio of the heights of the
first and the second peaks provides details about the amount of baryonic matter.  The best fit gives
$H_{0} = 67.81 \pm 0.92\ \mathrm{km\ s^{-1}\ Mpc^{-1}}$, $\Omega_{\Lambda} = 0.692 \pm 0.012$,
$\Omega_{b} = 0.0484 \pm 0.0005$, $\Omega_{dm} = 0.258 \pm 0.004$, and
$\Omega_{k} \equiv 1 - \Omega_{\Lambda} - \Omega_{b} - \Omega_{dm} =  -0.005_{-0.017}^{0.016}$.


%[5] Adams, F.C., Freese, K. & Guth, A.H. [1991], Phys. Rev. D43, 965.



%$\Omega_{m} = 0.308 \pm 0.012$
%$H_{0} = 67.81 \pm 0.92$
%$\Omega_{\Lambda} = 0.692 \pm 0.012$
%$\Omega_{b}h^{2} = 0.02226 \pm 0.00023$
%$\Omega_{c}h^{2} = 0.1186 \pm 0.0020$
%$N_{eff} = 3.046$
%$\Omega_{k} \equiv 1 - \Omega_{m} - \Omega_{\Lambda} = -0.005_{0.017}^{0.16}$ for $\Lambda CDM$

%========
%\subsection{Structure Formation} \label{subsec:structure}

%\endcsname

%====================================
%\section[Dark Matter Candidates][Dark Matter Candidates]{Dark Matter Candidates}
%\label{sec:dmcandidates}
Despite a strong claim for the existence of dark matter, there is no evidence for what its composition may
be.  A candidate for dark matter must have a lifetime much larger than the age of the universe, be electrically
neutral, and have a small matter-dark matter cross section.  Of course there may be more than one particle
that classifies as dark matter, but the sum of them should satisfy our observations of the universe.

%========
\subsection{Axions} \label{subsec:axions}
Axions were originally hypothesized by R.D. Peccei and Helen R. Quinn in 1977 as a solution to the strong CP
(charge and parity) problem (\citeref{Peccei1977}).  Quantum chromodynamics (QCD) predicts there should be
CP violation in strong interactions.

CP (charge and parity) violation in strong interactions has never been observed, despite its prediction
by quantum chromodynamics (QCD).  This forces a theoretically unjustified fine tuning of the model, which
is known as the strong CP problem.  Originally hypothesized by R.D. Peccei and Helen R. Quinn in 1977, the
axion - a new standard model particle - offered a solution (\citeref{Peccei1977}).  Shortly after it was
demonstrated that for axion decay constant $f_{a} > 10^{12}$ axions would be overproduced in the
early universe and cause the axion density $\Omega_{a} > 1 > \Omega_{dm}$ \citeref{Preskill1983}.  However,
a decay constant of $\sim 10^{12}$ could satisfy $\Omega_{dm}$.  Because the axion mass $m_{a}$ and $f_{a}$
are inversely proportional, one can then set a limit on $m_{a}$.
%The axion
%mass $m_{a} = 57(10^{11}GeV/f_{a})\ \mu$eV, which gives a lower bound of $m_{a} \sim 5\ \mu$eV.

Because axions naturally offer and explanation for dark matter there are a number of experiments dedicated to
finding them.  Cavity searches such as ADMX use a resonant microwave cavity inside a superconducting magnet
to convert axions in microwaves.  Others, like CASPEr apply NMR.


%========
\subsection{WIMPs} \label{subsec:wimps}
WIMPs (Weakly Interacting Massive Particles) are another favored candidate for dark matter.  As their name
suggests, they interact through the weak force and thereby would be difficult to observe.  They
are not constrained to the standard model, though would behave similarly
to neutrinos, which have a small cross-section and rarely interact with nuclei.  An additional requirement
is they must be produced early in the universe to account for observations of the CMB and galactic
structures.

At the beginning of the universe the temperature was hot enough where particles could annihilate with their
antiparticle counterpart and produce new particles, maintaining equilibrium.  As the universe cooled each
particle had a ``freeze-out", when they could no longer transfer freely to other particles.  Using the
$\Lambda$CDM model the density of DM in the universe today is given by

\begin{equation}
\Omega_{dm}h^{2} = \frac{3 \dot 10^{-27}\ \mathrm{cm^{3}\ s^{-1}}}{\langle \sigma_{\mathrm{ann}} v \rangle}
\end{equation}

\noindent where $h$ is the Hubble Constant divided by 100 and $\langle \sigma_{\mathrm{ann}} v \rangle$ is
the thermally averaged self-annihilation cross section
for dark matter.  Assuming DM has a cross-section and mass on the order of the weak force, such a
particle would give roughly the correct relic density of DM.  This is known as the ``WIMP miracle".

Another appealing argument for WIMPs is super-symmetry (SUSY), which is theorized to solve some problems
with the standard model,
predict WIMP-like particles of similar masses.  This has historically been one of the favored arguments
for dark matter.

 %========
\subsection{Cold, Warm, or Hot} \label{subsec:hot_vs_cold}
An important property of dark is whether or not it was relativistic in the early universe.  Hot dark matter (HDM)
is defined as being relativistic at the time it decoupled from other components, $t_{mathrm{dec}}$, and at
matter-radiation equality, $t_{\mathrm{rm}}$.  Warm dark matter (WDM) would have been relativistic at $t_{\mathrm{dec}}$
but not at $t_{\mathrm{rm}}$.  Cold dark matter (CDM) would have been non-relativistic at both.  Candidates for CDM include
WIMPs, axions, and primordial black holes while WDM might be the gravitino.  Neutrinos are possible candidates for both
WDM and HDM.

Understanding which DM universe we live in can come from looking at structure in the universe.  Because the structure
today came from fluctuations in the earliest moments of the Big Bang (\secref{subsec:cmb}) the cosmological layout
can illuminate the answer.

In HDM the relativistic particles are able to smooth out density perturbations, which is known as free
streaming.  In this case the first structures to form would be superclusters, followed by smaller-scale
features.  Observations show that this is not the case; galaxies have been around since before the universe
was 1 billion years old ($z \sim 6$) and superclusters are just forming today (\citeref{Ryden2003}).

CDM allows the early density perturbations to persist, causing smaller structures to materialize before
larger, consistent with galaxy surveys between $\sim 1$ Mpc to the horizon.  At scales $< 1$ Mpc and
$M \sim \mathrm{M_{\odot}}$ there are discrepancies, including under-dense cores for many galaxies that are DM-dominated
and significantly fewer satellite dwarf and small galaxies than predicted (\citeref{Moore1999}, \citeref{Klypin1999}).  Possible
solutions to the latter may be that dwarf galaxies have not accumulated enough baryonic matter to be visible
(\citeref{Simon2007}), merged, or been stripped by tidal forces of larger galaxies.

WDM has received a lot of interest since the CDM problems were observed.  Simulations have shown that
WDM would result in fewer subhalos, though other model-observation contradictions have been less
successful (\citeref{Bullock2017}, \citeref{Ogiya2017}).  However, since neutrinos have mass demonstrate there
was at least some non-CDM in the early universe.

Despite the problems with CDM, it remains the most favorable model for dark matter.  One possible outcome is
there is a mix of CDM and WDM, but if that's the case CDM would make up the considerable bulk of dark matter.


%====================================
\section[WIMP Detection Methods][WIMP Detection Methods]{WIMP Detection Methods}
\label{sec:detection}

 %========
\subsection{Colliders} \label{subsec:colliders}
One possible mechanism through which we might observe WIMP dark matter is through particle-antiparticle
annihilation.  This could be observed at particle colliders where energies can exceed
several TeV, thereby producing DM particle-antiparticle pairs, which
would escape undetected.  From momentum conservation there would be missing transverse energy (MET),
which would be carried in the DM.  The Large Hadron Collider (LHC) is investigating the quark sector with energies
exceeding 10 TeV, while the Large Electron-Positron (LEP) Collider is doing so for leptons at
$\sim 200$ GeV.  Both experiments are competitive at lower energies than noble gas experiments, particularly
for spin-dependent searches (\citeref{Fox2011}, \citeref{Alpigiani2017}).

Annihilation alone would not prove the discovery of dark matter, and would need
indirect or direct experiments to validate the results with their own detectors.  But
it would still be useful in narrowing the search region.


 %========
\subsection{Indirect Detection} \label{subsec:indirect}
Indirect detection looks for signatures of DM by observing standard model particles.  Such observables
may come from dark matter annihilation, wherein two DM particles annihilate and produce standard model
gamma rays or other particle-antiparticle pairs.  Alternatively if DM is unstable it may decay into
standard model particles that can be detected.

Indirect experiments look towards regions where they expect a large number of interactions.  One insight is
there could be relic DM trapped in the Sun, which would cause an observable flux of high energy
neutrinos (\citeref{Ellis1988}).  Other theories suspect DM annihilation in the galactic halo would produce
antiprotons, positrons, and gamma rays that would be detectable on Earth.  Because the the galactic center
has a large flux of cosmic rays it is difficult to distinguish dark matter from other astrophysical
sources.  Nearby ($\sim 50$ kpc) dwarf spherical
galaxies have become an attractive target where star formation regions have an expected low $\gamma$-ray
background (\citeref{Zitzer2016}).

Measurements of $\gamma$-rays would have to be from space because for the necessary energy range (GeV to TeV) photons interact
with matter via $e^{+}e^{-}$ pair production, so would not be able to pass through Earth's atmosphere.  However,
they can look for signatures such as showers of secondary particles and their Cerenkov
light as they pass through the atmosphere (\citeref{Bertone2005}).

% for Ellis1988 reference listed above get their references 6 and 7 and maybe 8 for citations, need access to PRL


 %========
\subsection{Direct Detection} \label{subsec:direct}
Direct detection looks for low energy ($\sim 1-100$ keV) nuclear recoil (a few theories predict DM-lepton
but they will not be discussed) (\citeref{Kopp2009}).  Given that the majority of dark matter must be
non-relativistic (\secref{subsec:hot_vs_cold}), we can calculate the differential recoil spectrum as \citeref{Undagoitia2016}

\begin{equation}
\frac{dR}{dE}(E, t) = \frac{\rho_{0}}{m_{\chi}m_{\mathrm{A}}} \int_{v_{\mathrm{min}}}^{v_{\mathrm{esc}}}
v f(\boldsymbol{v}, t) \frac{d\sigma_{\chi}}{dE}(E, t)\ d^{3}v
\end{equation}

\noindent where $\rho_{0}$ is the local dark matter density, $m_{\chi}$ is the mass of a
dark matter particle, $m_{\mathrm{A}}$ is the mass of the traget element, $v_{\mathrm{min}}$ is the minium velocity to produce
a recoil of energy E, $v_{\mathrm{esc}}$ is the escape velocity for WIMPs from the galaxy,
$f(\boldsymbol{v}, t)$ is the local velocity dispersion, and $\frac{d\sigma_{\chi}}{dE}(E, t)$ is the DM differential
cross-section.  Assuming the standard halo model (SHM) as discussed in \secref{subsec:dynamics} we can treat the WIMP
velocity to follow the Maxwell-Boltzmann distribution.



The
density of dark matter to be 0.2-0.56 GeV/cm$^{3}$ (\citeref{Read2014}) 


Because current theory
predicts the DM distribution to be in a halo around the galaxy (\secref{subsec:dynamics}), DM particles should
be passing through Earth and - if they're WIMPS - interact with the nuclei of standard model atoms.



The
energy deposited in such a collision would be manifested as scintillation, excitation (nucleus or electron),
ionization, or phonons.

 %====
\subsubsection{Bubble Chambers} \label{subsubsec:bubbles}
Superheated liquid detectors, also known as bubble chambers, consist of a liquid held in equilibrium by temperature and
pressure just below boiling point.  Typical
mediums are refrigerants such as CF$_{3}$I, C$_{2}$ClF$_{5}$, or C$_{4}$F$_{10}$.  The temperature
and pressure are maintained such that the refrigerant is just below boiling point, and can be set such that
the detector is insensitive to electronic interactions and has high rejection of alphas
(\citeref{Amole2015}).  If Nucleations in the liquid are recorded acoustically or visually, 
will transfer some
of its heat, vaporizing a small volume into gas, which is recorded acoustically or visually.  
a particle interacts with the liquid the heat from its interaction is sufficient to vaporize a small region, thereby creating
a bubble.

When superheated liquid detectors were first considered for WIMP detectors there were some crucial points
that needed to be addressed.  The downtime of such a detector was high because the temperature
For WIMP searches modifications had to be made from the original
devices including increased stability for near-continuous operation and operation as a counting experiment
(\citeref{Pullia2014}).

% read https://www.hindawi.com/journals/ahep/2014/387493/ to rewrite above section
%%This is the second chapter of the dissertation

%The following command starts your chapter. If you want different titles used in your ToC and at the top of the page throughout the chapter, you can specify those values here. Since Columbia doesn't want extra information in the headers and footers, the "Top of Page Title" value won't actually appear.

\chapter[Liquid Xenon and Time Projection Chambers][Liquid Xenon Detection]{Liquid Xenon Detection}
\label{chap:liquid_xe}




%====================================
\section{Genearl Properties}
\label{sec:properties}



%====================================
\section{Primary Scintillation}
\label{sec:primary_scintillation}



%====================================
\section{Secondary Scintillation}
\label{sec:secondary_scintillation}



%====================================
\section{Particle Interactions}
\label{sec:particles}

\section*{New Section}

By using the asterisk to start a new section, I keep the section from appearing in the table of contents.
If you want your sections to be numbered and to appear in the table of contents, remove the asterisk.


%%This is the third chapter of the dissertation

%The following command starts your chapter. If you want different titles used in your ToC and at the top of the page throughout the chapter, you can specify those values here. Since Columbia doesn't want extra information in the headers and footers, the "Top of Page Title" value won't actually appear.

\pagestyle{cu}
\graphicspath{{./Chapter3/Figures/}}
\chapter[The XENON1T Dark Matter Search][The XENON1T Dark Matter Search]{The XENON1T Dark Matter Search}

% https://arxiv.org/pdf/1801.07231.pdf for electron emission from wires


XENON1T is the third generation experiment of the XENON collaboration.  With a fiducial mass of $> 1000\ \mathrm{kg}$ it is the first
liquid xenon dark matter detector to reach the ton-scale era of DM detection.  Its large target mass and low radioactive background
makes it the most sensitive detector to spin-independent WIMPs.

In this chapter I describe the XENON1T experiment (\secref{sec:xenon1t_detector}) and give the results of the second science run
(\secref{sec:xenon1t_sr1}).

Lots of good info in Aprile2017b (instrument paper).

\section{The XENON1T Detector}
\label{sec:xenon1t_detector}




\subsection{PMTs}
\label{subsec:xenon1t_pmts}
A total of 248 Hamamatsu R11410-21 PMTs are installed in XENON1T.  The 127 PMTs in the top array are placed in a radial distribution to
maximize resolution of $r$ position reconstruction.  The 121 in the bottom array are packed as densely as possible to maximize light
collection.  The R11410 window is 76.2 mm in diameter and the photocathode yields an average QE to 178 nm of 34.5\% with 2.8\%
standard deviation (\citeref{Aprile2017b, Barrow2017}).  The high QE results from preselecting PMTs with $\mathrm{QE} > 28\%$ for
screening.

PMTs with the highest QE are placed in the bottom array while those with the lowest are stationed along the outside of the
top.  The difference in arrangement is strategic.  Due to liquid xenon's relatively large dielectric constant (1.95) an S1 will
often reflect off the surface and be redirected towards the bottom of the TPC.  For low-energy events - the relevant range for WIMP DM
searches - a nuclear recoils may only emit a small number ($\lesssim 100$) of photons, many of which never reach the PMTs.  Thus it is
most advantageous to position those with the highest quantum efficiency in the region most likely to see scintillation from an
S1.  Likewise, S2s easily produce enough scintillation to be observed by both arrays.  Therefore the QE of the top PMTs is comparatively
unimportant, and may even be advantageous for larger S2s where saturation can occur.

\begin{figure}
\centering
\includegraphics[width=\textwidth]{PMTQuantumEfficiency}
\caption{Quantum efficiency of top (left) and bottom (right) PMT arrays.  PMTs with highest QE are placed in the center of the bottom
array to maximize light collection while those with the lowest are placed in the outer region of the top.  Image credit:
\citeref{Aprile2017b}.}
\label{fig:xenon1t_pmt_qe}
\end{figure}

\begin{figure}
\centering
\includegraphics[width=0.8\textwidth]{Fig1FromAprile2015}
\label{fig:xenon1t_hamamatsu_pmt}
\end{figure}

The R11410-21 has 12 dynodes following the focusing electrode disk.  The first dynode is the largest and extends to the electrode to
maximize the probability of capturing photoelectrons.  A schematic can be be seen in \figref{fig:xenon1t_hamamatsu_pmt}.  The electrode,
dynode, and shield are stainless steel and are insulated with L-shaped quartz plates.  The window is
also made of quartz, since it is transparent to vacuum ultraviolet (VUV) photons.  Deposited on it is a low-temperature bialkali
photocathode.  The window is fixed with an aluminum seal to the faceplate flange, which along with the stem flange is constructed from
Kovar.  Because of the PMT body's large mass (71\% of total Kovar, 35\% of total) a low-\ce{^{60}Co} Kovar is chosen.  Finally, to insulate
the connections to each dynode the stem is ceramic.

Because radioactivity limits the fiducial volume and increases the event rate, making accidental coincidence and outlier events more
likely, XENON and Hamamatsu worked together to develop a highly radio-pure PMT.  There were several iterations of the R11410 model before
the R11410-21 was determined to be adequate.  Nearly all the \ce{^{137}Cs} and \ce{^{60}Co} comes from the Kovar, though the \ce{^{137}Cs}
content is negligible and the \ce{^{60}Co} is 3-10 times lower than older models.  The remaining screened isotopes, \ce{^{238}U},
\ce{^{228}Th}, \ce{^{228}Ra}, \ce{^{226}Ra}, and \ce{^{40}K}, are dominated by the ceramic stem (\citeref{Aprile2015}).  Unfortunately a
material that is more radio-pure and can insulate the dynode connections has not been found.  Sapphire was used in an iteration but
ultimately showed any improvement was minimal.

The dark count rate, or the number of signals per second above a threshold without a light source, is an important property to
characterize.  At ambient temperatures the primary cause is thermal electrons that scale with PMT voltage.  This becomes subdominant at
cryogenic temperatures to electron field emission and radioactivity (internal and external) as well as cosmic rays.  In a detector such
as XENON1T higher dark count rates make accidental coincidence more likely, which produces fake additional background and in the worst case
can place fake events in the signal region.  Because the rate is dependent on the threshold it can effectively be tuned.  However, because
for DM search we would like as low of a threshold as possible, choosing PMTs with low dark count rate is essential.

Another problematic feature is light emission from the phototube itself.  It has been observed to mainly occur in one of two ways.  The
first is through a discharge of intense light that can last for several seconds.  This so-called ``flash" is easily observable to other
PMTs that are facing it so identifying such a flash is simple.



An additional six Hamamatsu
R8520 PMTs reside in LXe outside the TPC near the top electrode for studying calibrations.  These PMTs have been used in a number of
LXe TPCs including XENON100, the predecessor to XENON1T (\citeref{Goetzke2017}, see \citeref{Aprile2012} for details on XENON100).

PMTs contribute a large amount of radioactivity, details in Aprile2015.

\begin{figure}
    \centering
    \begin{subfigure}[t]{0.45\textwidth}
        \centering
        \includegraphics[height=4.5cm]{PMTTopArray}
    \end{subfigure}%
    \begin{subfigure}[t]{0.45\textwidth}
        \centering
        \includegraphics[height=4.5cm]{PMTBottomArray}
    \end{subfigure}
    \caption{XENON1T top (left) and bottom (right) PMT arrays.  Top PMTs are installed inside the diving bell in a radial distribution
    to minimize uncertainty in radial position reconstruction.  Bottom PMTs are installed below the cathode and screening mesh that
    limits interference between the PMT and cathode electric fields, both of which can be seen in the figure.  They are packed tightly
    together to maximize lightcollection.  Image credit: \citeref{Aprile2017b}.}
	\label{fig:xenon1t_pmt_array}
\end{figure}

\begin{figure}
\centering
\includegraphics[width=0.8\textwidth]{Fig4Barrow2017}
\caption{(Left) Single photoelectron spectrum.}
\label{fig:xenon1t_pmt_spe}
\end{figure}




\subsection{TPC}
\label{subsec:xenon1t_tpc}

\begin{figure}
\centering
\includegraphics[width=0.8\textwidth]{XENON1TTPC}
\label{fig:xenon1t_tpc_tpc}
\end{figure}

\begin{figure}
\centering
\incluegraphics[width=0.6\textwidth]{Fig3Aprile2017b}
\label{fig:xenon1t_tpc_efield}
\end{figure}



\begin{figure}
\includegraphics[\width=0.8\textwidth]{AbsorptionSpectra}
\caption{\citeref{Ozone2005}}
\end{figure}


In an electric field $E$ an \electron that is freed but does not recombine with its parent or other ionized atoms will move anti-parallel
to the field at drift velocity $v_{d}$.  For $E \lesssim 100\ \mathrm{V\ cm^{-1}}$ \vd$\propto E$, $100 \lesssim E \lesssim 10^{3-4}$
\vd$\propto E^{1/2}$, and $E \gtrsim 10^{4}$ \vd plateaus at $\sim 3\ \mathrm{mm\ \mu s^{-1}}$ (\citeref{Miller1968}).

\begin{table}
 \centering
 \begin{tabular}{cc}
 \hline
 $E$ [V cm$^{-1}$] & \vd [mm $\mu$s$^{-1}$] \\
 \hline
 $\lesssim 100$ & \vd$\propto E$ \\
 $\sim 100-10^{3-4}$ & \vd$\propto E^{1/2}$ \\
 $\gtrsim 10^{4}$ & \vd$\sim 3$ \\
 \hline
 \caption{Drift velocity \vd as a function of electric field $E$ for LXe}
 \end{tabular}
\end{table}

\begin{figure}
\includegraphics[angle=0.5, width=0.8\textwidth]{DriftVelocity}
\caption{Drift velocity for solid and liquid xenon}
\label{fig:drift_velocity}
\end{figure}

As the electron cloud drifts it will diffuse both longitudinally (in the direction of $E$) and transversely (perpendicular to $E$).  The
diffusion coefficients $D_{L}$ and $D_{T}$ are dependent on the electric field with $D_{T}/D_{L} \sim 10$.  The electron spread can
be written as $\sigma_{D_{T}} = \sqrt{D_{T} t_{d}}$ where $t_{d} = d/v_{d}$ is the drift time and $d$ is the drift distance.

Extensive xenon distillation and purification occurs before it is used in a detector.  Nonetheless impurities outgas from detector
material and contaminate the LXe.  Electronegative impurities in particular present a problem since they will attach to a free \electron,
lowering the number that reach the top of the detector and decreasing the secondary scintillation as shown in \eqref{eq:impurity_attach}.

\begin{equation}
e^{-} + S \rightarrow S^{-}
\label{eq:impurity_attach}
\end{equation}

\noindent The amount of \electron captured is dependent on the time in the LXe.  Thus an advantage of larger \efields is a larger
\vd (up to a point) and thus less time in the liquid.  Doping LXe with organic materials such as butane can increase \vd at higher
\efields but they are not used in DM detectors due to difficulty in purifying (\citeref{Yoshino1976}).  By setting the rate at which
electrons are absorbed by impurities $dq/dt = -qk_{S}S$ where $S$ is the impurity concentration and $k_{S}$ is the attachment rate
constant we find

\begin{equation}
q(t) = q_{0}e^{-tk_{S}S} = q_{0}e^{-t/\tau_{e}}
\label{eq:lifetime_equation}
\end{equation}

\noindent where $\tau_{e} = (k_{S}S)^{-1}$ and is known as the electron lifetime.  $k_{S}$ is shown in \figref{fig:attachment_rate} for
O$_{2}$,
N$_{2}$O, and SF$_{6}$.  We see that for N$_{2}$O the attaching rate constant increases with \efield whereas \otwo and SF$_{6}$
decerase.  Typically impurity concentration is given in O$_{2}$-equivalent values - that is, the concentration of \otwo if it was solely
responsible for \electron attachment.  For modeling electron lifetime it turns out that using the \otwo curve in
\figref{fig:attachment_rate} gives a good approximation.  Removing such impurities will be discussed in detail in \secref{}.

\begin{figure}
\includegraphics[width=0.8\textwidth]{AttachmentRate}
\caption{Attaching rate constant $k_{S}$ from \citeref{Bakale1976} for \otwo, N$_{2}$O, and SF$_{6}$ with respect to electric field.  At
larger \efield $k_{S}$ increases for N$_{2}$O and decreases for \otwo and SF$_{6}$.}
\label{fig:attachment_rate}
\end{figure}

In a TPC a cathode at the bottom of the detector applies an electric field in the LXe.  The \electron drift towards the top where a
grounded gate rests a few millimeters below the LXe surface.  Directly above the gate by a couple centimeters is the anode, which
applies a strong electric field that extracts the electrons into the gas xenon (GXe).  An extracted electron will ionize and excite
GXe atoms, whose freed electrons will do so as well in what is known as electroluminescence.  The number of ionized and excited atoms
is proportional to the number of \electron extracted, hence it is also known as proportional scintillation.  The number of photons
$N_{\mathrm{ph}}$ produced traveling a distance $z$ is

\begin{equation}
\frac{dN_{\mathrm{ph}}}{dz} = \alpha \Big( \frac{E_{g}}{P} - \beta \Big) P
\label{eq:electronlum}
\end{equation}

\noindent where $\alpha = 70\ \mathrm{photons\ kV^{-1}}$, $\beta = 1.0\ \mathrm{kV\ cm^{-1}\ atm^{-1}}$, and $E_{g}$ and $P$ are the
GXe electric field and pressure, respectively (\citeref{Belogurov1995}).

For PMT use Fig. 1 of Aprile2015
%This is the fourth chapter of the dissertation

%The following command starts your chapter. If you want different titles used in your ToC and at the top of the page throughout the chapter, you can specify those values here. Since Columbia doesn't want extra information in the headers and footers, the "Top of Page Title" value won't actually appear.

\pagestyle{cu}
\graphicspath{{./Chapter4/Figures/}}
\chapter[Purity and the Electron Lifetime][Purity and the Electron Lifetime]{Purity and the Electron Lifetime}

In a noble element dark matter detection experiment the purity of the target mass is an essential consideration and must be measured
continuously.  To date the concentration of electronegative impurities has always been measured in these experiments, but no reliable
model has existed to explain and predict its behavior.

In this chapter I discuss the necessity of extremely pure xenon (\secref{}), explain the original model fit to XENON1T data
(\secref{}), and examine how abrupt changes in detector conditions alter the contamination (\secref{}).



\section{Importance and Procedure for Purifying Xenon}
\secref{sec:importance_procedure}
Purity usually refers to two distinct but correlated values, though the degree of the correlation can depend on the
experiment.  The first is radioactive elements of other noble elements that cannot be completely removed during distillation.  For xenon
our primary challenges are \ce{^{85}Kr} (\secref{subsubsec:backgrounds_electronic_krypton}) and \ce{^{222}Rn}
(\secref{subsubsec:backgrounds_electronic_radon}) as they have low-energy decays that can contaminate our region of interest (while
\ce{^{220}Rn} also leads to a low-energy \betadecay its half-life is too short to penetrate our detector and thus can be ignored).

The second consideration with regards to detector purity is contamination of electronegative impurities such as \ce{O_2} or
\ce{N_2}.  These attach to drifting electrons, lowering or even eliminating the S2.  This can have the largest impact at low energies
since the number of \electron is much fewer.  To correct for the expected initial number of electrons we can use the electron lifetime
$\tau_{\mathrm{e^-}}$, though this must be monitored consistently if not perpetually.  Of course, if the entire cloud of electrons is
removed by these impurities we cannot apply a correction since we have no knowledge of where in the detector it occurred or the energy
deposition.

This chapter is focused on the latter of these two purities, though its examination necessitates consideration of the former as we will
see.
%\include{./Conclusion/Conclusion}

%This final section includes your bibliography.



%\backmatter

\SingleSpacing %Start single-spacing text before you start the bibliography. We used \bibitemsep earlier in this document to keep bibliography items separated by one line of blank space, but we need to keep the entries themselves single-spaced.
\printbibliography %Print the bibliography. Your bibliography file is defined as Bibliography.bib earlier in this document by the command \addbibresource. It should be kept in the same folder as this file.

%\appendix

\setstretch{1.5}
%\include{./AppendixA/AppendixA}
%\include{./AppendixB/AppendixB}

\end{document} %All done! Now you're a doctor.