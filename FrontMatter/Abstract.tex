% This is the abstract of my dissertation.

\pagestyle{empty} % No page number in entire abstract
\begin{center}
  ABSTRACT

The XENON1T Spin-Independent WIMP Dark Matter Search Results and a Model to Characterize the Reduction of Electronegative Impurities in
its 3.2 Tonne Liquid Xenon Detector

Zachary Schuyler Greene
\end{center}

Over much of the last century evidence has been building for a new component of our universe that interacts primarily through
gravitation.  Known as cold dark matter, this non-luminous source is predicted to constitute 83\% of matter and 26\% of mass-energy in the
universe.  Experiments are currently searching for dark matter via its possible creation in particle colliders, annihilation in
high-density regions of the universe, and interactions with Standard Model particles.  So far dark matter has eluded detection so its
composition and properties remain a mystery.

Weakly interacting massive particles (WIMPs) are hypothetical elementary particles that interact on the scale of the weak nuclear
force.  They naturally satisfy predictions from extensions of the Standard Model, and are one of the most favored dark matter
candidates.  A number of direct detection experiments dedicated to measuring their predicted interactions with atomic nuclei have been
constructed over the last 25 years.

Liquid xenon dual phase time projection chambers (TPCs) have led the field for spin-independent WIMP searches at WIMP masses of
${>}\, 10\ \mathrm{GeV/c^2}$ for most of the last decade.  XENON1T is the first tonne-scale TPC, and with 278.8 days of dark matter data
has set the strictest limits on WIMP-nucleon interaction cross sections above WIMP masses of $6\ \mathrm{GeV/c^2}$, with a minimum of
$4.1 \times 10^{-47}\ \mathrm{cm^2}$ at $30\ \mathrm{GeV/c^2}$.  XENON1T and the analysis that led to this
result are discussed, with an emphasis on electronic and nuclear recoil calibration fits, which help discriminate between background and
WIMP-like events.

Interactions in liquid xenon produce light and charge that are measured in TPCs.  These signals are attenuated by electronegative
impurities including \ce{O_2} and \ce{H_2O}, which are homogeneously distributed throughout the liquid xenon.  The decrease in observables
enlarges the uncertainty in our analysis, and can decrease our sensitivity.  Methods on measuring the
charge loss are presented, and a physics model that describes the behavior of the electronegative impurity concentration over the
lifetime of XENON1T is derived.  The model is shown to successfully explain the more than two years of data.