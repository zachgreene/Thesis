% This is the abstract of my dissertation.

\pagestyle{empty} % No page number in entire abstract
\begin{center}
  ABSTRACT

The XENON1T Spin-Independent WIMP Dark Matter Search Results and a Model to Characterize the Reduction of Electronegative Impurities in
its 3.2 Tonne Liquid Xenon Detector

Zachary Schuyler Greene
\end{center}

Over much of the last century evidence has been building for a new component of our universe that interacts primarily through
gravitation.  Known as cold dark matter, this non-luminous source is predicted to constitute 83\% of matter and 26\% of mass-energy in the
universe.  Experiments are currently searching for dark matter via its possible creation in particle colliders, annihilation in
high-density regions of the universe, and interactions with Standard Model particles.  So far dark matter has eluded direct detection so its
composition and properties remain a mystery.

Weakly interaction massive particles (WIMPs) are hypothetical elementary particles that interact on the scale of the weak nuclear
force.  They naturally satisfy predictions from extensions of the Standard Model and are one of the most favored dark matter
candidates.  A number of direct detection experiments dedicated to observing their predicted interactions with atomic nuclei have been
constructed over the last 25 years.

Liquid xenon dual phase time projection chambers (TPCs) have led the field for spin-independent WIMP searches at
${>}\, 10\ \mathrm{GeV/c^2}$ for most
of the last decade.  XENON1T is the first tonne-scale TPC.

Electronegative impurities in liquid xenon attenuate light and charge, decreasing the signals in our detector.  A physics model is
presented that describes the behavior of the electronegative impurities over the lifetime of XENON1T.